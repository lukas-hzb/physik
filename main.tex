% Schlichtes und übersichtliches LaTeX-Layout
\documentclass[a4paper, 11pt]{article}

% --- Grundlegende Pakete ---
\usepackage[utf8]{inputenc}    % Zeichencodierung
\usepackage[T1]{fontenc}       % Schriftart-Kodierung
\usepackage[ngerman]{babel}    % Deutsche Sprache
\usepackage{csquotes}
\usepackage{lmodern}           % Moderne Schriftart
\usepackage{microtype}         % Mikrotypografie-Verbesserungen
\usepackage{parskip}           % Absätze durch Leerzeilen statt Einrückung
\usepackage[margin=2.5cm]{geometry} % Seitenränder

% --- Weitere nützliche Pakete ---
\usepackage{amsmath, amssymb}  % Mathematische Symbole und Formeln
\usepackage{physics}
\usepackage{siunitx}
\usepackage{enumitem}          % Weitere Art der Aufzählung
\usepackage{booktabs}          % Schöne Tabellen
\usepackage{graphicx}          % Bilder einbinden
\usepackage{xcolor}            % Farbige Schrift
\PassOptionsToPackage{hyphens}{url}\usepackage{hyperref} % Hyperlinks im PDF

% --- Zusätzliche Pakete für verbesserte Darstellung ---
\usepackage{tikz}              % Für Diagramme und Zeichnungen
\usetikzlibrary{calc}          % Für komplexe Berechnungen in TikZ
\usepackage{tcolorbox}         % Für farbige Boxen
\usepackage{multicol}          % Für mehrspaltige Layouts
\usepackage{float}             % Für eine bessere Platzierung von Figuren
\usepackage{xurl}              % Links werden immer gebrochen

% --- Anpassungen ---
\hypersetup{
    unicode=true,        % Allows non-ASCII characters in bookmarks
    colorlinks=true,
    linkcolor=blue!50!black,
    filecolor=blue,
    urlcolor=blue,
    citecolor=blue,
    pdftitle={Physik Lernzettel Klausur 4 - Magnetismus},
    pdfauthor={Lukas Harzbecker}
}

\setlist[itemize]{label=--, left=0.55em}    % Alternative Listen

\setlength{\parindent}{0pt}    % Keine Einrückung bei Absätzen

% --- Kopf- und Fußzeilen ---
\usepackage{fancyhdr}

\fancypagestyle{titleandtocstyle}{
    \fancyhf{}
    \fancyfoot[C]{} % Leerer Fußbereich für keine Seitenzahl
    \renewcommand{\headrulewidth}{0pt}
    \renewcommand{\footrulewidth}{0pt}
}

\fancypagestyle{contentstyle}{
    \fancyhf{}
    \fancyhead[L]{Lukas Harzbecker}
    \fancyhead[C]{Physik-LK -- Alle Themen}
    \fancyhead[R]{\rightmark}
    \fancyfoot[C]{\thepage}
    \renewcommand{\headrulewidth}{0.5pt}
    \renewcommand{\footrulewidth}{0pt}
}

% Section-Kurztitel fuer Header setzen (ohne Nummer)
\renewcommand{\sectionmark}[1]{\markright{#1}}
\renewcommand{\subsectionmark}[1]{}
\renewcommand{\subsubsectionmark}[1]{}

% --- Titelseite ---
\author{Lukas Harzbecker}
\date{\today}


\begin{document}

% --- Titelseite und Inhaltsverzeichnis ---

\pagestyle{empty}       % Keine Seitenzahlen im TOC

\title{Physik-Leistungskurs – Wiederholung aller Themen}

\maketitle
\thispagestyle{empty}   % Keine Seitenzahl auf Titelseite

\tableofcontents

\clearpage              % Startet neuen Abschnitt
\pagenumbering{arabic}  % Setzt Seitenzahlen auf arabisch
\setcounter{page}{1}    % Startet mit 1 auf der ersten Inhaltsseite

\pagestyle{contentstyle}

% --- Inhalt ---

\newpage
\section{Einleitung}
Dieses Dokument enthält detaillierte Herleitungen zentraler Größen und Formeln für den Physik-Leistungskurs in Baden-Württemberg (Abitur 2026), beginnend ab der elften Klasse. Darüber hinaus werden wichtige Phänomene sowie die Zusammenhänge zwischen verschiedenen Themengebieten erläutert. Die Kenntnisse aus Unter- und Mittelstufe werden dabei vorausgesetzt und nicht erneut behandelt.
Bei Rechnungen gilt – sofern nicht ausdrücklich anders angegeben – die Konvention, dass der wissenschaftliche Taschenrechner (WTR) auf den Bogenmaß-Modus (Radian) eingestellt sein sollte. \par
% --
Zusätzliche Erklärungen zu den einzelnen Themengebieten und weiterführende Informationen sind auf der Webseite \url{https://physikbuch.schule/index.html/} zu finden. \par

\newpage
\section{Schwingungstheorie}
Die nachfolgenden Grundlagen der Schwingungstheorie wurden in der ersten Klausur der elften Klasse abgefragt.

\subsection{Relevante Größen und deren Zusammenhänge}

\subsubsection{Frequenz ($f$) und Periodendauer ($T$)}
\label{sec:frequenz_und_periodendauer}
Die Frequenz $f$ gibt die Anzahl der Wiederholungen eines periodischen Vorgangs pro Sekunde an. Die Periodendauer $T$ beschreibt die Zeit, die ein periodischer Vorgang für eine vollständige Wiederholung benötigt.
Die Herleitung von $T$ und $f$ erfolgt zunächst über die \hyperref[sec:winkelgeschwindigkeit_und_kreisfrequenz]{Kreisfrequenz} $\omega$, die selbst hergeleitet werden muss.
\begin{gather}
    f = \frac{1}{T}
\end{gather}
Die Periodendauer eines Federpendels $T_\mathrm{fedp}$ hängt von der Masse $m$ des Körpers und der Federkonstanten $D$ ab. Nach dem Hooke'schen Gesetz ($F_\mathrm{rück} = -D \cdot s$) ist sie unabhängig von der Auslenkung. Sie berechnet sich zu:
\begin{gather}
    T_\mathrm{fedp} = 2 \pi \sqrt{\frac{m}{D}}
\end{gather}
\begin{center}
    \begin{tabular}{rl}
        $m$ & Masse des schwingenden Körpers \\
        $D$ & Federkonstante der Feder mit $[D] = \si{\newton\per\meter}$
    \end{tabular}
\end{center}
Die Periodendauer eines Fadenpendels $T_\mathrm{fadp}$ ist für kleine Auslenkungen ($\varphi \lesssim 10^\circ$) unabhängig von der Masse $m$ und der horizontalen Auslenkung $s_\mathrm{H}$. Sie wird näherungsweise berechnet durch:
\begin{gather}
    T_\mathrm{fadp} = 2 \pi \sqrt{\frac{l}{g}}
\end{gather}
\begin{center}
    \begin{tabular}{rl}
        $l$ & Länge des Fadens \\
        $g$ & Erdbeschleunigung ($\approx 9{,}81\,\mathrm{m/s^2}$)
    \end{tabular}
\end{center}
Die Einheiten der Frequenz und der Periodendauer sind
\begin{center}
    $[f] = \si{\hertz} = \mathrm{Hertz} = \si{\per\second}$ \\
    $[T] = \si{\second}$
\end{center}

\subsubsection{Winkelgeschwindigkeit ($\omega$) und Kreisfrequenz ($\omega_0$)}
\label{sec:winkelgeschwindigkeit_und_kreisfrequenz}
Die Winkelgeschwindigkeit $\omega$ ist die momentane Änderungsrate eines Winkels $\varphi$ und beschreibt die Rotationsgeschwindigkeit. Die Kreisfrequenz $\omega_0$ ist eine charakteristische, meist konstante Größe eines harmonischen Oszillators, die dessen freie Schwingungsrate bestimmt. Obwohl bei vielen linearen Schwingungen kein geometrischer Winkel im Sinn einer wirklichen Kreisbewegung vorliegt, wird $\omega$ als analoge Winkelrate verwendet, weil sich die zeitliche Entwicklung der Schwingung als Projektion einer Kreisbewegung darstellen lässt. Die beiden Begriffe werden im Kontext der Schwingungslehre trotz der Unterschiede oft synonym verwendet.

Mithilfe der Winkelgeschwindigkeit lassen sich die Periodendauer $T$ und die Frequenz $f$ herleiten.
\begin{gather}
    \omega = \frac{d\varphi}{dt} = 2\pi \cdot f
\end{gather}
Für die Kreisfrequenz $\omega_{0,\,\mathrm{fedp}}$ bei Federpendeln gilt nach der Herleitung unter~\ref{sec:federpendel}:
\begin{gather}
    \omega_{0,\,\mathrm{fedp}} = \sqrt{\frac{D}{m}}
\end{gather}
\begin{center}
    \begin{tabular}{rl}
        $D$ & Federkonstante der Feder mit $[D] = \si{\newton\per\meter}$ \\
        $m$ & Masse des schwingenden Körpers
    \end{tabular}
\end{center}
Die Kreisfrequenz eines Fadenpendels $\omega_{0,\,\mathrm{fadp}}$ ergibt sich für kleine Auslenkungen ($\varphi \lesssim 10^\circ$) aus der Herleitung unter~\ref{sec:fadenpendel}:
\begin{gather}
    \omega_{0,\,\mathrm{fadp}} = \sqrt{\frac{g}{l}}
\end{gather}
\begin{center}
    \begin{tabular}{rl}
        $l$ & Länge des Pendels \\
        $g$ & Erdbeschleunigung ($\approx 9{,}81\,\mathrm{m/s^2}$)
    \end{tabular}
\end{center}
Die Einheit der Winkelgeschwindigkeit und Kreisfrequenz ist
\begin{center}
    $[\omega] = [\omega_0] = \si{\per\second} = \si{\radian\per\second}$
\end{center}

\subsubsection{Rückstellkraft ($F_\mathrm{rück}$)}
\label{sec:rueckstellkraft}
Die Rückstellkraft $F_\mathrm{rück}$ wirkt auf eine Masse in einem harmonischen Oszillator, die aus ihrer Ruhelage ausgelenkt wird. Nach dem \hyperref[sec:hookesches_gesetz]{Hooke'schen Gesetz} ist sie proportional zur Auslenkung $s$.
\begin{gather}
    F_\mathrm{rück} = -D \cdot s
\end{gather}
\begin{center}
    \begin{tabular}{rl}
        $D$ & Federkonstante der Feder mit $[D] = \si{\newton\per\meter}$ \\
        $s$ & Auslenkung der Feder
    \end{tabular}
\end{center}
Das Minuszeichen zeigt an, dass die Kraft stets in Richtung der Ruhelage wirkt und die Masse dorthin zurückführt.

\subsubsection{Schwingungsgleichung mit Elongation ($s$), Amplitude ($\hat{s}_0$) und Phase ($\varphi_0$)}
\label{sec:schwingungsgleichung}
Die Schwingungsgleichung $s(t)$ beschreibt die zeitliche Entwicklung der Auslenkung eines schwingungsfähigen Systems infolge der rücktreibenden Kraft und lässt sich in einem s-t-Diagramm darstellen. $s(t)$ ist dabei die Elongation des Oszillators. Die allgemeine Form gilt nur für \hyperref[sec:harmonische_schwingung]{harmonische Schwingungen}. \hyperref[sec:gedaempfte_schwingung]{Gedämpfte Schwingungen} müssen gesondert behandelt werden.
\begin{align}
    s(t) &= \hat{s}_0 \cdot \sin(\omega t + \varphi_0) \\
    &= \hat{s}_0 \cdot \sin(2\pi f \, t + \varphi_0)
\end{align}
\begin{center}
    \begin{tabular}{rl}
        $\hat{s}_0$ & Amplitude (maximale Auslenkung) \\
        $\varphi_0$ & Anfangsphase
    \end{tabular}
\end{center}
Die konkrete Wahl von Sinus oder Kosinus und deren Vorzeichen hängt von den Anfangsbedingungen ab. Startet der Oszillator in der Ruhelage, eignet sich die Sinusfunktion. Beginnt er mit maximaler Auslenkung, wird die Kosinusfunktion verwendet. Liegt die Anfangsphase zwischen diesen Extremfällen, wird sie als Phasenwinkel $\varphi_0$ in die Gleichung eingesetzt.

Die ersten beiden zeitlichen Ableitungen der Elongation entsprechen der Geschwindigkeit und Beschleunigung des schwingenden Körpers:
\begin{gather}
    v(t) = \dot{s}(t) \\
    a(t) = \dot{v}(t) = \ddot{s}(t)
\end{gather}
Die Schwingungsgleichung erfüllt die lineare Differentialgleichung $\ddot{s}(t) + \omega^2 s(t) = 0$, die das grundlegende Modell einer ungedämpften harmonischen Schwingung darstellt.

\medskip
\begin{tcolorbox}[colframe=blue!30!gray, colback=blue!10, title=Vertiefung: Herleitung der Schwingungs-Differenzialgleichung]
    Betrachten wir ein Federpendel: Eine Masse $m$ ist an einer idealen Feder mit Federkonstante $D$ aufgehängt und kann sich reibungsfrei bewegen. Die Ruhelage wird bei $s=0$ festgelegt.

    Bei einer Auslenkung $s$ wirkt nach dem Hooke'schen Gesetz die rücktreibende Kraft
    \begin{gather*}
        F_\mathrm{rück,\,fed} = -D\,s.
    \end{gather*}
    Nach dem zweiten Newton'schen Gesetz gilt für die Summe aller Einzelkräfte
    \begin{gather*}
        F_\mathrm{sum} = m \cdot \ddot{s} = F_\mathrm{rück,\,fed}.
    \end{gather*}
    Damit folgt die lineare Differentialgleichung
    \begin{gather*}
        m \ddot{s} + D s = 0
        \quad \Rightarrow \quad
        \ddot{s} + \frac{D}{m}\,s = 0.
    \end{gather*}
    Mit der Definition $\omega = \sqrt{\tfrac{D}{m}}$ erhält man die normierte Form
    \begin{gather*}
        \ddot{s} + \omega^2 s = 0.
    \end{gather*}
    Ihre allgemeine Lösung ist eine harmonische Schwingung:
    \begin{gather*}
        s(t) = \hat{s}_0 \sin(\omega t + \varphi).
    \end{gather*}
\end{tcolorbox}

\subsubsection{Kinetische Energie ($W_\mathrm{kin}$)}
\label{sec:kinetische_energie}
Die kinetische Energie $W_\mathrm{kin}$ ist die Bewegungsenergie eines Körpers. Sie hängt von seiner Masse $m$ und seiner Geschwindigkeit $v$ ab.
\begin{gather}
    W_\mathrm{kin} = \tfrac{1}{2} m v^2
\end{gather}
\begin{center}
    \begin{tabular}{rl}
        $m$ & Masse des Körpers \\
        $v$ & Geschwindigkeit des Körpers
    \end{tabular}
\end{center}

\subsubsection{Spannenergie ($W_\mathrm{span}$)}
\label{sec:spannenergie}
Die Spannenergie $W_\mathrm{span}$ ist die in einer Feder gespeicherte potenzielle Energie, die durch ihre Auslenkung $s$ entsteht.
\begin{gather}
    W_\mathrm{span} = \tfrac{1}{2} D s^2
\end{gather}
\begin{center}
    \begin{tabular}{rl}
        $D$ & Federkonstante der Feder mit $[D] = \si{\newton\per\meter}$ \\
        $s$ & Auslenkung der Feder
    \end{tabular}
\end{center}

\subsection{Wichtige Konzepte und Vertiefung}

\subsubsection{Schwingung}
\label{sec:schwingung}
Eine Schwingung ist eine zeitlich periodische Hin- und Herbewegung eines Systems um eine stabile Gleichgewichtslage. Dabei führt die Trägheit des Körpers dazu, dass er die Gleichgewichtslage wiederholt überschreitet.

\subsubsection{Hooke'sches Gesetz}
\label{sec:hookesches_gesetz}
Das Hooke'sche Gesetz beschreibt, dass die Rückstellkraft $F_\mathrm{rück}$ eines elastischen Körpers proportional zu seiner Auslenkung $s$ aus der Gleichgewichtslage ist, wenn es sich bei dem Aufbau um einen harmonischen Oszillator handelt.
\begin{gather}
    F_\mathrm{rück} = -D \cdot s
\end{gather}

\subsubsection{Harmonische Schwingung}
\label{sec:harmonische_schwingung}
Harmonische Schwingungen sind zeitlich periodische Bewegungen, bei denen die rücktreibende Kraft proportional zur Auslenkung ist. Ihr Verlauf lässt sich durch eine Sinus- oder Kosinusfunktion darstellen.
Die in Abschnitt~\ref{sec:schwingungsgleichung} definierte Schwingungsgleichung gilt nur für harmonische Schwingungen. In der Praxis müssten stets Reibung und andere Störeinflüsse berücksichtigt werden, die nicht in der Schwingungsgleichung vorkommen.
\begin{itemize}
    \item Bei harmonischen Schwingungen bleibt die Gesamtenergie des Systems konstant, während die Energieformen (kinetische, potenzielle und Spannenergie) zeitlich wechseln.
    \item Ihr Verlauf lässt sich durch eine Sinus- oder Kosinusfunktion darstellen.
    \item Die maximale Auslenkung bleibt konstant.
    \item Die Rückstellkraft ist die einzige wirkende Kraft.
\end{itemize}

\medskip
\begin{tcolorbox}[colframe=red!30!gray, colback=red!10, title=Hinweis: Harmonisierung eines vertikalen Federpendels]
    \label{sec:harmonisierung_vertikales_federpendels}
    Bei einem vertikalen Federpendel wirkt zusätzlich zur Federkraft $F_\mathrm{Feder} = -D s$ die Gewichtskraft $F_\mathrm{g} = m g$ auf die Masse.
    Verschiebt man den Nullpunkt der Auslenkung auf die neue Gleichgewichtslage $s_\mathrm{eq}$, sodass
    \begin{gather*}
        F_\mathrm{rück,\,fed}(s_\mathrm{eq}) + F_\mathrm{g} = 0 \quad \Rightarrow \quad D s_\mathrm{eq} = m g,
    \end{gather*}
    wirken bei aus dieser Lage gemessenen Auslenkungen $\tilde{s} = s - s_\mathrm{eq}$ nur noch durch die Rückstellkraft
    \begin{gather*}
        F_\mathrm{rück} = -D \tilde{s}
    \end{gather*}
    und die Gewichtskraft wirkt nur noch als konstante Verschiebung, nicht mehr als zeitabhängige Kraft.

    Nach dem zweiten Newton'schen Gesetz folgt damit die Differentialgleichung deren allgemeine Lösung eine harmonische Schwingung ist. Somit verhält sich das vertikale Federpendel theoretisch wie ein idealer harmonischer Oszillator, solange Reibung und nichtlineare Effekte vernachlässigt werden.
\end{tcolorbox}

\subsubsection{Eigenfrequenz ($f_\mathrm{eigen}$)}
\label{sec:eigenfrequenz}
Die Eigenfrequenz ist die Frequenz, mit der ein schwingungsfähiges System ohne äußere Einflüsse aufgrund seiner eigenen physikalischen Eigenschaften frei schwingt.
Bei einem Federpendel hängt sie beispielsweise von der Masse $m$ und der Federkonstante $D$ ab:
\begin{gather}
    f_\mathrm{eigen} = \frac{\omega_0}{2\pi} = \frac{1}{2\pi} \sqrt{\frac{D}{m}}.
\end{gather}
\begin{center}
    \begin{tabular}{rl}
        $D$ & Federkonstante der Feder mit $[D] = \si{\newton\per\meter}$ \\
        $m$ & Masse des schwingenden Körpers
    \end{tabular}
\end{center}

\subsubsection{Erzwungene Schwingung}
\label{sec:erzwungene_schwingung}
Eine erzwungene Schwingung ist eine zeitlich periodische Bewegung eines schwingungsfähigen Systems, die durch eine kontinuierlich von außen wirkende Kraft aufrechterhalten wird.
Die Frequenz der Schwingung entspricht dabei der Frequenz der äußeren Anregung.
Im Gegensatz zur freien Schwingung hängt das System hier nicht nur von seinen inneren Eigenschaften, sondern von der äußeren Kraft ab.

\subsubsection{Erregerfrequenz ($f_\mathrm{erreg}$)}
\label{sec:erregerfrequenz}
Die Erregerfrequenz ist die Frequenz einer äußeren Kraft oder Anregung, die ein schwingungsfähiges System zum Schwingen bringt.
Bei erzwungenen Schwingungen bestimmt die Erregerfrequenz die Frequenz der resultierenden Schwingung.
Liegt sie nahe an der Eigenfrequenz des Systems, kann es zur \hyperref[sec:resonanz_und_resonazkatastrophe]{Resonanz oder zu einer Resonanzkatastrophe} kommen, bei der die Amplitude stark ansteigt.

\subsubsection{Resonanz und Resonanzkatastrophe}
\label{sec:resonanz_und_resonazkatastrophe}
Resonanz ist der Effekt, dass ein schwingungsfähiges System besonders stark auf eine äußere Anregung reagiert, wenn deren Frequenz nahe der Eigenfrequenz des Systems liegt.
Als Resonanzkatastrophe bezeichnet man die extrem starke, oft zerstörerische Schwingung, die auftritt, wenn die Anregungsfrequenz exakt der Eigenfrequenz entspricht und keine Dämpfung vorhanden ist.

Die Eigenfrequenz eines Systems wird ausschließlich von seinen inneren Eigenschaften wie Masse und Federkonstante bestimmt.
Wird es durch eine äußere Kraft mit der Erregerfrequenz $f_\mathrm{erreg}$ angeregt, versucht das System, dieser Frequenz zu folgen.
Liegt die Erregerfrequenz weit von der Eigenfrequenz entfernt, bleibt die Amplitude klein, da Energiezufuhr und Schwingung oft gegenphasig wirken.
Stimmen Erreger- und Eigenfrequenz überein, wird bei jeder Schwingung im optimalen Moment Energie zugeführt, wodurch die Amplitude stark anwächst – im Extremfall bis zur Resonanzkatastrophe.

Ein typisches Alltagsbeispiel ist das Anschubsen einer Schaukel: Erfolgt das Anschubsen im richtigen Moment (Erregerfrequenz = Eigenfrequenz), wächst die Auslenkung bei jedem Schub.

\subsubsection{Gedämpfte Schwingung ($s_\mathrm{dämpf}(t)$, $\hat{s}_\mathrm{dämpf}(t)$)}
\label{sec:gedaempfte_schwingung}
Eine gedämpfte Schwingung ist eine Schwingung, bei der die Amplitude mit der Zeit abnimmt, weil Energie durch Reibung, Luftwiderstand oder andere dissipative Kräfte verloren geht.
Die Auslenkung und die zeitabhängige Amplitude lassen sich schreiben als:
\begin{gather}
    s_\mathrm{dämpf}(t) = \hat{s}_0 \, e^{-\lambda t} \, \cos(\omega t), \\
    \hat{s}_\mathrm{dämpf}(t) = \hat{s}_0 \, e^{-\lambda t}
\end{gather}
\begin{center}
    \begin{tabular}{rl}
        $\hat{s}_0$ & Anfangsamplitude \\
        $\lambda$ & Dämpfungskoeffizient
    \end{tabular}
\end{center}
Der Dämpfungskoeffizient $\lambda$ gibt an, wie schnell die Amplitude abnimmt. Mit ihm lässt sich auch die Halbwertszeit $t_{1/2}$ der Amplitude bestimmen:
\begin{gather}
    t_{1/2} = \frac{\ln(2)}{\lambda}
\end{gather}
Die harmonische Schwingung aus Abschnitt~\ref{sec:schwingungsgleichung} wird durch die Exponentialfunktion \(\mathrm{e}^{-\lambda t}\) „eingehüllt", wodurch die Amplitude zeitlich abnimmt.

Bei kleiner Dämpfung ($\lambda \ll \omega_0$) bleibt die Schwingungsfrequenz nahezu unverändert. Bei starker Dämpfung kann die Schwingung kritisch oder überkritisch abklingen. Man unterscheidet drei Fälle:
\begin{itemize}
    \item \emph{Schwingfall:} $\lambda < \omega_0$, das System bleibt oszillierend, die Amplitude nimmt exponentiell ab.
    \item \emph{Grenzfall:} $\lambda = \omega_0$, das System kehrt ohne Überschwingen am schnellsten in die Gleichgewichtslage zurück.
    \item \emph{Kriechfall:} $\lambda > \omega_0$, das System kehrt langsam und ohne Oszillationen in die Gleichgewichtslage zurück.
\end{itemize}
Die Dämpfungskraft kann durch
\begin{gather}
    F_\mathrm{reib} = -k \cdot v
\end{gather}
beschrieben werden, wobei $k$ eine Materialkonstante und $v$ die Geschwindigkeit des Oszillators ist. Dies entspricht einer \emph{geschwindigkeitsproportionalen Reibung}.

\subsubsection{Überlagerung von Schwingungen}
\label{sec:ueberlagerung_von_schwingungen}
Das Superpositionsprinzip besagt, dass sich physikalische Größen wie Schwingungen überlagern, ohne sich gegenseitig zu beeinflussen. Das Ergebnis ist die Summe der Einzelgrößen. Zwei harmonische Schwingungen $s_1(t)$ und $s_2(t)$ überlagern sich zu:
\begin{align}
    s_\mathrm{ges}(t) &= s_1(t) + s_2(t) \\
    &= \hat{s}_1 \sin(\omega t + \varphi_1) + \hat{s}_2 \sin(\omega t + \varphi_2).
\end{align}
Die Phasenverschiebung $\varphi$ beschreibt die zeitliche Verschiebung der Schwingungen zueinander. Eine Phasendifferenz von $2\pi$ entspricht genau einer Periode.
\begin{itemize}
    \item $\varphi = 0$ oder $\varphi = 2\pi$: Die Schwingungen verlaufen gleichphasig.
    \item $\varphi = \pi$: Die Schwingungen verlaufen gegenphasig.
\end{itemize}
Bei der Interferenz von Schwingungen lassen sich zwei Fälle unterscheiden:
\begin{itemize}
    \item \emph{Konstruktive Interferenz:} Die Amplituden addieren sich, wodurch die resultierende Schwingung maximal verstärkt wird.
    \item \emph{Destruktive Interferenz:} Die Amplituden heben sich teilweise oder vollständig auf, wodurch die Schwingung abgeschwächt oder ausgelöscht wird. Ein praktisches Beispiel sind Noise-Cancelling-Kopfhörer, die destruktive Interferenz nutzen, um Umgebungsgeräusche zu reduzieren.
\end{itemize}
Treten zwei Schwingungen mit nahezu gleicher Frequenz $f_1 \approx f_2$ und ähnlicher Amplitude auf, so entsteht eine Schwebung. Die Amplitude schwankt dabei periodisch, wodurch die \enquote{lauten} und \enquote{leisen} Phasen hörbar werden. Ein Beispiel ist:
\begin{gather*}
    s(t) = 2\hat{s} \cdot \cos\left(2\pi \cdot \frac{f_1 - f_2}{2} \cdot t\right) \cdot \sin\left(2\pi \cdot \frac{f_1 + f_2}{2} \cdot t\right)
\end{gather*}
\begin{center}
    \begin{tabular}{rl}
        $f_R = \frac{f_1+f_2}{2}$ & Frequenz der resultierenden Schwingung \\
        $f_\mathrm{schweb} = |f_1 - f_2|$ & Schwebungsfrequenz \\
        $f_S = \frac{|f_1-f_2|}{2}$ & Frequenz der Einhüllenden
    \end{tabular}
\end{center}
Überlagert man zwei Schwingungen in orthogonaler Richtung, entstehen charakteristische Kurvenbilder. Die Bahnkurve im $x$-$y$-Diagramm heißt Lissajous-Figur:
\begin{align*}
    x(t) &= \hat{s}_1 \cdot \sin(2\pi f_1 \cdot t + \varphi_1) \\
    y(t) &= \hat{s}_2 \cdot \sin(2\pi f_2 \cdot t + \varphi_2)
\end{align*}
Bei rationalem Frequenzverhältnis $\left(\frac{f_1}{f_2} = \frac{m}{n}\right)$ entstehen geschlossene Figuren. Bei irrationalem Frequenzverhältnis füllen die Kurven langfristig die gesamte Fläche.

\subsection{Apparaturen und Sonstiges}

\subsubsection{Zeigermodell bei Schwingungen}
\label{sec:zeigermodell}
Das Zeigermodell ist eine anschauliche Methode, harmonische Schwingungen darzustellen.
Dabei wird die zeitliche Änderung einer Schwinggröße durch einen rotierenden Vektor (Zeiger) in der Ebene visualisiert.
\begin{figure}[H]
    \centering
    \includegraphics[width=0.9\linewidth]{figures/zeigerdiagramm.png}
    \caption{Schematische Darstellung des Zeigermodells.}
    \label{fig:zeigerdiagramm}
\end{figure}
Wie in \hyperref[fig:zeigerdiagramm]{Abbildung~\ref{fig:zeigerdiagramm}} dargestellt, entspricht die Länge des Zeigers (hier $\hat y$) der Amplitude $\hat s_0$.
Der Zeiger rotiert mit der Kreisfrequenz $\omega$ gegen den Uhrzeigersinn.
Die Projektion des Zeigers auf die y-Achse (hier $y(t)$) entspricht der momentanen Auslenkung $s(t)$ der Schwingung:
\begin{gather}
    s(t) = \hat s_0 \, \sin(\omega t + \varphi_0).
\end{gather}
Dieses Modell verdeutlicht anschaulich die Beziehung zwischen Kreisbewegung und harmonischer Schwingung.

\subsubsection{Federpendel}
\label{sec:federpendel}
Ein Federpendel ist ein mechanisches System, bei dem eine Masse an einer Feder hängt, auf einer Feder liegt oder zwischen zwei Federn eingespannt ist.
Unter idealisierten Bedingungen kann das System harmonisch schwingen.
Die Rückstellkraft der Feder und die Kreisfrequenz des Federpendels lassen sich schreiben als:
\begin{align}
    F_\mathrm{rück,\,fed} &= -D \, s, \\
    \omega_{0,\,\mathrm{fed}} &= \sqrt{\frac{D}{m}},
\end{align}
\begin{center}
    \begin{tabular}{rl}
        $D$ & Gesamthärte aller beteiligten Federn als Summe der Einzelhärten \\
        $s$ & Auslenkung aus der Ruhelage
    \end{tabular}
\end{center}
\begin{itemize}
    \item \emph{Horizontal:} Die Gewichtskraft wirkt senkrecht zur Schwingungsrichtung und wird durch die Normalkraft kompensiert, sodass die Schwingung ausschließlich durch die Feder bestimmt wird.
    \item \emph{Vertikal:} Die Gewichtskraft verschiebt die Ruhelage nach unten. Um die Schwingung korrekt zu beschreiben, wird diese neue Gleichgewichtslage als Nullpunkt der Auslenkung gewählt (vgl. Abschnitt~\ref{sec:harmonisierung_vertikales_federpendels}).
\end{itemize}

\medskip
\begin{tcolorbox}[colframe=purple!30!gray, colback=purple!10, title=Vertiefung: Herleitung der Kreisfrequenz beim Federpendel]
    Die Kreisfrequenz $\omega_{0,\,\mathrm{fedp}}$ bei Federpendeln berechnet sich nach dem zweiten Newton'schen Gesetz folgendermaßen:
    \begin{gather*}
        m \cdot \ddot{x} = F = - k \cdot x \\
        m \cdot \ddot{x} + k \cdot x = 0
    \end{gather*}
    Dies ist die Standardform der Differentialgleichung eines harmonischen Oszillators. Wir wählen als Lösungsansatz eine harmonische Schwingung:
    \begin{gather*}
        s(t) = \hat{s}_0 \cdot \sin(\omega t + \varphi_0) \\
        \ddot{s}(t) = - \hat{s}_0 \cdot \omega^2 \cdot \sin(\omega t + \varphi_0)
    \end{gather*}
    Einsetzen in die Standardform der Differenzialgleichung und Kürzen ergibt dann:
    \begin{gather*}
        - m \cdot \hat{s}_0 \cdot \omega^2 \cdot \sin(\omega t + \varphi_0) = - D \cdot \hat{s}_0 \cdot \sin(\omega t + \varphi_0) \\
        m \cdot \omega^2 = D
    \end{gather*}
    Für die Kreisfrequenz $\omega_{0,\,\mathrm{fedp}}$ bei Federpendeln ergibt sich:
    \begin{gather*}
        \omega_{0,\,\mathrm{fedp}} = \sqrt{\frac{D}{m}}
    \end{gather*}
\end{tcolorbox}

\subsubsection{Fadenpendel}
\label{sec:fadenpendel}
Ein Fadenpendel ist ein einfaches mechanisches System, bei dem eine Masse an einem ungefederten, leichten Faden hängt und unter der Wirkung der Schwerkraft um eine stabile Gleichgewichtslage schwingt. Bei sehr kleiner Auslenkung mit $\varphi \leq 10^\circ$ wird über die Kleinwinkelnäherung eine harmonische Schwingung angenommen.
\begin{gather}
    F_\mathrm{rück,\,fadp} \approx -\frac{m \cdot g \cdot s}{l} \\
    \omega_{0,\,\mathrm{fadp}} = \sqrt{\frac{g}{l}}
\end{gather}
\begin{center}
    \begin{tabular}{rl}
        $g$ & Erdbeschleunigung ($\approx 9{,}81\,\mathrm{m/s^2}$) \\
        $s$ & Auslenkung aus der Ruhelage (am Kreisabschnitt und horizontal) \\
        $l$ & Länge des Fadens
    \end{tabular}
\end{center}

\medskip
\begin{tcolorbox}[colframe=green!30!gray, colback=green!10, title=Vertiefung: Herleitung der Rückstellkraft beim Fadenpendel]
    Im Folgenden wird die Rückstellkraft $F_\mathrm{rück,\,fadp}$ unter Annahme der Kleinwinkelnäherung und mithilfe von \hyperref[fig:fadenpendel]{Abbildung~\ref{fig:fadenpendel}} hergeleitet. Die rücktreibende Kraft am Pendel ist derjenige Anteil der Gewichtskraft, der senkrecht zum Faden wirkt. Dabei ist $\varphi$ der Auslenkungswinkel:
    \begin{equation*}
        F_\mathrm{rück,\,fadp} = -m \cdot g \cdot \sin(\varphi)
    \end{equation*}
    Unter der Annahme, dass für kleine Winkel $x \approx s$ gilt, lässt sich über die trigonometrischen Beziehungen $\sin(\varphi)$ berechnen:
    \begin{equation*}
        \sin(\varphi) = \frac{x}{l} \approx \frac{s}{l}
    \end{equation*}
    Mit der Kleinwinkelnäherung folgt also für die Rückstellkraft:
    \begin{align*}
        F_\mathrm{rück,\,fadp}  &= -m \cdot g \cdot \sin(\varphi) \\
                               &\approx -m \cdot g \cdot \frac{s}{l}
    \end{align*}
    \begin{figure}[H]
        \centering
        \includegraphics[width=0.25\linewidth]{figures/fadenpendel.png}
        \caption{Schematische Darstellung eines Fadenpendels.}
        \label{fig:fadenpendel}
    \end{figure}
\end{tcolorbox}

\medskip
\begin{tcolorbox}[colframe=orange!30!gray, colback=orange!10, title=Vertiefung: Herleitung der Kreisfrequenz beim Fadenpendel]
    Die Herleitung der Kreisfrequenz $\omega_{0,\,\mathrm{fadp}}$ erfolgt über die bereits aus den Abschnitten~\ref{sec:rueckstellkraft} und~\ref{sec:federpendel} bekannten Gleichungen:
    \begin{equation*}
        D = \frac{|F_\mathrm{rück}|}{s} \qquad \omega_0 = \sqrt{\frac{D}{m}}
    \end{equation*}
    Versucht man, auch für das Fadenpendel eine Art Federhärte $D$ zu definieren, erhält man:
    \begin{equation*}
        D_\mathrm{fadp} = \frac{|F_\mathrm{rück,\,fadp}|}{s} = \frac{\frac{m \cdot g \cdot s}{l}}{s} = \frac{m \cdot g}{l}
    \end{equation*}
    Für die Kreisfrequenz ergibt sich:
    \begin{equation*}
        \omega_{0,\,\mathrm{fadp}} = \sqrt{\frac{D_\mathrm{fadp}}{m}} = \sqrt{\frac{\frac{m \cdot g}{l}}{m}} = \sqrt{\frac{g}{l}}
    \end{equation*}
\end{tcolorbox}

\subsection{Versuche}
\begin{itemize}
    \item Den Zusammenhang zwischen harmonischen mechanischen Schwingungen und linearer
    Rückstellkraft an Beispielen beschreiben
    \item Energieumwandlungen beim Federpendel erklären
\end{itemize}

\emph{Weitere Versuche hier einfügen.}


\newpage
\section{Wellentheorie}
Die nachfolgenden Grundlagen der Wellentheorie wurden in der zweiten Klausur der elften Klasse abgefragt.

\subsection{Relevante Größen und deren Zusammenhänge}

\subsubsection{Ausbreitungsgeschwindigkeit ($c$)}
\label{sec:ausbreitungsgeschwindigkeit}
Die Ausbreitungsgeschwindigkeit $c$ ist die Geschwindigkeit, mit der sich eine Welle oder ein Wellensignal durch ein Medium fortbewegt.
\begin{gather}
    c = \lambda f
\end{gather}
\begin{center}
    \begin{tabular}{rl}
        $\lambda$ & Wellenlänge \\
        $f$ & Frequenz
    \end{tabular}
\end{center}

\subsubsection{Wellenlänge ($\lambda$)}
\label{sec:wellenlaenge}
Die Wellenlänge $\lambda$ ist der räumliche Abstand zwischen zwei aufeinanderfolgenden Punkten gleicher Phase einer Welle, zum Beispiel zwischen zwei benachbarten Wellenbergen.
\begin{gather}
    \lambda=\frac{c}{f}
\end{gather}
\begin{center}
    \begin{tabular}{rl}
        $c$ & Ausbreitungsgeschwindigkeit der Welle \\
        $f$ & Frequenz
    \end{tabular}
\end{center}

\subsubsection{Phasenverschiebung ($\Delta \varphi$)}
\label{sec:phasenverschiebung}
Die Phasenverschiebung $\Delta \varphi$ ist der Unterschied im zeitlichen Verlauf zweier Schwingungen oder Wellen gleicher Frequenz, der angibt, um welchen Winkel oder Bruchteil einer Periode eine Welle gegenüber der anderen vor- oder nachläuft. Sie gibt die zeitliche Verschiebung zweier Wellen an.
\begin{gather}
    \Delta \varphi = \frac{2\pi}{\lambda} \cdot \delta
\end{gather}
\begin{center}
    \begin{tabular}{rl}
        $\lambda$ & Wellenlänge \\
        $\delta$ & Gangunterschied
    \end{tabular}
\end{center}
Die Einheit der Phasenverschiebung ist
\begin{center}
    $[\Delta \varphi] = \si{\radian}$
\end{center}

\subsubsection{Gangunterschied ($\delta$)}
\label{sec:gangunterschied}
Der Gangunterschied ist der Unterschied in den Weglängen, die zwei Wellen bis zu einem gemeinsamen Punkt zurückgelegt haben bzw. die räumliche Verschiebung der Wellen zueinander. Er ist entscheidend dafür, ob sich die kohärenten Wellen am Überlagerungspunkt konstruktiv (Verstärkung) oder destruktiv (Auslöschung) überlagern.
\begin{gather}
    \delta = \frac{\Delta \varphi}{2\pi} \cdot \lambda
\end{gather}
\begin{center}
    \begin{tabular}{rl}
        $\Delta \varphi$ & Phasenverschiebung \\
        $\lambda$ & Wellenlänge
    \end{tabular}
\end{center}
Damit entspricht ein Gangunterschied von $\lambda$ einer Phasenverschiebung von $2\pi$, während ein Gangunterschied von $\frac{\lambda}{2}$ einer Phasenverschiebung von $\pi$ entspricht.

\subsubsection{Schnelle ($v_s$)}
\label{sec:schnelle}
In der Wellenlehre bezeichnet die Schnelle $v_s$ die momentane Geschwindigkeit der Teilchen eines Mediums bei einer mechanischen Welle und gibt an, wie schnell sich die einzelnen Teilchen um ihre Gleichgewichtslage bewegen.
\begin{gather}
    v_s(t) = \dot{y}(t)
\end{gather}

\subsubsection{Wellengleichung mit Kreiswellenzahl ($y(x,\,t)$, $k$)}
\label{sec:wellengleichung_mit_kreiswellenzahl}
Die Wellengleichung $y(x,\,t)$ beschreibt die Abhängigkeit der Auslenkung einer Welle sowohl von der Zeit $t$ als auch vom Ort $x$. Sie entsteht, indem die zeitliche Schwingung $s(t)$ eines harmonischen Oszillators um den Ortsparameter $x$ erweitert wird. Dadurch wird nicht mehr nur die zeitliche Schwingung an einem festen Punkt, sondern die Ausbreitung dieser Schwingung im Raum beschrieben.
\begin{align}
    y(x,\,t) &= \hat y_0 \cdot \sin(\omega t - kx) \\
    &= \hat y_0 \cdot \sin\left(\frac{2\pi t}{T}  - \frac{2\pi x}{\lambda}\right) \\
    y(x,\,t) &= \hat y_0 \cdot \sin\left[2\pi \cdot \left(\frac{t}{T} - \frac{x}{\lambda}\right)\right]
\end{align}
\begin{center}
    \begin{tabular}{rl}
        $\hat y_0$ & Amplitude \\
        $\omega$ & Kreisfrequenz \\
        $k$ & Kreiswellenzahl (entspricht einer räumlichen Kreisfrequenz) \\
        $T$ & Periodendauer \\
        $\lambda$ & Wellenlänge
    \end{tabular}
\end{center}
Falls die Welle an der Stelle $x = 0$ oder am Zeitpunkt $t = 0$ nicht in der Ruhelage startet, muss anstelle von $\sin$ die Funktion $\cos$ verwendet oder eine entsprechende initiale Phasenverschiebung bzw. ein initialer Gangunterschied addiert werden; zudem ist das Vorzeichen je nach Ausbreitungsrichtung der Welle anzupassen. Im weiteren Verlauf des Dokuments wird die vereinfachte Form ohne eine Phase $\varphi'$ verwendet:
\begin{align}
    y(x,\,t) &= \hat y_0 \cdot \sin(\omega t + \Delta \varphi - kx + \delta) \\
    &= \hat y_0 \cdot \sin(kx - \omega t + \varphi')
\end{align}
\begin{center}
    \begin{tabular}{rl}
        $\varphi'$ & Effektive Phase aus $\Delta \varphi$ und $\delta$
    \end{tabular}
\end{center}
Die Einheit der Kreiswellenzahl ist
\begin{center}
    $[k] = \si{\per\meter} = \si{\radian\per\meter}$
\end{center}

\subsection{Wichtige Phänomene und Ergänzungen}

\subsubsection{Welle}
\label{sec:welle}
Eine Welle ist die räumliche und zeitliche Ausbreitung einer räumlichen und zeitlichen Schwingung bzw. Störung, bei der Energie, aber nicht dauerhaft Materie, von einem Ort zum anderen übertragen wird; sie erfolgt nach bestimmten periodischen Gesetzmäßigkeiten.

Eine Welle kann als Kette von gekoppelten Oszillatoren in einem Medium aufgefasst werden. Der erste Oszillator einer Kette wird in Schwingung versetzt. Jede Phase dieser erzwungenen Schwingung wird nach und nach von den anderen Körpern übernommen, als wären die Oszillatoren durch Federn verbunden. Durch diese Kopplung kann die Energie zwischen den Schwingkörpern weitergereicht werden.

Im Inneren von Flüssigkeiten und Gasen können im Gegensatz zu festen Körpern nur Längswellen entstehen. Die Oberfläche einer Flüssigkeit ist dagegen bestrebt, sich nach einer Störung wieder horizontal einzustellen. Diese Eigenschaft ermöglicht an einer Wasseroberfläche sogenannte \emph{Oberflächenwellen} mit einer Quer- als auch einer Längskomponente.

Die Ausbreitungsgeschwindigkeit $c$ hängt maßgeblich von der Masse und (Stärke der) Kopplung der Oszillatoren und somit dem Material/Medium zusammen. Ihre Energie ist als Spannerenergie $(W_\mathrm{span})$ und Bewegungsenergie $(W_\mathrm{kin})$ auf die einzelnen Oszillatoren verteilt.

\subsubsection{Querwelle bzw. Transversalwelle}
\label{sec:querwelle_bzw_transversalwelle}
Querwellen (Transversalwellen) sind im Unterschied zu \hyperref[sec:laengswelle_bzw_longitudinalwelle]{Längswellen} Wellen, bei denen die Schwingungsrichtung der Teilchen senkrecht zur Ausbreitungsrichtung der Welle steht.

\subsubsection{Längswelle bzw. Longitudinalwelle}
\label{sec:laengswelle_bzw_longitudinalwelle}
Längswellen (Longitudinalwellen) sind Wellen, bei denen die Schwingungsrichtung der Teilchen parallel zur Ausbreitungsrichtung der Welle verläuft. Dazu ist die Kompression und Expansion des Mediums erforderlich, was nur in festen Körpern möglich ist. Längswellen sind dabei meist schneller als Querwellen.

\subsubsection{Wasserwelle}
\label{sec:wasserwelle}
Wasserwellen stellen einen Spezialfall dar, da sie keine reinen \hyperref[sec:querwelle_bzw_transversalwelle]{Quer-} oder \hyperref[sec:laengswelle_bzw_longitudinalwelle]{Längswellen} sind. Vielmehr handelt es sich um Überlagerungen beider Wellenarten: Die Wasserteilchen führen näherungsweise Kreisbahnen aus, die sowohl eine senkrechte als auch eine parallele Komponente zur Ausbreitungsrichtung besitzen. In tieferen Schichten des Wassers nimmt der Radius dieser Kreisbahnen ab, bis die Bewegung in großer Tiefe praktisch vernachlässigbar ist. Deshalb zählen Wasserwellen zu den sogenannten Oberflächenwellen. Sie treten an der Grenzfläche zwischen Wasser und Luft auf und spielen insbesondere in der Ozeanographie und Strömungslehre eine bedeutende Rolle.

\subsubsection{Stehende Welle}
\label{sec:stehende_welle}
Eine stehende Welle entsteht, wenn sich zwei Wellen mit gleicher Frequenz, gleicher Amplitude und gleicher Geschwindigkeit überlagern, die in entgegengesetzten Richtungen laufen. Diese Überlagerung führt zu einem charakteristischen Muster, bei dem sich Bereiche ohne Schwingung, sogenannte \emph{Knoten}, mit Bereichen maximaler Auslenkung, den \emph{Bäuchen}, abwechseln. Die Welle scheint dabei stillzustehen, obwohl sie aus der kontinuierlichen Überlagerung der beiden sich bewegenden Wellen resultiert.

Die Wellenlänge einer stehenden Welle kann durch Messung des Abstands benachbarter Knoten bestimmt werden: $\lambda = 2 \cdot d_{knot}$

Ein klassisches Beispiel für stehende Wellen ist die schwingende Saite eines Musikinstruments. Die an beiden Enden fixierte Saite reflektiert die Wellen, wodurch stehende Wellen mit Knoten an den Enden entstehen. Je nach Frequenz bilden sich verschiedene Muster, von der Grundschwingung mit einem Bauch bis zu höheren Moden mit mehreren Knoten und Bäuchen.

Stehende Wellen entstehen auch in anderen Medien wie Luftsäulen oder Wasseroberflächen durch Reflexion oder Interferenz in begrenzten Systemen. Entscheidend sind dabei die Eigenfrequenzen des Systems, bei denen die Wellenlänge genau passt und Resonanz auftritt.

 Siehe auch: \url{https://www.leifiphysik.de/mechanik/mechanische-wellen/versuche/stehende-welle-simulation}

\subsubsection{Huygens'sches Prinzip mit Wellenfront und Wellennormale}
\label{sec:huygenssches_prinzip_mit_wellenfront_und_wellennormale}
Das Huygens'sche Prinzip besagt, dass jeder Punkt einer Wellenfront als Ausgangspunkt für neue Elementarwellen betrachtet werden kann. Diese Elementarwellen breiten sich in alle Richtungen mit der gleichen Geschwindigkeit aus wie die ursprüngliche Welle. Die Überlagerung dieser Wellen beschreibt die Fortpflanzung der Gesamten Wellenfront.

Dieses Prinzip erklärt zentrale Welleneffekte wie Beugung, Brechung oder Reflexion.

Bei ebenen Wellen wird die Wellenfront durch parallele, gleichphasige Elementarwellen erzeugt. Bei Kreiswellen sendet ein Punkt als Quelle neue Wellenfronten aus. Alle Punkte, die gleich weit vom Erregerzentrum entfernt sind, schwingen in Phase.

Die Wellennormale ist in der Physik eine Linie, die senkrecht zur Wellenfront verläuft und die Richtung der Wellenausbreitung angibt.

\subsubsection{Beugung}
\label{sec:beugung}
Die Beugung beschreibt das Phänomen, bei dem Wellen um Hindernisse oder durch enge Öffnungen abgelenkt werden und sich hinter dem Hindernis in neue Richtungen ausbreiten. Beugung ist besonders ausgeprägt, wenn die Wellenlänge der Welle vergleichbar mit der Größe des Hindernisses oder der Öffnung ist.
\begin{figure}[H]
    \centering
    \includegraphics[width=0.35\linewidth]{figures/beugung_schlitz.png}
    \caption{Schematische Darstellung der Beugung an einem Schlitz.}
    \label{fig:beugung_schlitz}
\end{figure}

\subsubsection{Brechung}
\label{sec:brechung}
Bei der Brechung tritt eine Welle von einem Medium in ein anderes mit unterschiedlicher Ausbreitungsgeschwindigkeit ein. Nach dem Huygens'schen Prinzip entstehen an jedem Punkt der Wellenfront neue Elementarwellen, die sich mit der Geschwindigkeit des jeweiligen Mediums ausbreiten.
Der Winkel zwischen dem einfallenden Strahl und der Lotrechten an der Grenzfläche wird als Einfallswinkel $\alpha$ bezeichnet, der Winkel des gebrochenen Strahls als Brechungswinkel $\beta$. Diese Größen stehen im folgenden Zusammenhang, dem Brechungsgesetz:
\begin{gather}
    \frac{\sin \alpha}{\sin \beta} = \frac{c_1}{c_2}
\end{gather}
\begin{center}
    \begin{tabular}{rl}
        $\alpha$ & Einfallswinkel (zwischen einfallender Welle und Lot) \\
        $\beta$ & Brechungswinkel (zwischen gebrochener Welle und Lot) \\
        $c_1, c_2$ & Ausbreitungsgeschwindigkeiten in den beiden Medien
    \end{tabular}
\end{center}
\begin{figure}[H]
    \centering
    \includegraphics[width=0.43\linewidth]{figures/brechung.png}
    \caption{Schematische Darstellung der Brechung am Übergang zwischen zwei Medien.}
    \label{fig:brechung}
\end{figure}

\subsubsection{Reflexion am festen und losen Ende}
\label{sec:reflexion_am_festen_und_losen_ende}
Die Reflexion einer Welle beschreibt das Zurückwerfen einer Welle an einem Hindernis. Die Art der Reflexion hängt davon ab, ob das reflektierende Ende fest oder los ist.
\begin{itemize}
    \item \emph{Festes Ende:} Die Welle wird mit Phasenumkehr reflektiert, d.\,h., die Auslenkung wird um $\pi$ verschoben. Die reflektierte Welle schwingt in entgegengesetzter Richtung zur ursprünglichen Auslenkung.
    \item \emph{Loses Ende:} Die Welle wird ohne Phasenverschiebung reflektiert. Die Auslenkung der reflektierten Welle behält die gleiche Richtung wie die ursprüngliche Welle.
\end{itemize}
Siehe auch: \url{https://www.leifiphysik.de/mechanik/mechanische-wellen/grundwissen/reflexion}

Der Verlauf der reflektierten Welle lässt sich mithilfe des Huygens'schen Prinzips herleiten: Jeder Punkt der einfallenden Wellenfront wirkt als Quelle neuer Elementarwellen, die sich in alle Richtungen ausbreiten.
\begin{figure}[H]
    \centering
    \includegraphics[width=1\linewidth]{figures/reflexion_huygens.png}
    \vspace{-0.8cm} % Ausgleich für dicken Rand am Bild
    \caption{Schematische Darstellung der Reflexion nach Huygens.}
    \label{fig:reflexion_huygens}
\end{figure}
Die Richtung der reflektierten Welle wird durch den Einfallswinkel $\alpha$ zum Lot bestimmt. Dabei gilt das Reflexionsgesetz:
\begin{gather}
    \alpha_\mathrm{einfall} = \alpha_\mathrm{reflektiert}
\end{gather}

\subsubsection{Interferenz}
\label{sec:interferenz}
Interferenz ist das Phänomen, das auftritt, wenn zwei oder mehr Wellen im gleichen Raum aufeinandertreffen und sich überlagern. Diese Überlagerung kann konstruktiv oder destruktiv sein, je nach \hyperref[sec:phasenverschiebung]{Phasenverschiebung} der Wellen. Die Phasenverschiebung wiederum lässt sich häufig durch den sogenannten \hyperref[sec:gangunterschied]{Gangunterschied} erklären, also den Unterschied in den zurückgelegten Weglängen der Wellen. Auch bei der Überlagerung von Wellen kann wie bei der Überlagerung von Schwingungen Schwebung auftreten, was gut bei Schallwellen wahrnehmbar ist.

Bei konstruktiver Interferenz verstärken sich die Wellen, wenn ihre Auslenkungen in die gleiche Richtung zeigen, was zu einer größeren Amplitude führt. Bei destruktiver Interferenz schwächen sich die Wellen ab, wenn ihre Auslenkungen in entgegengesetzte Richtungen zeigen und sich gegenseitig auslöschen oder vermindern.

Die resultierende Elongation an jedem Punkt ist das Ergebnis der Vektoraddition aller Einzelelongationen.

\subsubsection{Kohärenz}
\label{sec:kohaerenz}
Kohärenz beschreibt die Eigenschaft von Wellen, in einem festen phasenmäßigen Zusammenhang zueinander zu stehen. Zwei Wellen sind kohärent, wenn ihr Phasenunterschied über die Zeit konstant bleibt. Man unterscheidet dabei zwischen \emph{zeitlicher Kohärenz}, die eine feste Phasenbeziehung an einem Ort über die Zeit beschreibt, und \emph{räumlicher Kohärenz}, bei der an verschiedenen Orten einer Wellenfront ein fester Phasenbezug vorliegt. Kohärenz ist die Voraussetzung dafür, dass sich Wellen durch Überlagerung (Interferenz) verstärken oder auslöschen können.

Ein Beispiel für hohe Kohärenz ist das Licht eines Lasers: Die ausgesendeten Wellen haben nahezu die gleiche Frequenz und eine feste Phasenbeziehung. Gewöhnliche Lichtquellen wie Glühlampen sind hingegen inkohärent, da sie viele verschiedene Frequenzen abstrahlen und die Phasenbeziehungen der einzelnen Wellen zufällig variieren. Das Laserlicht zeichnet sich zusätzlich durch eine hohe Bündelung (geringe Divergenz) und Monochromasie (Einfarbigkeit) aus, die im Zusammenhang mit der Kohärenz eine gezielte technische Nutzung ermöglichen.

\subsubsection{Eigenschwingungen mit festen und losen Enden}
\label{sec:eigenschwingungen_mit_festen_und_losen_Enden}
Eigenschwingungen entstehen, wenn ein System so schwingt, dass an seinen Randbedingungen (fest oder frei) immer ein Knoten (feste Stelle, keine Auslenkung) oder ein Bauch (freie Stelle, maximale Auslenkung) passt. Da somit nur bestimmte Wellenlängen $\lambda$ erlaubt sind, ergeben sich nur bestimmte Frequenzen $f$, die als Eigenfrequenzen bezeichnet werden.

Die niedrigste Frequenz wird als Grundfrequenz $f_1$ und die höheren als Oberfrequenzen oder Harmonische $f_k$ bezeichnet.

Bei \emph{zwei festen oder zwei freien Enden}, wie bei einer Gitarrensaite, die an beiden Enden eingespannt ist, oder einer Luftsäule mit zwei offenen Enden, enthält die Länge $l$ immer ein ganzzahliges Vielfaches von $\frac{\lambda}{2}$:
\begin{gather}
    l = k \cdot \frac{\lambda}{2}
\end{gather}
Mit der Gleichung $c = f \lambda$ folgt:
\begin{gather}
    f_k = \frac{k c}{2 l} = k \cdot f_1
\end{gather}
\begin{center}
    \begin{tabular}{rl}
        $k$ & Ordnung der Harmonischen (ganzzahlig) \\
        $c$ & Ausbreitungsgeschwindigkeit \\
        $l$ & Länge des schwingenden Systems
    \end{tabular}
\end{center}
Das heißt: Die Eigenfrequenzen sind ganzzahlige Vielfache der Grundfrequenz und bilden die Harmonischen.

Bei \emph{einem festen und einem freien Ende}, also zum Beispiel eine Saite, die an einem Ende eingespannt, am anderen aber frei schwingbar ist, oder eine Orgelpfeife mit einem geschlossenen und einem offenen Ende, passt in die Länge $l$ immer ein ungeradzahliges Vielfaches von $\frac{\lambda}{4}$
\begin{gather}
    l = (2k - 1) \cdot \frac{\lambda}{4}
\end{gather}
Daraus folgt für $f_k$:
\begin{gather}
    f_k = \frac{(2k-1) \cdot c}{4l}
\end{gather}
Auch hier bezeichnet man die Grundfrequenz $f_1$ als erste harmonische Schwingung und die höheren Frequenzen als ungerade Harmonische.

\subsubsection{Polarisation, Polfilter und Analysator}
\label{sec:polarisation_polfilter_und_analysator}
Die Polarisation einer Welle beschreibt die Richtung, in der die Schwingung der Welle erfolgt. Bei elektromagnetischen Wellen, wie Licht, bezieht sich dies auf die Richtung des elektrischen Feldvektors. Schwingt dieser Vektor schnell und ungeordnet, spricht man von unpolarisiertem Licht.

Ein Polarisationsfilter (Polfilter) kann verwendet werden, um aus unpolarisiertem Licht polarisiertes Licht zu erzeugen. Der Filter lässt nur Licht mit einer bestimmten Schwingungsrichtung passieren. Polfilter bestehen typischerweise aus parallelen Molekülketten, die den Teil der Welle absorbieren, der in dieselbe Richtung wie die Moleküle schwingt, während der senkrecht dazu schwingende Teil durchgelassen wird.

Ein Analysator ist ein weiterer Polfilter, der nach einem ersten Polfilter angeordnet wird, um die Polarisation des durchgelassenen Lichts zu überprüfen oder zu messen. Durch Drehung des Analysators kann man feststellen, ob das Licht linear polarisiert ist und in welcher Richtung seine Schwingung verläuft. Man unterscheidet verschiedene Arten der Polarisation:
\begin{itemize}
    \item \emph{Lineare Polarisation:} Der elektrische Feldvektor schwingt in einer festen Richtung.
    \item \emph{Kreis- oder elliptische Polarisation:} Der elektrische Feldvektor rotiert während der Ausbreitung, sodass die Spitze des Vektors eine Kreis- oder Ellipsenbahn beschreibt.
\end{itemize}
Dadurch lassen sich Polarisationszustände von Licht genau analysieren und gezielt einsetzen.

Polarisationsfilter werden in der Fotografie verwendet, um Reflexionen zu reduzieren und die Sättigung von Farben zu erhöhen. Sie sind besonders nützlich bei der Aufnahme von Landschaften, da sie störende Reflexionen von Wasseroberflächen, Glas oder anderen reflektierenden Oberflächen eliminieren können. Dies führt zu klareren und kontrastreicheren Bildern.

\subsection{Apparaturen und Sonstiges}

\subsubsection{Zeigermodell bei Wellen}
\label{sec:zeigermodell_bei_wellen}
Das Zeigermodell bei Wellen stellt eine harmonische Welle durch gegen den Uhrzeigersinn rotierende Pfeile (Zeiger) dar, die sich an jedem Ort mit derselben Frequenz drehen. Die absolute Länge der Zeiger entspricht der Amplitude der Welle. Betrachtet man die Projektion der Zeiger auf die y-Achse, so ergibt sich daraus die momentane Auslenkung an jedem Ort. In einer Momentaufnahme sieht man, dass die Zeiger an verschiedenen Orten um unterschiedliche Winkel gedreht sind -- diese Unterschiede entsprechen den Phasenverschiebungen entlang der Ausbreitungsrichtung der Welle. Auf diese Weise macht das Modell anschaulich, wie sich eine Welle aus vielen einzelnen Schwingungen zusammensetzt, die alle dieselbe Frequenz haben, aber ortsabhängig mit verschiedener Phase verlaufen.
\begin{figure}[H]
    \centering
    \includegraphics[width=0.8\linewidth]{figures/zeigermodell_bei_wellen.png}
    \caption{Darstellung des Zeigermodells bei Wellen.}
    \label{fig:zeigermodell_bei_wellen}
\end{figure}

\subsubsection{Darstellung von Wellen zu bestimmten Zeitpunkten}
\label{sec:darstellung_von_wellen_zu_bestimmten_zeitpunkten}
Fixiert man die Zeit $t$ und betrachtet die Welle in Abhängigkeit vom Ort $x$ in einem $y$-$x$-Diagramm, so erhält man ein räumliches Bild der Welle. Es beschreibt, wie die Welle zu einem bestimmten Zeitpunkt an verschiedenen Orten ausgelenkt ist. Eine harmonische Welle kann allgemein durch folgende Gleichung beschrieben werden:
\begin{gather*}
    y(x,\,t) = \hat y_0 \cdot \sin(\omega t - kx)
\end{gather*}
Für einen festen Zeitpunkt $t = t_0$ reduziert sich die Gleichung auf Folgendes:
\begin{align*}
    y(x,\,t) &= \hat y_0 \cdot \sin(\omega t_0 - kx) \\
             &= \hat y_0 \cdot \sin(-kx + \varphi'(t_0))
\end{align*}
Dabei ist $\varphi'(t_0) = \omega t_0$ eine effektive Phase, die die Zeitabhängigkeit enthält. Außerdem muss beachtet werden, dass sich die Welle bis zum Zeitpunkt $t_0$ nur bis zur Position $x(t_0) = c \cdot t_0$ ausgebreitet hat. Das entsprechende Diagramm darf daher nur bis zu diesem Ort $x(t_0)$ gezeichnet werden. Es ist sinnvoll, die Welle von dieser Stelle aus rückwärts zu zeichnen, um die Phasenverschiebung korrekt darzustellen. Beginnt die Welle beispielsweise mit einer Auslenkung aus der Ruhelage nach oben, so muss sie auch im Diagramm von rechts nach links mit einer Auslenkung nach oben beginnen. Der Grund ist, dass beim weiteren Fortschreiten der Welle die Teilchen links der Wellenfront zuerst nach oben ausgelenkt werden.
\begin{figure}[H]
    \centering
    \includegraphics[width=0.9\linewidth]{figures/welle_zu_fester_zeit.png}
    \caption{Darstellung einer Welle zum Zeitpunkt $t_0$.}
    \label{fig:welle_zu_fester_zeit}
\end{figure}

\subsubsection{Darstellung von Wellen an bestimmten Orten}
\label{sec:darstellung_von_wellen_an_bestimmten_orten}
Fixiert man den Ort $x$ und betrachtet die Welle in Abhängigkeit von der Zeit $t$, so erhält man ein zeitliches Bild der Welle als $y$-$t$-Diagramm. Es beschreibt quasi wie bei einer harmonischen Schwingung, wie sich die Auslenkung an diesem festen Punkt $x_0$ im Laufe der Zeit verändert. Eine harmonische Welle lässt sich allgemein durch folgende Gleichung beschreiben:
\begin{gather*}
    y(x,\,t) = \hat y_0 \cdot \sin(\omega t - kx)
\end{gather*}
Für einen festen Ort $x = x_0$ ergibt sich Folgendes:
\begin{align*}
    y(x_0, t) &= \hat{y}_0 \cdot \sin\!\left(\omega t - kx_0\right) \\
              &= \hat{y}_0 \cdot \sin\!\left(\omega t + \varphi'(x_0)\right)
\end{align*}
Dabei ist $\varphi'(x_0) = -kx_0$ eine effektive Phase, die die Ortsabhängigkeit enthält. Hierbei bietet es sich an, die welle von $t_0 = \tfrac{x_0}{c}$, also dem Zeitpunkt, an dem die Welle den Ort $x_0$ erreicht, aus nach rechts zu zeichnen. Beginnt die Welle an diesem Ort zum Beispiel mit einer Auslenkung aus der Ruhelage nach unten, so muss sie auch im Diagramm Aus der Ruhelage nach unten starten. Der Grund ist, dass die Teilchen an diesem Ort $x_0$ zuerst nach unten ausgelenkt werden, wenn sie bei $t_0$ von der die Welle erreicht werden.
\begin{figure}[H]
    \centering
    \includegraphics[width=0.9\linewidth]{figures/welle_an_festem_ort.png}
    \caption{Darstellung einer Welle bei $x_0$.}
    \label{fig:welle_an_festem_ort}
\end{figure}

\subsubsection{Noise-Cancelling durch Gegenschall}
\label{sec:noise-cancelling_durch_gegenschall}
Beim Noise-Cancelling wird das Prinzip der destruktiven Interferenz genutzt: Ein störendes Schallsignal (z.\,B. Umgebungslärm) wird durch ein Mikrofon aufgenommen und elektronisch verarbeitet. Anschließend erzeugt das System eine zweite Schallwelle, die exakt dieselbe Frequenz und Amplitude, jedoch eine Phasenverschiebung von $180^\circ$ besitzt. Treffen beide Wellen aufeinander, heben sie sich gegenseitig weitgehend auf, sodass das resultierende Schallsignal stark abgeschwächt wird. Dieser Effekt ist besonders bei tiefen, gleichmäßigen Frequenzen (z.\,B. Motorengeräuschen) wirksam. Bei unregelmäßigen, hochfrequenten Geräuschen ist die Unterdrückung technisch schwieriger, da diese sich nicht so einfach exakt gegenphasig erzeugen lassen.

\subsubsection{Tsunami}
\label{sec:tsunami}
Ein Tsunami entsteht durch plötzliche, starke Verlagerungen großer Wassermassen, die meist durch ein Erdbeben unter dem Meeresboden verursacht werden. Dabei verschieben sich Gesteinsschichten entlang einer tektonischen Platte, wodurch der Meeresboden entweder angehoben oder abgesenkt wird. Diese Verlagerung überträgt sich auf die darüber liegende Wassersäule, wodurch eine Welle erzeugt wird. Andere Ursachen können auch Vulkanausbrüche, Erdrutsche oder Meteoriteneinschläge ins Wasser sein.

Die Wellen eines Tsunamis breiten sich kreisförmig von ihrem Entstehungsort aus und erreichen Geschwindigkeiten von bis zu \SI{800}{\kilo\meter\per\hour}, abhängig von der Tiefe des Wassers. In tiefem Wasser sind die Wellen kaum zu bemerken, da ihre Amplitude gering ist und sich die Energie auf eine große Fläche verteilt. Wenn der Tsunami flacheres Küstenwasser erreicht, wird die Geschwindigkeit aufgrund der geringeren Wassertiefe reduziert. Gleichzeitig steigt die Höhe der Wellen (Amplitude), weil die Wellenlänge abnimmt und die gesamte Energie der Welle auf eine kleinere Fläche konzentriert wird. Dieser Prozess nennt sich Wellenaufsteilung und ist über die folgenden Formeln nachvollziehbar:
\begin{gather}
    c = \lambda \cdot f = \sqrt{g \cdot h} \\
    \lambda \propto \sqrt{h}
\end{gather}
\begin{center}
    \begin{tabular}{rl}
        $\lambda$ & Wellenlänge \\
        $f$ & Frequenz \\
        $g$ & Erdbeschleunigung ($\approx 9{,}81\,\mathrm{m/s^2}$) \\
        $h$ & Wassertiefe \\
        $c$ & Ausbreitungsgeschwindigkeit
    \end{tabular}
\end{center}
Schließlich erreicht der Tsunami die Küste und kann dort mit großer Zerstörungskraft auftreten, da riesige Wassermassen auf Land treffen.

\subsubsection{Erdbeben und Seismograf}
\label{sec:erdbeben_und_seismograf}
Ein Erdbeben entsteht durch plötzliche Verschiebungen von Gesteinsmassen im Erdinneren, meist entlang von Verwerfungen. Dabei werden mechanische Wellen freigesetzt, die sich kugelförmig vom Herd ausbreiten.
\begin{itemize}
    \item \emph{Primärwellen (P-Wellen):} Längswellen, die sich am schnellsten ausbreiten und zuerst an Messstationen ankommen.
    \item \emph{Sekundärwellen (S-Wellen):} Querwellen, die langsamer als P-Wellen sind und danach registriert werden.
    \item \emph{Oberflächenwellen:} Breiten sich nur entlang der Erdoberfläche aus und treffen zuletzt ein, verursachen oft die stärksten Zerstörungen.
\end{itemize}
Die Laufzeitdifferenz von P- und S-Wellen ermöglicht die Abschätzung der Entfernung zum Erdbebenherd.

Ein Seismograf kann die Erschütterungen des Bodens aufzeichnen und besteht zum Beispiel aus einem schweren Pendel oder Topfmagneten, der wegen seiner Trägheit gegenüber dem bewegten Untergrund nahezu unbewegt bleibt. Relativbewegungen zwischen Untergrund und Masse werden durch einen Schreibstift oder Sensor registriert und liefern ein Seismogramm der Bodenbewegung.

\subsection{Versuche}
\begin{itemize}
    \item Erklären, dass ein Beobachter, der sich relativ zu einem Wellensender bewegt, eine andere
    Frequenz beziehungsweise Wellenlänge wahrnimmt als die von der Quelle erzeugte
    (Doppler-Effekt, Rotverschiebung und Blauverschiebung)
    \item Eindimensionale stehende Transversalwellen beschreiben und als Interferenzphänomen
    erklären (Bäuche, Knoten, Eigenfrequenzen, Stellen konstruktiver beziehungsweise destruktiver
    Interferenz, Reflexion an festen beziehungsweise losen Enden, Wellenlängenbestimmung mittels
    Knotenabstand)
    \item Mithilfe des Gangunterschieds die Überlagerung zweidimensionaler kohärenter Wellen
    beschreiben
\end{itemize}

\emph{Weitere Versuche hier einfügen.}


\newpage
\section{Elektrizitätslehre}
Die nachfolgenden Grundlagen der Elektrizitätslehre wurden in der dritten Klausur der elften Klasse abgefragt.

\subsection{Relevante Größen und deren Zusammenhänge}

\subsubsection{Elektrische Ladung ($Q$, $q$)}
\label{sec:elektrische_ladung}
Die elektrische Ladung $Q$ oder $q$ ist eine fundamentale physikalische Größe, die angibt, wie stark ein Teilchen an elektromagnetischen Wechselwirkungen beteiligt ist. $Q$ bezeichnet dabei eine makroskopische Gesamtladung, $q$ die Ladung einzelner Teilchen. Sie ist die Ursache aller elektrischen Erscheinungen.

Die kleinste bekannte Ladungseinheit ist die Elementarladung $e$. Elektronen tragen die Ladung $-e$, Protonen die Ladung $+e$. Negativ geladene Körper besitzen einen Elektronenüberschuss, positiv geladene einen Elektronenmangel. Gleichnamige Ladungen stoßen sich ab, ungleichnamige ziehen sich an.
\begin{gather}
    e \approx 1{,}602 \times 10^{-19}\,\si{\coulomb}
\end{gather}
Ein grundlegendes Prinzip ist die Ladungserhaltung: Ladung kann nicht erzeugt oder vernichtet werden. Bei der Neutralisation bleiben die einzelnen Ladungen bestehen, ihre Wirkung hebt sich jedoch nach außen auf.
Jede Ladung ist Quelle eines elektrischen Feldes, das die Kräfte zwischen den Ladungen vermittelt (vgl. Abschnitt~\ref{sec:elektrisches_feld}).

Die Einheit der elektrischen Ladung ist
\begin{center}
    $[Q] = \si{\coulomb} = \mathrm{Coulomb} = \si{\ampere\second}$
\end{center}

\subsubsection{Flächenladungsdichte ($\sigma_{\mathrm{allg}}$, $\sigma_{\mathrm{hom}}$)}
\label{sec:flaechenladungsdichte}
Die Flächenladungsdichte $\sigma$ beschreibt die auf einer Fläche $A$ gespeicherte elektrische Ladung $Q$ pro Flächeneinheit. Sie charakterisiert somit die Verteilung von Ladung auf Oberflächen.

Allgemein gilt:
\begin{gather}
    \sigma_{\mathrm{allg}} = \frac{Q}{A}
\end{gather}
\begin{center}
    \begin{tabular}{rl}
        $Q$ & Elektrische Ladung \\
        $A$ & Fläche
    \end{tabular}
\end{center}
Für homogene Felder, wie sie beispielsweise im Plattenkondensator auftreten, ergibt sich:
\begin{gather}
    \sigma_{\mathrm{hom}} = \varepsilon_0 \cdot E_{\mathrm{hom}}
\end{gather}
\begin{center}
    \begin{tabular}{rl}
        $\varepsilon_0$ & Elektrische Feldkonstante \\
        $E_{\mathrm{hom}}$ & Elektrische Feldstärke
    \end{tabular}
\end{center}
Die Einheit der Flächenladungsdichte ist
\begin{center}
    $[\sigma] = \si{\coulomb\per\meter\squared}$
\end{center}

\subsubsection{Elektrische Stromstärke ($I$)}
\label{sec:stromstaerke}
Die elektrische Stromstärke $I$ beschreibt die Menge an elektrischer Ladung $\Delta Q$, die pro Zeitintervall $\Delta t$ durch einen Leiterquerschnitt transportiert wird. Sie ist damit ein Maß für die Bewegung von Ladungsträgern, die entweder in Metallen als Elektronen oder in Flüssigkeiten und Gasen als Ionen auftreten können.
\begin{gather}
    I = \frac{\Delta Q}{\Delta t}
\end{gather}
\begin{center}
    \begin{tabular}{rl}
        $\Delta Q$ & Transportierte Ladungsmenge \\
        $\Delta t$ & Zeitintervall
    \end{tabular}
\end{center}
Elektrischer Strom ist also der gerichtete Transport von Ladungsträgern durch einen Stoff oder im Vakuum. Die Leitfähigkeit eines Materials hängt davon ab, wie fest die Elektronen an die Atome gebunden sind. Metalle besitzen viele frei bewegliche Elektronen und sind daher gute Leiter, während Isolatoren wie Glas oder Kunststoff kaum freie Ladungsträger enthalten und elektrischen Strom nahezu nicht leiten.

Die Einheit der elektrischen Stromstärke ist
\begin{center}
    $[I] = \si{\ampere} = \mathrm{Ampere} = \si{\coulomb\per\second}$
\end{center}

\subsubsection{Elektrisches Potenzial ($\varphi$)}
\label{sec:elektrisches_potenzial}
Das elektrische Potenzial $\varphi$ beschreibt die potenzielle Energie $W$ pro Ladungseinheit $q$ an einem bestimmten Punkt im Raum. Es gibt an, welche Arbeit notwendig ist, um eine Probeladung von einem Bezugsniveau an diesen Punkt zu bringen.
\begin{gather}
    \varphi = \frac{W}{q}
\end{gather}
\begin{center}
    \begin{tabular}{rl}
        $W$ & Potenzielle Energie \\
        $q$ & Probeladung
    \end{tabular}
\end{center}
Das Potenzial eines Punktes $P$ entspricht der Spannung zwischen $P$ und einem gewählten Bezugspunkt. Die Spannung zwischen zwei Punkten $P_1$ und $P_2$ ist die Potenzialdifferenz $\varphi_{P_1} - \varphi_{P_2}$. Spannung ist somit der Unterschied der Potenziale.

Beim Transport einer positiven Ladung von der positiven zur negativen Platte eines Kondensators verringert sich ihre potenzielle Energie entsprechend der durchlaufenen Potenzialdifferenz um $\Delta W_{\mathrm{pot}} = q \cdot (\varphi_{P_1} - \varphi_{P_2}) = q \cdot U$. Dabei ist der Weg, den die Ladung zurücklegt, unerheblich. Auf halber Strecke besitzt die Ladung die Hälfte der Energie und das halbe Potenzial.

Das höhere (positive) Potenzial befindet sich am Pluspol, das niedrigere (negative) am Minuspol. Äquipotenzialflächen verbinden Punkte gleichen Potenzials und verlaufen stets senkrecht zu den Feldlinien (vgl. Abschnitt~\ref{sec:elektrisches_feld}).

Die Einheit des elektrischen Potenzials ist
\begin{center}
    $[\varphi] = \si{\volt} = \mathrm{Volt} = \si{\joule\per\coulomb}$
\end{center}

\medskip
\begin{tcolorbox}[colframe=blue!30!gray, colback=blue!10, title=Vertiefung: Vergleich des elektrisches Potenzials und der Spannung mit Gravitationsfeldern]
    Das elektrische Potenzial $\varphi$ an einem Punkt im Raum ist vergleichbar mit der Höhe im Gravitationsfeld:
    \begin{itemize}
        \item Wie ein Körper im Gravitationsfeld von höherem zu niedrigerem Potenzial (von oben nach unten) fällt, bewegt sich eine positive Ladung im elektrischen Feld von höherem zu niedrigerem elektrischen Potenzial.
        \item Die Spannung $U$ zwischen zwei Punkten entspricht dem Höhenunterschied im Gravitationsfeld. Sie gibt an, wie viel Energie pro Ladungseinheit bei der Bewegung zwischen diesen Punkten umgesetzt wird.
        \item Äquipotenzialflächen sind vergleichbar mit Höhenlinien auf einer Landkarte.
    \end{itemize}
\end{tcolorbox}

\subsubsection{Coulomb-Potenzial ($\varphi_{\mathrm{coul}}$)}
\label{sec:coulomb_potenzial}
Das Coulomb-Potenzial $\varphi_{\mathrm{coul}}$ ist ein Spezialfall des \hyperref[sec:elektrisches_potenzial]{elektrischen Potenzials} und beschreibt das von einer einzelnen Punktladung $Q$ erzeugte elektrische Potenzial in einem Abstand $r$ von der Ladung.

Für eine Punktladung gilt:
\begin{gather}
    \varphi_{\mathrm{coul}} = \frac{1}{4\pi\varepsilon_0} \cdot \frac{Q}{r}.
\end{gather}
\begin{center}
    \begin{tabular}{rl}
        $Q$ & Punktladung \\
        $r$ & Abstand zur Ladung \\
        $\varepsilon_0$ & Elektrische Feldkonstante
    \end{tabular}
\end{center}
Bei mehreren Punktladungen addieren sich die Potenziale vektorfrei nach dem Superpositionsprinzip. Das Innere einer geladenen Hohlkugel ist feldfrei und besitzt überall das gleiche Potenzial wie an der Oberfläche.
Das Potenzial nimmt mit dem Abstand $r$ wie $\frac{1}{r}$ ab, während die elektrische Feldstärke $\vec{E}$ mit $\frac{1}{r^2}$ abnimmt.

\subsubsection{Elektrische Spannung ($U$)}
\label{sec:elektrische_spannung}
Die elektrische Spannung beschreibt die Arbeit pro Ladungseinheit, die beim Transport einer Ladung zwischen zwei Punkten verrichtet wird und entsteht, wenn entgegengesetzte Ladungen unter Energiezufuhr getrennt werden. Sie entspricht der Differenz des \hyperref[sec:elektrisches_potenzial]{elektrischen Potenzials} zwischen diesen Punkten. In verschiedenen elektrischen Feldern ergeben sich aus den geltenden Formeln für das elektrische Potenzial unterschiedliche Formeln für die Spannung. Der Weg, den die Ladung zurücklegt, ist für die Spannung unerheblich.
\begin{gather}
    U_{\mathrm{allg}} = \frac{W}{q} \\
    U_{\mathrm{hom}} = E \cdot d \\
    U_{\mathrm{rad}} = \varphi_{\mathrm{coul,\,r_1}} - \varphi_{\mathrm{coul,\,r_2}} = \frac{Q}{4\pi\varepsilon_0} \cdot \left(\frac{1}{r_1} - \frac{1}{r_2}\right)
\end{gather}
\begin{center}
    \begin{tabular}{rl}
        $W$ & Arbeit \\
        $q$ & Ladung \\
        $E$ & Feldstärke \\
        $d$ & Abstand der Punkte entlang der Feldlinien \\
        $r_1, r_2$ & Abstände zur Punktladung $Q$
    \end{tabular}
\end{center}
Die zweite Formel gilt für homogene Felder, wie sie z.~B. in Plattenkondensatoren zwischen zwei Äquipotenzialflächen mit dem Abstand $d$ auftreten. Die dritte Formel beschreibt radialsymmetrische Felder, wie sie um einzelne Punktladungen entstehen.

Die Einheit der elektrischen Spannung ist
\begin{center}
    $[U] = \si{\volt} = \mathrm{Volt} = \si{\joule\per\coulomb}$
\end{center}

\subsubsection{Elektrische Feldstärke ($E_\mathrm{allg}$, $E_\mathrm{pkond}$, $E_\mathrm{rad}$)}
\label{sec:elektrische_feldstaerke}
Die elektrische Feldstärke $\vec{E}$ ist ein Maß für die Stärke eines elektrischen Feldes und beschreibt die auf eine positive Probeladung $q$ wirkende Kraft $F_\mathrm{el}$ pro Ladungseinheit. Damit charakterisiert sie eine Eigenschaft des Feldes selbst und ist unabhängig von der Wahl der Probeladung.
\begin{gather}
    E_\mathrm{allg} = \frac{F_\mathrm{el}}{q}
\end{gather}
\begin{center}
    \begin{tabular}{rl}
        $F_\mathrm{el}$ & Elektrische Kraft \\
        $q$ & Probeladung
    \end{tabular}
\end{center}
In speziellen Feldern ergeben sich vereinfachte Zusammenhänge. In einem homogenen Feld, wie es im Plattenkondensator auftritt, gilt
\begin{gather}
    E_\mathrm{hom} = \frac{U}{d}
\end{gather}
\begin{center}
    \begin{tabular}{rl}
        $U$ & Angelegte Spannung \\
        $d$ & Plattenabstand
    \end{tabular}
\end{center}
In einem radialsymmetrischen Feld, das von einer Punktladung $Q$ erzeugt wird, folgt aus dem Coulomb-Gesetz folgende Beziehung:
\begin{gather}
    E_\mathrm{rad} = \frac{1}{4\pi\varepsilon_0} \cdot \frac{Q}{r^2}
\end{gather}
\begin{center}
    \begin{tabular}{rl}
        $Q$ & Erzeugende Ladung \\
        $r$ & Abstand von der Ladung \\
        $\varepsilon_0$ & Elektrische Feldkonstante
    \end{tabular}
\end{center}
Die Richtung des Vektors $\vec{E}$ ist stets die Richtung, in die eine positive Probeladung beschleunigt würde. Treffen mehrere Felder zusammen, so überlagern sich die Feldstärken nach dem Superpositionsprinzip vektoriell.

Die Einheit der elektrischen Feldstärke ist
\begin{center}
    $[E] = \si{\newton\per\coulomb} = \si{\volt\per\meter}$
\end{center}

\subsubsection{Elektrische Feldkonstante ($\varepsilon_0$)}
\label{sec:elektrische_feldkonstante}
Die elektrische Feldkonstante $\varepsilon_0$ ist eine fundamentale Naturkonstante, die als Proportionalitätsfaktor in den Gleichungen der Elektrostatik auftritt. Sie verknüpft die Stärke des elektrischen Feldes mit den zugrunde liegenden Ladungsverteilungen und legt damit die Ausbreitungseigenschaften elektrischer Felder im Vakuum fest.

\begin{gather*}
    \varepsilon_0 \approx 8{,}85 \times 10^{-12}\,\si{\coulomb\squared\per\newton\per\meter\squared}
\end{gather*}
In der Feldtheorie verbindet $\varepsilon_0$ die Flächenladungsdichte $\sigma$ mit der erzeugten Feldstärke $E$:
\begin{gather}
    \sigma = \varepsilon_0 \cdot \varepsilon_r \cdot E
\end{gather}
\begin{center}
    \begin{tabular}{rl}
        $\varepsilon_r$ & Relative Permittivität des Mediums
    \end{tabular}
\end{center}
Die Feldkonstante tritt in den Formeln für \hyperref[sec:homogene_felder]{homogene Felder} und für \hyperref[sec:elektrische_feldstaerke]{radialsymmetrische Felder} explizit auf und spielt eine zentrale Rolle in den Maxwell-Gleichungen.

Die Einheit der elektrischen Feldkonstante ist
\begin{center}
    $[\varepsilon_0] = \si{\coulomb\squared\per\newton\per\meter\squared}$
\end{center}

\subsubsection{Elektrische Kapazität ($C_{\mathrm{allg}}$, $C_{\mathrm{pkond}}$, $C_{\mathrm{kugel}}$) und Permittivitätszahl ($\varepsilon_r$)}
\label{sec:elektrische_kapazitaet}
Die elektrische Kapazität gibt an, welche Ladung ein Kondensator oder eine Kugel bei einer bestimmten Spannung $U$ beziehungsweise bei einem Potenzialunterschied speichern kann. Sie wird allgemein definiert als:
\begin{gather}
    C_{\mathrm{allg}} = \varepsilon_r \cdot \frac{Q}{U}
\end{gather}
Da in Luft oder im Vakuum die relative Permittivität $\varepsilon_r \approx 1$ ist, vereinfacht sich dies zu:
\begin{gather}
    C_{\mathrm{allg}} = \frac{Q}{U}
\end{gather}
\begin{center}
    \begin{tabular}{rl}
        $\varepsilon_r$ & Relative Permittivität (Stoffkonstante) \\
        $Q$ & Gespeicherte Ladung \\
        $U$ & Anliegende Spannung
    \end{tabular}
\end{center}
Die Einheit der Kapazität ist
\begin{center}
    $[C] = \si{\farad} = \mathrm{Farad} = \si{\coulomb\per\volt}$
\end{center}
Für die Herleitung der Kapazität eines Plattenkondensators werden die Formeln für die \hyperref[sec:flaechenladungsdichte]{Flächenladungsdichte} $\sigma$ miteinander verknüpft und nach $Q$ umgestellt:
\begin{gather*}
    \sigma_{\mathrm{allg}} = \frac{Q}{A} = \sigma_{\mathrm{hom}} = \varepsilon_0 \cdot E_{\mathrm{allg}} \\
    Q = \varepsilon_0 \cdot E_{\mathrm{allg}} \cdot A
\end{gather*}
Setzt man $E_{\mathrm{allg}} = \tfrac{U}{d}$ ein, ergibt sich:
\begin{align}
    C_{\mathrm{pkond}} &= \varepsilon_r \cdot \frac{\varepsilon_0 \cdot E_{\mathrm{allg}} \cdot A}{U} \nonumber \\
                        &= \varepsilon_r \cdot \frac{\varepsilon_0 \cdot \frac{U}{d} \cdot A}{U} \nonumber \\
    C_{\mathrm{pkond}} &= \varepsilon_0 \cdot \varepsilon_r \cdot \frac{A}{d}
\end{align}
\begin{center}
    \begin{tabular}{rl}
        $\varepsilon_0$ & Elektrische Feldkonstante \\
        $\varepsilon_r$ & Relative Permittivität des Dielektrikums zwischen den Platten \\
        $A$ & Fläche einer Kondensatorplatte \\
        $d$ & Abstand der Platten
    \end{tabular}
\end{center}
Für die Kapazität einer Kugel wird das \hyperref[sec:coulomb_potenzial]{Coulomb-Potenzial} an der Oberfläche verwendet. Für eine Kugel mit Ladung $Q$ und Radius $R$ gilt:
\begin{gather*}
    \varphi_{\mathrm{coul},\,R} = \frac{1}{4\pi\varepsilon_0} \cdot \frac{Q}{R}
\end{gather*}
Das Referenzpotenzial im Unendlichen beträgt $\varphi_{\mathrm{coul},\,\infty} = 0\si{\volt}$, sodass die Spannung gegen Unendlich lautet:
\begin{gather*}
    U = \varphi_{\mathrm{coul},\,R} - \varphi_{\mathrm{coul},\,\infty} = \frac{1}{4\pi\varepsilon_0} \cdot \frac{Q}{R}
\end{gather*}
Die Kapazität der Kugel ergibt sich dann über die allgemeine Definition $C = \varepsilon_r \tfrac{Q}{U}$ zu:
\begin{align}
    C_{\mathrm{kugel}} &= \varepsilon_r \cdot \frac{Q}{\frac{1}{4 \pi \varepsilon_0} \cdot \frac{Q}{R}} \nonumber \\
    C_{\mathrm{kugel}} &= 4 \pi \cdot \varepsilon_0 \varepsilon_r \cdot R
\end{align}
\begin{center}
    \begin{tabular}{rl}
        $\varepsilon_0$ & Elektrische Feldkonstante \\
        $\varepsilon_r$ & Relative Permittivität des Dielektrikums, das die Kugel umgibt \\
        $R$ & Radius der Kugel
    \end{tabular}
\end{center}

\medskip
\begin{tcolorbox}[colframe=red!30!gray, colback=red!10, title=Vertiefung: Veränderung der Kapazität in einem Kondensator]
    Die Erhöhung der Kapazität durch Dielektrika funktioniert folgendermaßen: Das Dielektrikum wird durch das elektrische Feld polarisiert und erzeugt ein Gegenfeld, das das ursprüngliche Feld abschwächt. Da das $d$ in $E = \frac{U}{d}$ konstant bleibt und sich $E$ verkleinert, muss sich die Spannung $U$ ebenfalls verringern. Mit $C_{\mathrm{pkond}} = \frac{Q}{U}$ und konstanter Ladung $Q$ führt die geringere Spannung zu einer größeren Kapazität.
\end{tcolorbox}

\subsubsection{Kraft im elektrischen Feld ($F_{\mathrm{el,\,hom}}$, $F_{\mathrm{el,\,rad}}$)}
\label{sec:kraft_im_elektrischen_feld}
Die elektrische Kraft $F_{\mathrm{el}}$ beschreibt die Wechselwirkung einer Ladung $q$ mit einem elektrischen Feld. Sie ist proportional zur Ladung und zur Stärke des Feldes. In homogenen Feldern gilt:
\begin{gather}
    F_{\mathrm{el,\,hom}} = q \cdot E_{\mathrm{hom}} = \frac{W_{\mathrm{el,\,hom}}}{s}
\end{gather}
\begin{center}
    \begin{tabular}{rl}
        $q$ & Ladung \\
        $E_{\mathrm{hom}}$ & Homogene Feldstärke \\
        $W_{\mathrm{el,\,hom}}$ & Arbeit \\
        $s$ & Wegstrecke
    \end{tabular}
\end{center}
In radialsymmetrischen Feldern, etwa um eine Punktladung $Q$, berechnet sich die Kraft nach dem Coulomb-Gesetz:
\begin{gather}
    F_{\mathrm{el,\,rad}} = q \cdot E_{\mathrm{rad}} = \frac{1}{4\pi \varepsilon_0} \cdot \frac{Q q}{r^2}
\end{gather}
\begin{center}
    \begin{tabular}{rl}
        $Q, q$ & Ladungen \\
        $r$ & Abstand der Ladungen \\
        $\varepsilon_0$ & Elektrische Feldkonstante
    \end{tabular}
\end{center}
Die Richtung der Kraft hängt von der Vorzeichen der Ladung ab: Bei positiver Ladung wirkt die Kraft in Richtung des Feldes, bei negativer entgegen der Feldrichtung.

Das Coulomb-Gesetz für radialsymmetrische Felder ähnelt formal dem Newton'schen Gravitationsgesetz, der Unterschied liegt lediglich in den Vorzeichen und der Art der Wechselwirkung (vgl. Abschnitt~\ref{sec:elektrisches_potenzial}).

\subsubsection{Elektrische Energie ($W_{\mathrm{el,\,allg}}$, $ W_{\mathrm{el,\,hom}}$, $W_{\mathrm{el,\,rad}}$)}
\label{sec:elektrische_energie}
Die elektrische Energie $W_{\mathrm{el}}$ beschreibt die Arbeit, die beim Transport einer Ladung $q$ in einem elektrischen Feld verrichtet wird. Sie hängt von der Ladung und der \hyperref[sec:elektrisches_potenzial]{Potenzialdifferenz} zwischen Start- und Endpunkt ab. Allgemein gilt für eine Ladung, die zwischen zwei Punkten mit der Spannung $U$ bewegt wird:
\begin{gather}
    W_{\mathrm{el,\,allg}} = q \cdot U
\end{gather}
Für den Transport einer Ladung $q$ über eine Strecke $s$ im homogenen elektrischen Feld (konstante Feldstärke $E$), wie etwa zwischen den Platten eines Plattenkondensators, gilt:
\begin{gather}
    W_{\mathrm{el,\,hom}} = F_{\mathrm{el,\,hom}} \cdot s = E_{\mathrm{hom}} \cdot q \cdot s
\end{gather}
Im radialen elektrischen Feld, z.\,B. um eine Punktladung $Q$, berechnet sich die Arbeit für die Bewegung von $q$ von $r_1$ nach $r_2$ über das Integral der Kraft:
\begin{align}
    W_{\mathrm{el,\,rad}} &= \int_{r_1}^{r_2} F_{\mathrm{el,\,rad}} \, dr \nonumber \\
                          &= \frac{Qq}{4 \pi \varepsilon_0} \int_{r_1}^{r_2} \frac{1}{r^2} \, dr \nonumber \\
    W_{\mathrm{el,\,rad}} &= \frac{Qq}{4\pi\varepsilon_0} \cdot \left(\frac{1}{r_1} - \frac{1}{r_2}\right)
\end{align}
\begin{center}
    \begin{tabular}{rl}
        $Q, q$ & Ladungen \\
        $U$ & Spannung \\
        $E_{\mathrm{hom}}$ & Feldstärke \\
        $s$ & Wegstrecke \\
        $r_1, r_2$ & Abstände
    \end{tabular}
\end{center}
Die elektrische Energie ist wegunabhängig: Sie hängt nur von Start- und Endpunkt ab, nicht vom gewählten Weg.

Für kleine Teilchen wird häufig das Elektronenvolt verwendet:
\begin{gather}
    1 \, \mathrm{eV} = 1{,}602 \times 10^{-19} \, \si{\joule}
\end{gather}
Ein Helium-Kern mit zwei Elementarladungen gewinnt beim Durchlaufen einer Spannung von $1\,\si{\volt}$ die Energie $2 \, \mathrm{eV}$.

\subsubsection{Energie im geladenen Kondensator ($W_{\mathrm{kond}}$, $W_{\mathrm{pkond}}$)}
\label{sec:energie_im_geladenen_kondensator}
Die im geladenen Kondensator gespeicherte elektrische Energie $W_{\mathrm{kond}}$ kann auf verschiedene Weisen ausgedrückt werden. Allgemein gilt:
\begin{gather}
    W_{\mathrm{kond}} = \frac{1}{2} Q U = \frac{1}{2} C U^2 = \frac{1}{2} \frac{Q^2}{C}
\end{gather}
Speziell beim Plattenkondensator kann die Energie durch Einsetzen von $C_{\mathrm{pkond}} = \varepsilon_0 \cdot \varepsilon_r \cdot \frac{A}{d}$ und $U_{\mathrm{hom}} = E \cdot d$ berechnet werden. Mit $V = A \cdot d$ als Volumen zwischen den Platten ergibt sich:
\begin{align}
    W_{\mathrm{pkond}} &= \frac{1}{2} C U^2 \\
                       &= \frac{1}{2} \left(\varepsilon_0 \varepsilon_r \frac{A}{d}\right) \left(E_{\mathrm{hom}} d\right)^2 \\
    W_{\mathrm{pkond}} &= \frac{1}{2} \varepsilon_0 \varepsilon_r E_{\mathrm{hom}}^2 V
\end{align}
\begin{center}
    \begin{tabular}{rl}
        $Q$ & Ladung \\
        $C$ & Kapazität \\
        $U$ & Spannung \\
        $V$ & Volumen des elektrischen Feldes
    \end{tabular}
\end{center}
Der Faktor $\frac{1}{2}$ entsteht wie unten erklärt, weil die Spannung während des Ladevorgangs linear ansteigt. Beim Entladen des Kondensators wird die gespeicherte Energie gleichmäßig freigesetzt. Zieht man die Platten eines geladenen Kondensators auseinander, der nicht mit einer Spannungsquelle verbunden ist, steigt die Energie, da das vom Feld eingenommene Volumen zunimmt, obwohl die Ladung unverändert bleibt. Die Energie liegt folglich im elektrischen Feld selbst.

\medskip
\begin{tcolorbox}[colframe=green!30!gray, colback=green!10, title=Vertiefung: Herkunft des Faktors $\frac{1}{2}$]
    Die Formel $W = QU$ von oben würde nur gelten, wenn während des gesamten Ladevorgangs eine konstante Spannung $U$ anliegen würde. In Wirklichkeit steigt die Spannung jedoch kontinuierlich von $0 \, \si{\volt}$ bis zum Endwert $U$ an. Der Faktor $\frac{1}{2}$ entspricht der Dreiecksfläche unter dem Graphen in einem $U$-$Q$-Diagramm. Beim Entladen sinkt die Spannung auf die gleiche Weise ab.
\end{tcolorbox}

\subsubsection{Energiedichte im Plattenkondensator ($\rho_{W,\,\mathrm{pkond}}$)}
\label{sec:energiedichte_im_plattenkondensator}
Die Energiedichte $\rho_{W,\,\mathrm{pkond}}$ beschreibt die im elektrischen Feld eines Plattenkondensators pro Volumeneinheit gespeicherte Energie. Sie ergibt sich aus der gespeicherten Energie $W_{\mathrm{pkond}}$ und dem Volumen $V$ zwischen den Platten:
\begin{gather}
    \rho_{W,\,\mathrm{pkond}} = \frac{W_{\mathrm{pkond}}}{V} = \frac{1}{2} \varepsilon_0 \varepsilon_r E^2.
\end{gather}
\begin{center}
    \begin{tabular}{rl}
        $E$ & Elektrische Feldstärke \\
        $\varepsilon_0, \varepsilon_r$ & Feldkonstante und Permittivitätszahl
    \end{tabular}
\end{center}
Die Energie steckt ausschließlich im elektrischen Feld selbst, nicht in den Ladungen. Die Energiedichte ist proportional zum Quadrat der Feldstärke $E$ und unabhängig davon, wie das Feld erzeugt wurde.

Die Einheit der Energiedichte ist
\begin{center}
    $[\rho_W] = \si{\joule\per\cubic\meter}$
\end{center}

\subsubsection{Be- und Entladen eines Kondensators ($U(t)$, $I(t)$, $Q(t)$, $\tau$)}
\label{sec:be_und_entladen_eines_kondensators}
Beim Entladen eines Kondensators ändern sich Spannung, Strom und gespeicherte Ladung zeitabhängig nach einem exponentiellen Gesetz. Die Größen nehmen jeweils mit der Zeitkonstante
\begin{gather}
    \tau = R C
\end{gather}
ab, die das Tempo des Entladevorgangs bestimmt: Je größer der Widerstand $R$ oder die Kapazität $C$, desto langsamer verläuft die Entladung. Die zeitabhängigen Größen lassen sich durch folgende Gleichungen beschreiben:
\begin{gather}
    U_{\mathrm{entl}}(t) = U_0 \cdot e^{-t / (RC)} \\
    I_{\mathrm{entl}}(t) = -\frac{U_0}{R} \cdot e^{-t / (RC)} \\
    Q_{\mathrm{entl}}(t) = Q_0 \cdot e^{-t / (RC)} = C \cdot U_0 \cdot e^{-t / (RC)} \\
    \tau_{\mathrm{entl}} = RC
\end{gather}
Beim Beladen eines Kondensators steigt die Spannung $U_{\mathrm{bel}}(t)$ exponentiell gegen den Endwert $U_0$. Der Ladestrom $I_{\mathrm{bel}}(t)$ nimmt dabei exponentiell ab, während die gespeicherte Ladung $Q_{\mathrm{bel}}(t)$ mit dem gleichen Verlauf wie die Spannung zunimmt. Die Zeitkonstante $\tau = RC$ bestimmt auch hier, wie schnell der Kondensator den Endwert erreicht:
\begin{gather}
    U_{\mathrm{bel}}(t) = U_0 \cdot \left(1 - e^{-t / (RC)}\right) \\
    I_{\mathrm{bel}}(t) = \frac{U_0}{R} \cdot e^{-t / (RC)} \\
    Q_{\mathrm{bel}}(t) = Q_0 \cdot \left(1 - e^{-t / (RC)}\right) = C \cdot U_0 \cdot \left(1 - e^{-t / (RC)}\right) \\
    \tau_{\mathrm{bel}} = RC
\end{gather}
\begin{center}
    \begin{tabular}{rl}
        $\tau$ & Zeitkonstante \\
        $R$ & Widerstand \\
        $C$ & Kapazität \\
        $U_0, Q_0$ & Maximale Spannung und Ladung
    \end{tabular}
\end{center}
Die Exponentialfunktion beschreibt die schnelle Änderung zu Beginn und die langsame Annäherung an den Endwert. Der Ladestrom ist beim Beladen zu Beginn maximal und nimmt mit zunehmender Spannung ab, während beim Entladen Spannung, Strom und Ladung kontinuierlich abnehmen.

\subsection{Wichtige Phänomene und Ergänzungen}

\subsubsection{Elektrisches Feld}
\label{sec:elektrisches_feld}
Ein elektrisches Feld vermittelt die Kräfte zwischen elektrischen Ladungen. Es beschreibt den Raum, in dem auf eine Probeladung eine Kraft wirkt, und besitzt typische Eigenschaften, die sich anschaulich durch Feldlinien darstellen lassen. Ladungen sind die Quellen und Senken des Feldes: Positive Ladungen erzeugen Feldlinien, während negative Ladungen Feldlinien \enquote{anziehen}.

Feldlinien veranschaulichen die Richtung der Kraft, die auf eine positive Probeladung im elektrischen Feld wirkt. Die Dichte der Feldlinien gibt dabei Aufschluss über die Stärke des Feldes: Je dichter die Feldlinien beieinander liegen, desto stärker ist das elektrische Feld an dieser Stelle. Treffen mehrere Felder, etwa von verschiedenen Ladungen, zusammen, so addieren sich die Feldstärken vektoriell gemäß dem Superpositionsprinzip. Feldlinien verlaufen stets senkrecht auf der Oberfläche von Leitern und konzentrieren sich besonders an Kanten und Spitzen, was dort zu einer erhöhten Feldstärke führt. Äquipotenzialflächen verbinden alle Punkte gleichen elektrischen Potenzials und stehen immer senkrecht zu den Feldlinien.

\medskip
\begin{tcolorbox}[colframe=orange!30!gray, colback=orange!10, title=Vertiefung: Zeichnen von Feldlinien]
    Beim Zeichnen von Feldlinien sind verschiedene Konfigurationen zu beachten:
    \begin{itemize}
        \item Einzelne Punktladungen (positiv/negativ): Radiale Feldlinien - Siehe auch: Dipolfeld
        \item Zwei gleichnamige Ladungen: Abstoßung, Feldlinien krümmen sich voneinander weg
        \item Zwei ungleichnamige Ladungen: Anziehung, Feldlinien verbinden beide Ladungen
        \item Plattenkondensator: Parallele Feldlinien zwischen den Platten
        \item Kondensator mit leitendem Ring: Abschirmung des Feldes innerhalb des Rings
        \item Feldlinien an Kanten und Spitzen: Erhöhte Feldstärke, Feldlinien konzentrieren sich
        \item Feldlinien beim Faraday-Käfig: Das innere des Käfigs ist in Summe feldfrei
    \end{itemize}


 Siehe auch: \url{https://www.leifiphysik.de/elektrizitaetslehre/ladungen-felder-mittelstufe/grundwissen/feldlinien} (Einige Beispiele mit Erklärung)
\end{tcolorbox}

\subsubsection{Äquipotenziallinien}
\label{sec:aequipotenziallinien}
Äquipotenziallinien verbinden Punkte gleichen elektrischen Potenzials in einem Feld. Sie stehen immer senkrecht zu den Feldlinien und verdeutlichen, dass keine Arbeit verrichtet wird, wenn eine Ladung entlang einer Äquipotenziallinie bewegt wird.

In homogenen Feldern, wie sie z.\,B. in Plattenkondensatoren auftreten, sind die Äquipotenziallinien parallel und gleichmäßig verteilt. In radialsymmetrischen Feldern, etwa um Punktladungen, sind die Äquipotenziallinien konzentrisch um die Ladung angeordnet. Bei mehreren Ladungen können die Äquipotenziallinien komplexe Muster bilden, die die Überlagerung der Potenziale widerspiegeln.
\begin{figure}[H]
    \centering
    \begin{minipage}{0.45\linewidth}
        \centering
        \includegraphics[width=\linewidth]{figures/aequipotenziallinien_ungleichnamige_ladungen.png}
        \caption{Äquipotenziallinien bei zwei ungleichnamigen Ladungen.}
        \label{fig:aequipotenziallinien_ungleichnamige_ladungen}
    \end{minipage}\hfill
    \begin{minipage}{0.45\linewidth}
        \centering
        \includegraphics[width=\linewidth]{figures/aequipotenziallinien_gleichnamige_ladungen.png}
        \caption{Äquipotenziallinien bei zwei gleichnamigen Ladungen.}
        \label{fig:aequipotenziallinien_gleichnamige_ladungen}
    \end{minipage}
\end{figure}

\subsubsection{Bewegung geladener Teilchen im elektrischen Feld}
\label{sec:bewegung_geladener_teilchen_im_elektrischen_feld}
Geladene Teilchen bewegen sich im elektrischen Feld nach den Gesetzen der Kinematik, wobei die elektrische Kraft die beschleunigende Wirkung übernimmt.

In einem homogenen Feld wirkt auf eine Ladung $q$ eine konstante Kraft $F_{\mathrm{el,\,hom}}$, die eine gleichmäßige Beschleunigung $a$ verursacht:
\begin{gather}
    a = \frac{F_{\mathrm{el,\,hom}}}{m} = \frac{qE}{m} \\
    v = a \cdot t = \frac{qE \cdot t}{m}
\end{gather}
\begin{center}
    \begin{tabular}{rl}
        $F_{\mathrm{el,\,hom}}$ & Elektrische Kraft \\
        $m, q$ & Masse und Ladung des Teilchens \\
        $E$ & Feldstärke
    \end{tabular}
\end{center}
Die Geschwindigkeit $v$ der Ladung kann auch über deren kinetische Energie $W_{\mathrm{kin}}$ berechnet werden, da die Arbeit des elektrischen Feldes $W_{\mathrm{el,\,allg}}$ mit der Spannung $U$ vollständig in kinetische Energie umgesetzt wird:
\begin{align}
    W_{\mathrm{kin}}        &= W_{\mathrm{el,\,allg}} \nonumber \\
    \frac{1}{2} m \cdot v^2 &= q \cdot U, \quad \nonumber \\
    v                       &= \sqrt{\frac{2qU}{m}}
\end{align}
Bei der Ablenkung eines Elektronenstrahls senkrecht zu den Feldlinien (hier y-Richtung) überlagern sich gleichförmige Bewegung in x-Richtung und gleichmäßig beschleunigte Bewegung in y-Richtung:
\begin{gather}
    s_y(t) = \frac{1}{2} a_y t^2, \quad s_x(t) = v_x \cdot t \\
    a_y = \frac{F_{\mathrm{el,\,hom}}}{m} = \frac{q \cdot E_\mathrm{pkond}}{m} = \frac{q \cdot U_\mathrm{pkond}}{m \cdot d} \\
    s_y(t) = \frac{1}{2} \frac{q \cdot U_\mathrm{pkond}}{m \cdot d} \cdot t^2
           = \frac{1}{2} \frac{q \cdot U_\mathrm{pkond}}{m \cdot d} \cdot \left(\frac{s_x(t)}{v_x}\right)^2
\end{gather}
\begin{center}
    \begin{tabular}{rl}
        $s_x, s_y$ & Auslenkung in x- und y-Richtung \\
        $v_x$ & Eintrittsgeschwindigkeit \\
        $U_\mathrm{pkond}$ & Ablenkspannung
    \end{tabular}
\end{center}
Die Bewegung des Teilchens entspricht dem waagerechten Wurf in der Mechanik. Die horizontale Geschwindigkeitskomponente bleibt konstant, während die Ablenkung proportional zur Ablenkspannung $U_\mathrm{pkond}$ und umgekehrt proportional zum Quadrat der Eintrittsgeschwindigkeit $v_x$ ist.

\subsubsection{Homogene Felder ($E_{\mathrm{hom}}$)}
\label{sec:homogene_felder}
In einem homogenen elektrischen Feld ist die Feldstärke $E$ an jedem Punkt gleich groß und zeigt in die gleiche Richtung. Ein solches Feld kann näherungsweise zwischen den Platten eines Kondensators erzeugt werden.
\begin{gather}
    E_{\mathrm{hom}} = \frac{U}{d}
\end{gather}
Homogene Felder entstehen idealerweise zwischen parallelen Platten eines Plattenkondensators. Ihre Feldlinien verlaufen parallel und mit konstantem Abstand. Die Spannung zwischen den Platten ist bei konstanter Feldstärke $E$ proportional zum Plattenabstand $d$.

\subsubsection{Radialsymmetrische Felder ($E_{\mathrm{rad}}$)}
\label{sec:radialsymmetrische_felder}
Radialsymmetrische elektrische Felder entstehen um Punktladungen oder kugelförmig symmetrisch verteilte Ladungen. Die Feldstärke nimmt mit dem Quadrat der Entfernung vom Zentrum ab und ist damit nicht konstant wie im homogenen Feld.
\begin{gather}
    E_{\mathrm{rad}}(r) = \frac{1}{4\pi\varepsilon_0} \cdot \frac{Q}{r^2}
\end{gather}
Die Feldlinien radialsymmetrischer elektrischer Felder verlaufen radial: bei positiver Ladung nach außen, bei negativer nach innen. Ihre Feldstärke nimmt quadratisch mit der Entfernung $r$ ab. Das Coulomb-Gesetz beschreibt die Kraftwirkung zwischen Punktladungen und bildet die Grundlage dieses Feldes.

 Siehe auch: \url{http://www.physik.osz-biv.de/GK/ph-1_2013/coulomb.php} (Begründung des Proportionalitätsfaktors $\frac{1}{4 \pi \cdot \varepsilon_0}$ über Eigenschaften einer Kugeloberfläche)

\subsubsection{Influenz}
\label{sec:influenz}
Unter Influenz versteht man die Verschiebung freier Ladungsträger in einem neutralen Leiter unter dem Einfluss eines äußeren elektrischen Feldes. Auf der feldnahen Seite sammeln sich dabei Ladungen entgegengesetzter Polarität, während sich auf der feldfernen Seite gleichnamige Ladungen ansammeln. Der Leiter bleibt insgesamt elektrisch neutral, da keine neuen Ladungen entstehen, sondern lediglich vorhandene verschoben werden. Dieser Effekt lässt sich beispielsweise mit einem Elektroskop beobachten.

Im Unterschied zur Polarisation von Isolatoren, bei der sich lediglich gebundene Ladungen geringfügig innerhalb der Moleküle verschieben, erfolgt bei der Influenz eine makroskopische Umverteilung freier Ladungsträger im gesamten Leiter.

\subsubsection{Polarisation}
\label{sec:polarisation}
Unter Polarisation versteht man anders als bei der Influenz die Verschiebung gebundener Ladungen in Atomen oder Molekülen, wenn ein äußeres elektrisches Feld angelegt wird. Es entstehen innerhalb der Teilchen selbst indizierte elektrische Dipole oder vorhandene Dipole richten sich aus. Diese Verschiebung führt zu einer leichten Trennung von positiven und negativen Ladungen, wodurch ein inneres Gegenfeld erzeugt wird, das dem äußeren Feld entgegenwirkt.

In Isolatoren sind die Ladungsträger fest an ihre Positionen gebunden, sodass keine freien Ladungen zur Verfügung stehen. Die Polarisation bewirkt durch die Erzeugung eines Gegenfeldes eine Verringerung der effektiven Feldstärke im Material, was durch die relative Permittivität $\varepsilon_r$ beschrieben wird. Diese Größe gibt an, um welchen Faktor die Kapazität eines Kondensators durch das Einbringen des Dielektrikums erhöht wird.

Bei Molekülen mit dauerhaftem Dipolmoment (z.~B. Wasser) richten sich die Dipole im elektrischen Feld aus, was zur Orientierungspolarisation führt.

\subsection{Apparaturen und Sonstiges}

\subsubsection{Plattenkondensator}
\label{sec:plattenkondensator}
Der Plattenkondensator stellt die einfachste Bauform eines Kondensators dar und erzeugt näherungsweise ein homogenes elektrisches Feld zwischen seinen parallelen Platten.

\begin{gather*}
    E_{\mathrm{hom}} = \frac{U}{d} \\
    \sigma_{\mathrm{hom}} = \varepsilon_0 \cdot E_{\mathrm{hom}} \\
    C_{\mathrm{hom}} = \varepsilon_0 \cdot \varepsilon_r \cdot \frac{A}{d} \\
    F_{\mathrm{el,\,hom}} = \frac{W_{\mathrm{el,\,hom}}}{s} = q \cdot E_{\mathrm{hom}} \\
    W_{\mathrm{hom}} = F_{\mathrm{el,\,hom}} \cdot s = E_{\mathrm{hom}} \cdot q \cdot s \\
    W_{\mathrm{pkond}} = \frac{1}{2} C U^2 = \frac{1}{2} \varepsilon_0 \varepsilon_r E_{\mathrm{hom}}^2 V
\end{gather*}
Wird ein geladener Kondensator, der von einer Spannungsquelle getrennt ist, auseinandergezogen, so steigt die im Kondensator gespeicherte Energie, da der felderfüllte Raum vergrößert wird. Wenn beim Auseinanderziehen der Kondensator weiterhin an eine Spannungsquelle angeschlossen bleibt, ändert sich die Situation im Vergleich zum getrennten Kondensator: Da die Spannung $U$ konstant bleibt, muss die Feldstärke $E_{\mathrm{hom}} = \tfrac{U}{d}$ bei zunehmendem Plattenabstand $d$ abnehmen. Die Kapazität $C = \varepsilon_0 \varepsilon_r \tfrac{A}{d}$ nimmt ebenfalls ab, sodass zuletzt auch die gespeicherte Energie mit zunehmendem Abstand abnimmt.

Kondensatoren können zur Speicherung von Ladung und Energie verwendet werden. Es ist zu beachten, dass die Energie des elektrischen Feldes im Kondensator von der Energie eines darin befindlichen Teilchens zu unterscheiden ist.

 Siehe auch: \url{https://www.leifiphysik.de/elektrizitaetslehre/ladungen-elektrisches-feld/aufgabe/auslenkung-im-homogenen-elektrischen-feld} (Rückstellkraft eines Fadenpendels und Kondensator)

\subsubsection{Elektroskop}
\label{sec:elektroskop}
Das Elektroskop ist ein Instrument zum qualitativen Nachweis elektrischer Ladungen. Es besteht aus einem leitenden Gehäuse und einem beweglichen Zeiger oder Folien, die die gleiche Ladung wie das Gehäuse erhalten, sobald Ladung zugeführt wird.
\begin{figure}[H]
    \centering
    \includegraphics[width=0.35\linewidth]{figures/elektroskop.png}
    \caption{Schematische Darstellung eines Elektroskops.}
    \label{fig:elektroskop}
\end{figure}
Die Wirkung beruht auf der gegenseitigen Abstoßung gleichnamiger Ladungen. Je größer die aufgebrachte Ladung, desto stärker schlägt der Zeiger oder die Folien aus. Zusätzlich lassen sich Influenzeffekte beobachten, wenn geladene Körper in die Nähe des Elektroskops gebracht werden, wodurch sich die Ladungen im Gerät verschieben.

\subsubsection{Faraday-Käfig}
\label{sec:faraday_kaefig}
Ein Faraday-Käfig bezeichnet eine leitende Hülle, die den Innenraum vor äußeren elektrischen Feldern abschirmt. Das zugrunde liegende Prinzip beruht auf der Influenz: Trifft ein äußeres Feld auf den Leiter, so verschieben sich die freien Ladungsträger innerhalb der leitenden Hülle. Auf der dem Feld zugewandten Seite sammeln sich entgegengesetzte Ladungen, während sich auf der abgewandten Seite gleichnamige Ladungen anhäufen. Diese Ladungsverteilung erzeugt ein inneres elektrisches Feld, das dem äußeren Feld genau entgegenwirkt. Dadurch heben sich beide Felder im Inneren auf, sodass der Raum innerhalb des Faraday-Käfigs \enquote{feldfrei} bleibt. Die äußeren Feldlinien enden auf der einen Seite des Leiters und setzen sich von der anderen Seite aus fort.

Die Abschirmung funktioniert unabhängig von der Form des Leiters, solange eine geschlossene oder nahezu geschlossene leitende Hülle vorliegt. Sie gilt jedoch ausschließlich für elektrische Felder; magnetische Felder können so nicht abgeschirmt werden. Die Ladungsverteilung auf der Oberfläche des Käfigs stellt sich dabei selbsttätig so ein, dass die Feldfreiheit im Inneren gewährleistet bleibt.

 Siehe auch: \url{https://www.youtube.com/watch?v=pda98jkZSkg} (Erklärung des Faraday-Käfigs)

\subsubsection{Braun'sche Röhre (Elektronenstrahlröhre)}
\label{sec:braunsche_roehre}
Die Braun'sche Röhre dient zur Demonstration der Beschleunigung und Ablenkung von Elektronen in elektrischen Feldern. Alte Röhrenfernseher, -monitore und Oszilloskope basieren auf diesem Prinzip.
\begin{figure}[H]
    \centering
    \includegraphics[width=0.75\linewidth]{figures/braunsche_roehre.jpeg}
    \caption{Schematische Darstellung einer Elektronenstrahlröhre.}
    \label{fig:braunsche_roehre}
\end{figure}
Elektronen werden von der Glühkathode (negativ) zur Anode (positiv) beschleunigt. Ihre kinetische Energie ergibt sich direkt aus der durchlaufenen Anodenspannung. Ein negativer Wehnelt-Zylinder sorgt für die Fokussierung des Elektronenstrahls, während Ablenkplatten die zweidimensionale Ablenkung ermöglichen. Die Bahn der Elektronen wird durch einen Leuchtschirm sichtbar gemacht. Ihre Bahn folgt den Gesetzen aus Abschnitt~\ref{sec:bewegung_geladener_teilchen_im_elektrischen_feld}.

 Siehe auch: \url{https://www.leifiphysik.de/elektrizitaetslehre/bewegte-ladungen-feldern/ausblick/braunsche-roehre} (Formeln und Animation zur Braunschen Röhre)

\subsection{Versuche}

\paragraph{Kondensatorexperimente}
\begin{itemize}
    \item Ladung und Entladung eines Kondensators über verschiedene Widerstände
    \item Nachweis der Kapazitätsabhängigkeit von Plattenabstand, Plattenfläche und Dielektrikum
    \item Energiemessung beim Auseinanderziehen der Kondensatorplatten
    \item Technische Anwendungen beschreiben (zum Beispiel Standlicht beim Fahrrad)
\end{itemize}

\paragraph{Feldliniendarstellung}
\begin{itemize}
    \item Sichtbarmachung von Feldlinien mit Grießkörnern in Rizinusöl
    \item Untersuchung verschiedener Elektrodenkonfigurationen
    \item Nachweis von Äquipotenziallinien mit Spannungsmessungen
\end{itemize}

\paragraph{Elektronenstrahlen}
\begin{itemize}
    \item Geschwindigkeitsbestimmung von Elektronen aus der Beschleunigungsspannung
    \item Ablenkung im elektrischen Feld (Braunsche Röhre)
    \item Bestimmung der spezifischen Elektronenladung $\frac{e}{m}$
\end{itemize}

\emph{Weitere Versuche hier einfügen.}


\newpage
\section{Magnetismus}
Die nachfolgenden Grundlagen des Magnetismus wurden in der vierten Klausur der elften Klasse abgefragt.

\subsection{Relevante Größen und deren Zusammenhänge}

\subsubsection{Magnetische Flussdichte ($B_{\mathrm{allg}}$, $B_{\mathrm{la,\,sc}}$)}
\label{sec:magnetische_flussdichte}
Die magnetische Flussdichte $\vec{B}$ beschreibt die Stärke eines magnetischen Feldes und ist analog zur elektrischen Feldstärke $\vec{E}$ im elektrischen Feld. Sie gibt an, wie dicht die magnetischen Feldlinien in einem Bereich sind.

Wirkt eine Lorentzkraft $\vec{F}_{\mathrm{lor}}$ senkrecht auf einen Leiter, kann die magnetische Flussdichte in diesem Leiter berechnet werden:
\begin{gather}
    B_{\mathrm{allg}} = \frac{F_{\mathrm{lor}}}{I \cdot l_{\mathrm{leiter}}}
\end{gather}
\begin{center}
    \begin{tabular}{rl}
        $F_{\mathrm{lor}}$ & Lorentzkraft \\
        $I$ & Stromstärke \\
        $l_{\mathrm{leiter}}$ & Leiterlänge im Magnetfeld
    \end{tabular}
\end{center}
$B$ entspricht der Flächendichte des magnetischen Flusses $\Phi$, der senkrecht durch ein Flächenelement $A$ hindurchtritt:
\begin{gather}
    B_{\mathrm{allg}} = \frac{\Phi}{A}
\end{gather}
\begin{center}
    \begin{tabular}{rl}
        $\Phi$ & Magnetischer Fluss \\
        $A$ & Fläche
    \end{tabular}
\end{center}
In langen und schlanken Spulen (d.\,h. Länge $l$ deutlich größer als Durchmesser) gilt:
\begin{gather}
    B_{\mathrm{la,\,sc}} = \mu_0 \cdot \mu_r \cdot \frac{N \cdot I}{l}
\end{gather}
\begin{center}
    \begin{tabular}{rl}
        $N$ & Windungszahl der Spule \\
        $l$ & Länge der Spule \\
        $\mu_r$ & relative Permeabilität des Materials \\
        $\mu_0$ & magnetische Feldkonstante (aus WTR)
    \end{tabular}
\end{center}
Damit lässt sich auch der magnetische Fluss in langen und schlanken Spulen berechnen.

Die Einheit der magnetischen Flussdichte ist
\begin{center}
    $[B] = \si{\tesla} = \mathrm{Tesla} = \si{\newton\per\ampere\per\meter}$
\end{center}

\subsubsection{Magnetischer Fluss ($\Phi$)}
\label{sec:magnetischer_fluss}
Der magnetische Fluss $\Phi$ beschreibt die Menge des Magnetfeldes, die eine bestimmte Fläche durchsetzt. Er hängt sowohl von der Größe der Fläche als auch von der Orientierung des Feldes gegenüber der Fläche ab.

Die Fläche wird durch den Vektor $\vec{A}$ dargestellt, dessen Richtung der Flächennormale entspricht. Seine Länge entspricht der Größe der Fläche. Der Winkel $\theta$ gibt an, wie schräg das Magnetfeld $\vec{B}$ auf die Fläche trifft. Nur die senkrecht zur Fläche verlaufende Komponente des Feldes trägt zum Fluss bei.
\begin{gather}
    \Phi = A_{\perp} \cdot B \cdot \cos(\theta)
\end{gather}
In der vektoriellen Schreibweise wird dies kompakt durch das Skalarprodukt dargestellt. Diese Form ist besonders nützlich, wenn die Orientierung von Fläche und Magnetfeld nicht einfach geometrisch beschrieben werden kann:
\begin{gather}
    \Phi = \vec{A} \cdot \vec{B}
\end{gather}
Die Einheit des magnetischen Flusses ist
\begin{center}
    $[\Phi] = \si{\weber} = \mathrm{Weber} = \si{\tesla\meter\squared} = \si{\volt\second}$
\end{center}

\subsubsection{Lorentzkraft ($\vec{F}_{\mathrm{lor,\,leiter}}$, $\vec{F}_{\mathrm{lor,\,frei}}$)}
\label{sec:lorentzkraft}
Die Lorentzkraft $\vec{F}_{\mathrm{lor}}$ beschreibt die Kraft, die auf einen stromdurchflossenen Leiter oder auf frei bewegliche Ladungen in einem Magnetfeld wirkt. Sie entsteht durch die Wechselwirkung zwischen dem elektrischen Strom beziehungsweise den bewegten Ladungen und dem Magnetfeld $\vec{B}$.

Für einen Leiter mit Strom $I$ wirkt die Kraft auf den Abschnitt der Länge $l_{\mathrm{leiter}}$, der senkrecht zur Richtung des Magnetfeldes steht:
\begin{gather}
    \vec{F}_{\mathrm{lor,\,leiter}} = I \cdot l_{\mathrm{leiter}} \cdot \vec{B}
\end{gather}
Für einzelne freie Ladungen kann die Lorentzkraft aus der Stromformel abgeleitet werden. Mit $I = \frac{\Delta Q}{\Delta t}$ und $\frac{\Delta l}{\Delta t} = v$ folgt:
\begin{align}
    \vec{F}_{\mathrm{lor,\,frei}} &= I \, l_{\mathrm{leiter}} \, \vec{B} \nonumber \\
    &= \frac{\Delta Q}{\Delta t} \, l_{\mathrm{leiter}} \, \vec{B} \nonumber \\
    &= Q \, \frac{\Delta l}{\Delta t} \, \vec{B} \nonumber \\
    \vec{F}_{\mathrm{lor,\,frei}} &= Q \, \vec{v} \times \vec{B} \Rightarrow Q \cdot v \cdot B
\end{align}
\begin{center}
    \begin{tabular}{rl}
        $\vec{v}$ & Geschwindigkeit der Ladung $Q$ (z.\,B. eines Elektrons) \\
        $Q$ & Ladung (z.\,B. Ladung eines Elektrons  $-e$) \\
        $\vec{B}$ & Magnetische Flussdichte
    \end{tabular}
\end{center}
Das Kreuzprodukt \enquote{$\times$} zeigt an, dass die Lorentzkraft immer senkrecht zu Bewegungsrichtung und Magnetfeld wirkt. Ohne das Kreuzprodukt gilt die Formel nur für Ladungen, die sich senkrecht zum Magnetfeld bewegen.

\subsubsection{Induktivität ($L$)}
\label{sec:induktivität}
Die Induktivität $L$ ist eine physikalische Größe, die angibt, wie stark eine Spule oder ein Leiterstromkreis eine Änderung des Stromes durch die Induktion einer Spannung behindert. Sie hängt von der Geometrie der Spule, der Anzahl der Windungen und dem verwendeten Kernmaterial ab.

Die Induktivität $L$ ist definiert als das Verhältnis zwischen dem magnetischen Fluss $\Phi$ und dem Strom $I$:
\begin{gather}
    L = N \cdot \frac{\Phi}{I}
\end{gather}
Für eine lange, schlanke Spule der Länge $l$, Querschnittsfläche $A$ und $N$ Windungen im Medium mit Permeabilität $\mu_0 \mu_r$ ergibt sich:
\begin{gather}
    L = \mu_0 \mu_r \cdot \frac{N^2 \cdot A}{l}
\end{gather}
Bei der Selbstinduktion gilt für den Einschaltvorgang mit $I(0) = 0$:
\begin{align}
    U(t) &= U_0 + U_{\mathrm{ind}}(t) \\
    &= U_\mathrm{netz} - L \cdot \dot{I}(t) \\
    R_\mathrm{sp} \cdot I(t) &= U_\mathrm{netz} - L \cdot \dot{I}(t) \\
    L &= \frac{U_\mathrm{netz} - R_\mathrm{sp} \cdot I(0)}{\dot{I}(0)} \\
    &= \frac{U_\mathrm{netz}}{\dot{I}(0)}
\end{align}
\begin{center}
    \begin{tabular}{rl}
        $U_\mathrm{netz}$ & Netzspannung (Quellspannung) \\
        $R_\mathrm{sp}$ & Spulenwiderstand \\
        $\dot{I}(0)$ & Stromanstiegsgeschwindigkeit zum Zeitpunkt $t=0$
    \end{tabular}
\end{center}
$\dot{I(0)}$ ist die Anfangsänderung des Stromes, also die Steigung der Strom-Zeit-Kurve zum Zeitpunkt $t = 0$, was z.\,B. an einem Graphen abgelesen werden kann.

Die Einheit der Induktivität ist
\begin{center}
    $[L] = \si{\henry} = \mathrm{Henry} = \si{\volt\second\per\ampere}$
\end{center}

\subsubsection{Induktionsspannung ($U_{\mathrm{ind}}$)}
\label{sec:induktionsspannung}
Die Induktionsspannung $U_{\mathrm{ind}}$ ist die Spannung, die in einer Spule oder Leiterschleife entsteht, wenn sich der durch sie hindurchtretende magnetische Fluss $\Phi$ ändert. Sie ist ein direktes Ergebnis des Faraday’schen Induktionsgesetzes.

Für eine Spule mit $N_{\mathrm{ind}}$ Windungen gilt allgemein:
\begin{align}
    U_{\mathrm{ind}} &= - N_{\mathrm{ind}} \cdot \left( \frac{\Phi(t_2) - \Phi(t_1)}{t_2 - t_1} \right) \nonumber \\
                     &= - N_{\mathrm{ind}} \cdot \frac{d\Phi}{dt} \\
                     &= - N_{\mathrm{ind}} \cdot \dot{\Phi}(t)
\end{align}
Bleiben entweder die durchsetzte Fläche $A_{\perp}$ oder die magnetische Flussdichte $B$ konstant, ergeben sich vereinfachte Einzelformen:
\begin{align}
    U_{\mathrm{ind,\,konst.\,}A_{\perp}} &= - N_{\mathrm{ind}} \cdot A_{\perp} \cdot \frac{\Delta B}{\Delta t} \\
    U_{\mathrm{ind,\,konst.\,}B} &= - N_{\mathrm{ind}} \cdot B \cdot \frac{\Delta A_{\perp}}{\Delta t}
\end{align}
Unter der Annahme, dass sowohl $B(t)$ als auch $A_{\perp}(t)$ zeitabhängig sind, kann die zusammengesetzte Form über die Produktregel dargestellt werden:
\begin{gather}
    \Phi(t) = B(t) \cdot A_{\perp}(t) \\
    U_{\mathrm{ind}}(t) = - N_{\mathrm{ind}} \cdot \frac{d\Phi}{dt} \\
    \frac{d\Phi}{dt} = A_{\perp}(t) \cdot \frac{dB}{dt} + B(t) \cdot \frac{dA_{\perp}}{dt} \\
    \Rightarrow U_{\mathrm{ind}}(t) = - N_{\mathrm{ind}} \left( A_{\perp}(t) \cdot \frac{dB}{dt} + B(t) \cdot \frac{dA_{\perp}}{dt} \right) \\
    U_{\mathrm{ind}}(t) = U_{\mathrm{ind,\,konst.\,}A_{\perp}}(t) + U_{\mathrm{ind,\,konst.\,}B}(t)
\end{gather}
Bei der \hyperref[sec:bewegungsinduktion]{Bewegungsinduktion} ändert sich die durchsetzte Fläche $A_{\perp}$ durch die Bewegung eines Leiters (oft $N = 1$) im Magnetfeld. In diesem Fall wird die Induktionsspannung durch die Änderung der Fläche in Abhängigkeit der Geschwindigkeitbestimmt:
\begin{equation*}
    U_{\mathrm{ind,\,bew}} = -B \cdot \frac{dA}{dt} = -B \cdot \frac{d(x \cdot l)}{dt} = - B \cdot l \cdot \frac{dx}{dt} = - B \cdot l \cdot v_{\perp}
\end{equation*}
Bei der \hyperref[sec:selbstinduktion]{Selbstinduktion} wird die Induktionsspannung durch die Änderung des eigenen Stromes in der Spule verursacht. In diesem Fall gilt eine spezielle Form des Induktionsgesetzes:
\begin{align}
    U_{\mathrm{ind}} &= - N \cdot \frac{{d\Phi }}{{dt}} \nonumber \\
                     &= - N \cdot A \cdot \frac{{dB}}{{dt}} \nonumber \\
                     &= - N \cdot A \cdot \frac{{d\left( {{\mu _0} \mu_r \cdot \frac{N}{l} \cdot I} \right)}}{{dt}} \nonumber \\
                     &= - N \cdot A \cdot {\mu _0} \mu_r \cdot \frac{N}{l} \cdot \frac{{dI}}{{dt}} \nonumber \\
                     &= - {\mu _0} \mu_r  \cdot \frac{{A \cdot {N^2}}}{l} \cdot \frac{{dI}}{{dt}} \nonumber \\
    U_{\mathrm{ind,\,sel}} &= - L \cdot \frac{dI}{dt} = - L \cdot \dot{I}
\end{align}

\subsubsection{Energie im Magnetfeld ($W_{\mathrm{mag}}$)}
\label{sec:energie_im_magnetfeld}
In einem Magnetfeld kann Energie gespeichert werden, ähnlich wie in einem elektrischen Feld. Besonders deutlich wird dies bei einer stromdurchflossenen Spule, die ein Magnetfeld aufbaut. Der Stromfluss muss Arbeit gegen die induzierte Gegenspannung verrichten, wodurch Energie im Magnetfeld gespeichert wird.

Die gespeicherte magnetische Energie in einer Spule mit Induktivität $L$ und Strom $I$ ergibt sich zu:
\begin{gather}
    W_{\mathrm{mag}} = \frac{1}{2} \, L \, I^2
\end{gather}
Der Strom wächst an. Dabei verrichtet die Spannungsquelle Arbeit gegen die Selbstinduktionsspannung. Diese Arbeit wird in Form von magnetischer Energie im Feld gespeichert. Solange der Strom konstant ist, bleibt die Energie im Feld selbst gespeichert. Beim Ausschalten, also beim Abbau des Stroms, wird die gespeicherte Energie wieder frei, entweder geht sie wieder an den Stromkreis und wirkt dem Abbau des Stroms entgegen oder sie wird durch Widerstände in Wärme umgesetzt. Die Formel gilt immer dann, wenn sich durch eine Spule mit Induktivität $L$ bereits ein stationärer Strom $I$ eingestellt hat oder der momentane Wert $I(t)$ bekannt ist.

\subsection{Wichtige Konzepte und Vertiefung}

\subsubsection{Magnet}
\label{sec:magnet}
Ein Magnet ist ein Körper, der ein Magnetfeld erzeugt und dadurch bestimmte Materialien wie Eisen, Nickel oder Kobalt anziehen kann. Jeder Magnet besitzt stets zwei Pole, den Nordpol und den Südpol. Gleichnamige Pole stoßen einander ab, während ungleichnamige sich anziehen. Man unterscheidet zwischen Dauermagneten, die ein permanentes Magnetfeld besitzen, und Elektromagneten, deren Magnetwirkung nur während des Stromflusses auftritt.

\subsubsection{Magnetfeld}
\label{sec:magnetfeld}
Ein Magnetfeld ist der Raum um einen Magneten, in dem magnetische Kräfte wirken. Es wird durch Feldlinien dargestellt, die stets vom Nordpol zum Südpol verlaufen. Die Dichte der Feldlinien zeigt die Feldstärke an: Je dichter die Linien, desto stärker das Magnetfeld.
Ein Magnetfeld übt auf bewegte geladene Teilchen eine Kraft aus, die als Lorentzkraft $F_{\mathrm{lor}}$ bezeichnet wird. Die Richtung des Feldes lässt sich beispielsweise mit Eisenfeilspänen oder einer Kompassnadel sichtbar machen.

\subsubsection{Induktion}
\label{sec:induktion}
Ändert sich ein Magnetfeld oder die magnetische Flussdichte $\vec{B}$ in einer Leiterschleife, so wird eine Induktionsspannung $U_{\mathrm{ind}}$ erzeugt. Dies ist die Grundlage des Faraday'schen Induktionsgesetzes.
Eine Induktionsspannung tritt immer dann auf, wenn sich eine der Größen ändert, die den magnetischen Fluss $\Phi$ durch die Leiterschleife bestimmen:
\begin{itemize}
    \item die magnetische Flussdichte $\vec{B}$ des Feldes,
    \item die von der Leiterschleife eingeschlossene Fläche $A$,
    \item und theoretisch auch der Winkel $\varphi$ zwischen der Flächennormalen der Leiterschleife und der Richtung des Magnetfeldes.
\end{itemize}
Der magnetische Fluss durch eine Leiterschleife im rechten Winkel zum Magnetfeld ist gegeben durch
\begin{gather}
    \Phi = \vec{B} \cdot \vec{A} = B \, A,
\end{gather}
wobei sich eine Änderung einer der drei Größen direkt in einer Induktionsspannung widerspiegelt:
\begin{gather}
    U_{\mathrm{ind}} = - N_{\mathrm{ind}} \frac{d\Phi}{dt}.
\end{gather}

\subsubsection{Bewegungsinduktion ($U_{\mathrm{ind,\,bew}}$)}
\label{sec:bewegungsinduktion}
Bewegungsinduktion ist die Erzeugung einer Induktionsspannung $U_{\mathrm{ind,\,bew}}$, wenn ein Leiter sich relativ zu einem Magnetfeld bewegt, sodass sich der magnetische Fluss durch den Leiter ändert.
Bei einem Leiter, der auf zwei Leiterschienen orthogonal durch ein konstantes Magnetfeld gleitet, scheint sich zunächst keine der für die Induktion relevanten Größen verändert zu werden. Dennoch ist eine Induktionsspannung $U_{\mathrm{ind,\,bew}}$ messbar. Während $B$ konstant bleibt, verändert sich hier die wirksame Fläche $A$. Dies liegt daran, dass sich durch die Bewegung des Leiter der vom diesem Leiter und den Schienen umschlossene Flussbereich ändert.

In diesem Fall ändert sich $A$ mit der Zeit. Nach dem Induktionsgesetz mit $N_{\mathrm{ind}} = 1$:
\begin{equation*}
    U_{\mathrm{ind}} = -\frac{d\Phi}{dt} = -\frac{d(B \cdot A)}{dt}
\end{equation*}
Da B konstant ist, folgt für eine rechteckige Fläche der Größe $l_{\mathrm{konstant}} \cdot x$, die vom Magnetfeld durchsetzt wird:
\begin{equation*}
    U_{\mathrm{ind,\,bew}} = -B \cdot \frac{dA}{dt} = -B \cdot \frac{d(x \cdot l)}{dt} = - B \cdot l \cdot \frac{dx}{dt} = - B \cdot l \cdot v_{\perp}
\end{equation*}
Das gleiche Ergebnis lässt sich anschaulich über die Lorentzkraft nach dem Prinzip des Halleffekts nachvollziehen. Die durch den Stab in eine Richtung mitgenommenen Elektronen erfahren eine Lorentzkraft $F_{\mathrm{lor}}$, wodurch eine Ladungstrennung und damit die Spannung $U_{\mathrm{ind}}$ entsteht. Dies geschieht so lange bis $F_{\mathrm{el}} = F_{\mathrm{lor}}$. Dann gilt folgendes:
\begin{equation}
    U_{\mathrm{ind,\,bew}} = B \cdot v \cdot l
\end{equation}

\subsubsection{Selbstinduktion ($U_{\mathrm{ind,\,sel}}$)}
\label{sec:selbstinduktion}
Selbstinduktion beschreibt den Effekt, dass in einer Spule durch eine zeitliche Änderung des eigenen Stromes eine Induktionsspannung entsteht, die der Ursache der Stromänderung entgegenwirkt (Lenz'sche Regel).

Ändert sich der Strom $I(t)$ auf den Wert $I_0$ in einer Spule mit der Windungszahl $N$ und Induktivität $L$, ändert sich der magnetische Fluss und in der Spule selbst wird eine Spannung induziert. Der Strom geht nicht sofort auf seinen stationären Endwert ${I_0} = \frac{U_0}{R}$, sondern steigt allmählich auf diesen Endwert an. Das liegt daran, dass $I(t) = \frac{U(t)}{R} = \frac{U_0 - U_{\mathrm{ind,\,sel}}(t)}{R}$ gilt. Die Induktionsspannung lässt sich aus dem Induktionsgesetz herleiten:
\begin{align}
    U_{\mathrm{ind}} &= - N \cdot \frac{{d\Phi }}{{dt}} \nonumber \\
                     &= - N \cdot A \cdot \frac{{dB}}{{dt}} \nonumber \\
                     &= - N \cdot A \cdot \frac{{d\left( {{\mu _0} \mu_r \cdot \frac{N}{l} \cdot I} \right)}}{{dt}} \nonumber \\
                     &= - N \cdot A \cdot {\mu _0} \mu_r \cdot \frac{N}{l} \cdot \frac{{dI}}{{dt}} \nonumber \\
                     &= - {\mu _0} \mu_r  \cdot \frac{{A \cdot {N^2}}}{l} \cdot \frac{{dI}}{{dt}} \nonumber \\
    U_{\mathrm{ind,\,sel}} &= - L \cdot \frac{dI}{dt}
\end{align}
\begin{center}
    \begin{tabular}{rl}
        $A$ & Querschnittsfläche der Spule \\
        $l$ & Länge der Spule \\
        $\mu_0$ & magnetische Feldkonstante (aus WTR) \\
        $\mu_r$ & relative Permeabilität des Materials \\
        $N$ & Windungszahl der Spule \\
        $L$ & Induktivität der Spule \\
        $I$ & Aktueller Strom durch die Spule
    \end{tabular}
\end{center}
Diese Spannung wirkt der Stromänderung entgegen und begrenzt somit die Geschwindigkeit, mit der sich der Strom aufbaut oder abbaut.

Die Induktivität $L$ hängt von der Geometrie der Spule, der Windungszahl $N$ und dem magnetischen Kernmaterial ab. Je größer $L$, desto stärker wirkt die Selbstinduktion.
Ein typisches Beispiel ist der Einschaltvorgang einer Spule in einem Gleichstromkreis: Beim Einschalten steigt der Strom nicht sofort auf seinen Maximalwert, sondern exponentiell, da die Spule eine Gegeninduktionsspannung erzeugt. Beim Ausschalten wirkt die gleiche Spannung in entgegengesetzter Richtung.

 Siehe auch: \url{https://www.leifiphysik.de/elektrizitaetslehre/elektromagnetische-induktion/grundwissen/selbstinduktion-und-induktivitaet} (Weiterführende Informationen zur Selbstinduktion) und \url{https://physikbuch.schule/inductance-and-self-inductance.html} (Darstellung der Spannung und Stromstärke in Abhängigkeit der Zeit)

\subsubsection{Lenz'sche Regel}
\label{sec:lenzsche_regel}
Die Lenz'sche Regel besagt, dass jeder Induktionsstrom stets so gerichtet ist, dass er der Ursache seiner Entstehung entgegenwirkt.

 Siehe auch: \url{https://www.leifiphysik.de/elektrizitaetslehre/elektromagnetische-induktion/grundwissen/lenzsche-regel} (Weiterführende Informationen zur Lenz'schen Regel)

\subsubsection{Linke-Hand-Regel}
\label{sec:linke_hand_regel}
Die Linke-Hand-Regel dient dazu, die Richtung der Lorentzkraft auf einen stromdurchflossenen Leiter im Magnetfeld zu bestimmen, wenn man mit den negativen Ladungsträgern, also Elektronen, arbeitet. Dabei zeigt der Daumen in die Richtung des Elektronenflusses (von Minus nach Plus), der Zeigefinger in die Richtung der Magnetfeldlinien (vom Nordpol zum Südpol) und der Mittelfinger in die Richtung der wirkenden Lorentzkraft $\vec{F}_{\mathrm{lor}}$.

Sind Magnetfeld und Elektronenflussrichtung parallel zueinander, so verschwindet die Lorentzkraft ($F_{\mathrm{lor}} = 0$).

\subsubsection{Linke-Faust-Regel}
\label{sec:linke_faust_regel}
Die Linke-Faust-Regel wird verwendet, um die Richtung des Magnetfeldes um einen stromdurchflossenen Leiter zu bestimmen, wenn Elektronen (negative Ladungsträger) von Minus nach Plus fließen. Dabei umschließen die Finger der linken Hand den Leiter, während der Daumen in Richtung des Elektronenflusses zeigt. Die gekrümmten Finger geben dann die Richtung des Magnetfeldes an, dessen Feldlinien kreisförmig um den Leiter verlaufen.

\subsubsection{Hall-Effekt ($U_{\mathrm{hall}}$)}
\label{sec:hall_effekt}
Der Hall-Effekt beschreibt das Entstehen einer messbaren Spannung $U_{\mathrm{hall}}$ quer zu einem stromdurchflossenen Leiter, wenn Elektronen sich senkrecht zu einem Magnetfeld bewegen. Er ermöglicht eine direkte Bestimmung der magnetischen Flussdichte $\vec{B}$, auch in inhomogenen Feldern, wo andere Verfahren versagen.
Die Ursache ist die Ablenkung der Elektronen durch die Lorentzkraft $\vec{F}_{\mathrm{lor}}$, die eine Ladungstrennung im Leiter hervorruft. Dadurch entsteht ein elektrisches Feld, das eine entgegengesetzte elektrische Kraft $\vec{F}_{\mathrm{el}}$ ausübt. Im Gleichgewicht gilt $\vec{F}_{\mathrm{el}} = \vec{F}_{\mathrm{lor}}$, und die Hallspannung bleibt konstant messbar.
\begin{enumerate}
    \item Elektronen bewegen sich senkrecht zu einem Magnetfeld $\vec{B}$ und werden gemäß der \hyperref[sec:linke_hand_regel]{Linke-Hand-Regel} abgelenkt.
    \item Es entsteht eine Ladungstrennung im Leiter, wodurch sich ein elektrisches Feld $E$ aufbaut.
    \item Im Gleichgewicht heben sich Lorentzkraft und elektrische Kraft gegenseitig auf:
    \begin{gather}
        F_{\mathrm{el}} = F_{\mathrm{lor,\,frei}}
    \end{gather}
    \item Zwischen den gegenüberliegenden Seiten des Leiters bildet sich die Hallspannung:
    \begin{gather}
        \frac{U_{\mathrm{hall}} \cdot q}{h} = e \cdot v \cdot B \\
        U_{\mathrm{hall}} = B \cdot v \cdot h \\
        B = \frac{U_{\mathrm{hall}}}{v \cdot h}
    \end{gather}
    \begin{center}
    \begin{tabular}{rl}
        $h$ & Breite des Leiters, auf der die Elektronen getrennt werden \\
        $v$ & Geschwindigkeit der Elektronen \\
        $q = e$ & Ladung der Elektronen
    \end{tabular}
\end{center}
\end{enumerate}

\subsubsection{Kreisbahnen mit Lorentzkraft als Zentripetalkraft ($r$, $m$)}
\label{sec:kreisbahnen_mit_lorentzkraft_als_zentripetalkraft}
Bewegt sich ein geladenes Teilchen senkrecht zu den Magnetfeldlinien $\vec{B}$, ohne dass ein elektrisches Feld zur Kompensation vorhanden ist, so wirkt die Lorentzkraft $\vec{F}_{\mathrm{lor}}$ als Zentripetalkraft $\vec{F}_{\mathrm{zp}}$. Das Teilchen beschreibt dadurch eine Kreisbahn. Aus dieser Bewegung lassen sich der Radius $r$ der Kreisbahn sowie die Masse $m$ des Teilchens bestimmen.
\begin{align}
    |\vec{F}_{\mathrm{zp}}| &= |\vec{F}_{\mathrm{lor}}| \nonumber \\
    \frac{m \cdot v^2}{r} &= q \cdot v \cdot |\vec{B}| \nonumber \\
    r &= \frac{m \cdot v}{q \cdot |\vec{B}|} \\
    m &= \frac{q \cdot |\vec{B}| \cdot r}{v}
\end{align}

\subsubsection{Elektrische Wirbelströme}
\label{sec:elektrische_wirbelströme}
Wirbelströme sind elektrische Ströme, die in leitfähigen Materialien induziert werden, wenn sich das Magnetfeld in ihrer Nähe ändert. Sie entstehen durch die Induktionswirkung, wenn ein Leiter in ein sich änderndes Magnetfeld gebracht wird oder wenn sich das Magnetfeld selbst ändert.
Die Wirbelströme fließen in geschlossenen Schleifen innerhalb des Materials und erzeugen dabei eigene Magnetfelder, die der Änderung des ursprünglichen Magnetfeldes entgegenwirken (Lenz'sche Regel). Dies führt zu Energieverlusten in Form von Wärme, was in Anwendungen wie Transformatoren und Elektromotoren berücksichtigt werden muss.
Um Wirbelströme zu minimieren, werden in elektrischen Geräten oft dünne, isolierte Schichten aus leitfähigem Material verwendet, die als Laminate bezeichnet werden. Diese Schichten unterbrechen die Wirbelstrompfade und reduzieren so die Verluste.

\subsubsection{Polarlicht}
\label{sec:polarlicht}
Polarlichter (Aurora borealis auf der Nordhalbkugel und Aurora australis auf der Südhalbkugel) treten in der oberen Atmosphäre ab etwa 80 km Höhe in der Nähe der magnetischen Pole auf. Ursache ist der Sonnenwind, ein Strom geladener Teilchen, vor allem Elektronen und Protonen, die von der Sonne ins All geschleudert werden.
Wenn diese Teilchen auf das Magnetfeld der Erde treffen, werden sie durch die Lorentzkraft abgelenkt und entlang der Feldlinien in eine sogenannte magnetische Flasche gelenkt, in der sie zwischen Nord- und Südpol hin- und herpendeln können.
In der Nähe der Pole dringen die Teilchen schließlich in die Atmosphäre ein und stoßen mit Atomen und Molekülen, vor allem Sauerstoff und Stickstoff, zusammen. Dabei werden diese angeregt; beim Rückfall in den Grundzustand wird Licht ausgesendet, das als sichtbares Polarlicht erscheint. Die Farbe hängt vom jeweiligen Gas und der Höhe der Emission ab, etwa grünes Licht bei angeregtem Sauerstoff in ca. 100 km Höhe.

\subsection{Apparaturen und Sonstiges}

\subsubsection{Fadenstrahlrohr}
\label{sec:fadenstrahlrohr}
Das Fadenstrahlrohr dient zur Bestimmung der spezifischen Ladung $\tfrac{e}{m}$, also des Verhältnisses von Elementarladung $e$ zur Elektronenmasse $m$.
Elektronen werden in einer Glühkathode freigesetzt, durch eine Beschleunigungsspannung beschleunigt und treten mit Geschwindigkeit $v$ in das homogene Magnetfeld zwischen zwei Helmholtz-Spulen ein. Dort wirkt die Lorentzkraft $\vec{F}_{\mathrm{lor,\,frei}}$ senkrecht zur Bewegungsrichtung und zwingt die Elektronen auf eine Kreisbahn mit Radius $r$.
Die für die Kreisbewegung notwendige Zentripetalkraft wird vollständig durch die Lorentzkraft aufgebracht:
\begin{align}
    F_{\mathrm{zp}} &= F_{\mathrm{lor,\,frei}} \\
    \frac{m \cdot v^2}{r} &= e \cdot v \cdot B \\
    \frac{e}{m} &= \frac{v}{B \cdot r}
\end{align}
Mit bekannter Elementarladung $e$ lässt sich so die Elektronenmasse $m$ bestimmen. Der Elektronenstrahl ist durch die von ihm verursachte Leuchterscheinung im Restgas sichtbar.
\begin{figure}[H]
    \centering
    \includegraphics[width=0.5\linewidth]{figures/fadenstrahlrohr.png}
    \caption{Schematische Darstellung eines Fadenstrahlrohrs.}
    \label{fig:fadenstrahlrohr}
\end{figure}

 Siehe auch:\url{https://www.leifiphysik.de/elektrizitaetslehre/bewegte-ladungen-feldern/versuche/fadenstrahlrohr} (Erläuterungen und interaktive Simulation)

\subsubsection{Wienfilter ($v$)}
\label{sec:wienfilter}
Der Wienfilter dient zur Geschwindigkeitsselektion geladener Teilchen. Er basiert auf der Überlagerung eines elektrischen Feldes $\vec{E}$ und eines magnetischen Feldes $\vec{B}$, die zueinander sowie zur Bewegungsrichtung der Teilchen senkrecht stehen.
\begin{figure}[H]
    \centering
    \includegraphics[width=0.4\linewidth]{figures/wienfilter.png}
    \caption{Schematische Darstellung eines Wienfilters.}
    \label{fig:wienfilter}
\end{figure}
Bedingung für einen ungestörten Durchgang der Teilchen ist, dass die elektrische Kraft und die Lorentzkraft betragsmäßig gleich groß und entgegengesetzt gerichtet sind:
\begin{align}
    |\vec{F}_{\mathrm{el}}| &= |\vec{F}_{\mathrm{lor}}| \nonumber \\
    q \cdot |\vec{E}| &= q \cdot v \cdot |\vec{B}|
\end{align}
Damit ergibt sich für die Geschwindigkeit:
\begin{gather}
    v = \frac{|\vec{E}|}{|\vec{B}|}
\end{gather}
Sind die beiden Kräfte nicht im Gleichgewicht, werden die Teilchen je nach Geschwindigkeit zu stark oder zu schwach abgelenkt und erreichen den Ausgang nicht. Die resultierende Bahnkurve lässt sich analog zum waagerechten Wurf verstehen, bei dem auf das Teilchen die Vektorsumme $\vec{F}_{\mathrm{el}} + \vec{F}_{\mathrm{lor}}$ wirkt.

 Siehe auch: \url{https://www.leifiphysik.de/elektrizitaetslehre/bewegte-ladungen-feldern/grundwissen/wienscher-geschwindigkeitsfilter} (Weiterführende Erklärung)

\subsubsection{Massenspektrometer}
\label{sec:massenspektrometer}
Ein Massenspektrometer ist ein Messgerät, mit dem die Masse oder die Ladung von Teilchen bestimmt werden kann.
Zunächst werden die Teilchen durch einen \hyperref[sec:wienfilter]{Wienfilter} auf eine einheitliche Geschwindigkeit $v$ selektiert. Anschließend treten sie in ein Magnetfeld $\vec{B}$ ein, dessen Feldlinien senkrecht zur Bewegungsrichtung der Teilchen verlaufen. Dort wirkt auf sie die Lorentzkraft $\vec{F}_{\mathrm{lor}}$, sodass sie auf einer halbkreisförmigen Bahn abgelenkt werden.

Am Ende treffen die Teilchen auf einer Detektor- bzw. Messplatte auf. Der beobachtete Ablenkungsradius $r$ hängt von der Masse $m$, der Ladung $q$ und der Geschwindigkeit $v$ ab. Da $q$ und $v$ bekannt sind, kann die Masse der Teilchen berechnet werden:
\begin{align}
    |\vec{F}_{\mathrm{zp}}| &= |\vec{F}_{\mathrm{lor}}| \nonumber \\
    \frac{m \cdot v^2}{r} &= q \cdot v \cdot |\vec{B}| \nonumber \\
    m &= \frac{q \cdot |\vec{B}| \cdot r}{v}
\end{align}

\subsubsection{Generator}
\label{sec:generator}
Ein Generator ist eine technische Anwendung des Induktionsgesetzes, mit der mechanische Energie in elektrische Energie umgewandelt wird. Er bildet die Grundlage für die Stromerzeugung in Kraftwerken.
Eine Leiterschleife (Spule) rotiert in einem Magnetfeld. Dabei ändert sich der magnetische Fluss $\Phi$ durch die Spule periodisch. Nach dem Faraday'schen Induktionsgesetz wird dadurch eine Spannung $U_{\mathrm{ind}}$ induziert:
\begin{gather}
    U_{\mathrm{ind}}(t) = - N_{\mathrm{ind}} \cdot \dot{\Phi}
\end{gather}
Für eine rechteckige Spule der Fläche $A$ im homogenen Magnetfeld $\vec{B}$ gilt:
\begin{gather}
    \Phi(t) = B \cdot A \cdot \cos(\omega t)
\end{gather}
Daraus ergibt sich für den Wechselstromgenerator eine sinusförmige Wechselspannung, die über Schleifringe abgenommen wird:
\begin{gather}
    U_{\mathrm{ind}}(t) = N_{\mathrm{ind}} \cdot B \cdot A \cdot \omega \cdot \sin(\omega t)
\end{gather}
Bei einem Gleichstromgenerator wandelt ein Kommutator die Wechselspannung in eine Gleichspannung um, indem er die Polarität der Spannung bei jeder halben Umdrehung umkehrt. Dadurch fließt im äußeren Stromkreis immer ein gleichgerichteter Strom. Das Prinzip entpricht dem umgekehrten Aufbau eines Elektromotors, bei dem elektrische Energie in mechanische Energie umgewandelt wird.
\begin{figure}[H]
    \centering
    \includegraphics[width=0.5\linewidth]{figures/generator.png}
    \caption{Schematische Darstellung eines Wechselstromgenerators.}
    \label{fig:generator}
\end{figure}

\subsubsection{Transformator}
\label{sec:transformator}
Ein Transformator ist eine Anwendung des Induktionsgesetzes, die zur Umwandlung von Wechselspannungen dient. Er besteht aus einer Primärspule mit $N_1$ Windungen und einer Sekundärspule mit $N_2$ Windungen, die beide über einen gemeinsamen Eisenkern magnetisch gekoppelt sind. Fließt durch die Primärspule ein Wechselstrom, so erzeugt er ein zeitlich veränderliches Magnetfeld im Eisenkern. Dieses Magnetfeld durchsetzt die Sekundärspule und induziert dort eine Spannung.

Zwischen den Spannungen der beiden Spulen gilt das Windungsgesetz:
\begin{gather}
    \frac{U_1}{U_2} = \frac{N_1}{N_2}
\end{gather}
Ein Transformator mit $N_2 > N_1$ erhöht die Spannung und wird als Hochtransformator bezeichnet. Umgekehrt verringert ein Transformator mit $N_2 < N_1$ die Spannung und wirkt als Tieftransformator. Unter der Annahme idealer Bedingungen ohne Energieverluste gilt für die Ströme:
\begin{gather}
    \frac{I_1}{I_2} = \frac{N_2}{N_1}
\end{gather}
Somit bleibt die übertragene Leistung konstant:
\begin{align}
    P_1 &= P_2 \\
    U_1 \cdot I_1 &= U_2 \cdot I_2
\end{align}
In der Realität treten allerdings Verluste wie durch Wirbelströme und ohmsche Widerstände auf, die den Wirkungsgrad leicht verringern.
\begin{figure}[H]
    \centering
    \vspace{-1cm}
    \includegraphics[width=0.5\linewidth]{figures/transformator.png}
    \vspace{-1cm}
    \caption{Schematischer Aufbau eines Transformators.}
    \label{fig:transformator}
\end{figure}

\subsubsection{Induktionskochplatte}
\label{sec:induktionskochplatte}
Eine Induktionskochplatte nutzt das Prinzip der elektromagnetischen Induktion, um Wärme direkt im Kochgeschirr zu erzeugen. Unter der Glaskeramikplatte befindet sich eine Spule, durch die ein hochfrequenter Wechselstrom fließt. Dadurch entsteht ein zeitlich veränderliches Magnetfeld, das den Boden des Kochgeschirrs durchsetzt.
Im leitfähigen Boden des Topfes werden Wirbelströme induziert. Diese Ströme erfahren aufgrund des elektrischen Widerstands des Materials Verluste, die sich in Wärme umwandeln. Zusätzlich trägt die sogenannte magnetische Hysterese im ferromagnetischen Material zur Erwärmung bei.

Die Wärme entsteht somit direkt im Kochgeschirr und nicht auf der Herdplatte selbst, was den Energieverlust verringert und die Effizienz erhöht. Voraussetzung ist, dass das Kochgeschirr ferromagnetisch ist (z.\,B. Eisen oder spezieller Edelstahl), da nur so ein ausreichendes Magnetfeld gekoppelt werden kann.

Verändert sich der Strom durch die Spule, so verändert sich auch das Magnetfeld über dem Kochfeld. Nach der \hyperref[sec:lenzsche_regel]{Lenz'schen Regel} sind die im Topfboden entstehenden Wirbelströme stets so gerichtet, dass ihr eigenes Magnetfeld der Ursache der Induktion entgegenwirkt. Mithilfe der \hyperref[sec:linke_faust_regel]{Linke-Faust-Regel} lässt sich die Richtung des von der Spule erzeugten Magnetfeldes bestimmen. Es stellen sich Wirbelströme ein, die ebenfalls ein Magnetfeld erzeugen, das dem durch den Strom in der Spule erzeugten Magnetfeld entgegenwirkt. Diese Ströme verlaufen also kreisförmig im Boden des Kochgeschirrs und erzeugen durch den elektrischen Widerstand die gewünschte Wärmeentwicklung. Neben den Wirbelströmen wird auch durch die Rotation der Teilchen im ferromagnetischen Material infolge der Umpolung des Magnetfeldes durch den Topfboden Wärme erzeugt (Hystereseverluste).
\begin{figure}[H]
    \centering
    \includegraphics[width=0.6\linewidth]{figures/induktionsherd.png}
    \caption{Schematische Darstellung einer Induktionskochplatte.}
    \label{fig:induktionskochplatte}
\end{figure}

\subsubsection{Induktionsschleife}
\label{sec:induktionsschleife}
Eine Induktionsschleife ist eine Anwendung des Faraday’schen Induktionsgesetzes. Dabei handelt es sich um eine Leiterschleife (meist eine Drahtschleife), die in die Fahrbahn eingelassen ist und über eine Auswerte-Elektronik mit einem Wechselstrom gespeist wird. Durch den Strom fließt ein Magnetfeld, das die Schleife durchsetzt.

Wenn ein Fahrzeug (mit überwiegend metallischer Karosserie) über die Induktionsschleife fährt, verändert es das Magnetfeld in der Umgebung der Schleife: Zum einen werden Wirbelströme im Fahrzeugboden induziert, zum anderen verändert das Metall die magnetische Leitfähigkeit des Raumes. Dadurch ändert sich der effektive magnetische Fluss $\Phi$ durch die Schleife. Der Wechselstrom sorgt dafür, dass ein zeitlich veränderliches Magnetfeld erzeugt wird. Dieses veränderliche Feld induziert auch bei ruhenden Metallteilen Wirbelströme. Der Effekt ist messbar, unabhängig davon, ob sich das Fahrzeug bewegt oder nicht.

Nach dem Induktionsgesetz
\begin{gather}
    U_{\mathrm{ind}} = -N \, \frac{d\Phi}{dt}
\end{gather}
entsteht eine messbare Induktionsspannung. Sie führt zu einer Änderung der Resonanzfrequenz des Oszillators, mit dem die Schleife betrieben wird. Die Elektronik erkennt diese Änderung und registriert so, dass ein Fahrzeug die Schleife überfährt.

Induktionsschleifen werden im Straßenverkehr vielseitig eingesetzt, etwa zur Steuerung von Ampelanlagen, zur Erfassung von Fahrzeugzahlen oder zur Auslösung von Schranken.

\subsection{Versuche}

\subsubsection{Versuch 1: Leiterschleife in Hufeisenmagneten}
\label{sec:versuch1_leiterschleife}
\emph{Versuchsaufbau:}
Eine Leiterschleife befindet sich innerhalb eines Hufeisenmagneten, so dass sie frei schwingen kann. An die Leiterschleife wird eine elektrische Spannung $U$ angelegt, wodurch ein Strom $I$ fließt.
\emph{Beobachtung:}
Die Leiterschleife wird aus ihrer Ruheposition ausgelenkt.
\emph{Erklärung:}
Die Auslenkung der Leiterschleife entsteht durch die auf den stromdurchflossenen Leiter wirkende Lorentzkraft $\vec{F}_{\mathrm{lor}}$, die senkrecht auf die Richtung des Stromes $\vec{I}$ und des Magnetfeldes $\vec{B}$ wirkt. Die Kraft ist maximal, wenn $\vec{I} \perp \vec{B}$ steht.
\begin{gather}
    |\vec{F}_{\mathrm{lor}}| = I \cdot l_{\mathrm{leiter}} \cdot |\vec{B}|
\end{gather}

\subsubsection{Versuch 2: Parallele Leiter}
\label{sec:versuch2_parallele_leiter}
\emph{Versuchsaufbau:}
Zwei leitende Drähte hängen senkrecht und parallel in einem Gerüst. An beide wird die gleiche Spannung $U$ angelegt, wodurch ein Strom $I$ fließt.
\emph{Beobachtung:}
Die beiden Drähte ziehen sich gegenseitig an.
\emph{Erklärung:}
Die Anziehung der Drähte entsteht durch die Wechselwirkung der Magnetfelder, die jeder Draht aufgrund des Stromes erzeugt. Jeder Draht erzeugt ein kreisförmiges Magnetfeld $\vec{B}$ um sich, dessen Richtung mit der \hyperref[sec:linke_faust_regel]{Linke-Faust-Regel} bestimmt werden kann. Die Lorentzkraft $\vec{F}_{\mathrm{lor}}$ auf den jeweils anderen Draht ist senkrecht zu Stromrichtung und Magnetfeld:
\begin{gather}
    |\vec{F}_{\mathrm{lor}}| = I \cdot l_{\mathrm{leiter}} \cdot |\vec{B}|
\end{gather}
\begin{figure}[H]
    \centering
    \begin{tikzpicture}[scale=4.5, every node/.style={font=\small}]
        % Drähte (Stromrichtung: in die Ebene), Abstand = 0.5
        \node at (0,0) {\huge\textbf{$\otimes$}};
        \node at (-0.05,-0.1) {A};

        \node at (0.5,0) {\huge\textbf{$\otimes$}};
        \node at (0.55,-0.1) {B};

        % Magnetfeldlinien von A (gegen Uhrzeigersinn), Radius = 0.5
        \draw[->] (0.5,0)   arc (0:90:0.5);
        \draw[->] (0,0.5)   arc (90:180:0.5);
        \draw[->] (-0.5,0)  arc (180:270:0.5);
        \draw[->] (0,-0.5)  arc (270:360:0.5);

        % Magnetfeldlinien von B (gegen Uhrzeigersinn), Radius = 0.5
        \draw[->] (1.0,0)       arc (0:90:0.5);
        \draw[->] (0.5,0.5)     arc (90:180:0.5);
        \draw[->] (0,0)         arc (180:270:0.5);
        \draw[->] (0.5,-0.5)    arc (270:360:0.5);

        % Lorentzkraft-Pfeile (rot)
        \draw[->, thick, red] (0,0.0)   -- (0.2,0.0)   node[midway, above] {$\vec{F}_{\mathrm{lor}}$};
        \draw[->, thick, red] (0.5,0.0) -- (0.3,0.0)   node[midway, above] {$\vec{F}_{\mathrm{lor}}$};

        % Magnetfeldrichtung (blau)
        \draw[->, thick, blue] (0,0.0)      -- (0,-0.2)     node[right] {$\vec{B}_A$};
        \draw[->, thick, blue] (0.5,0.0)    -- (0.5,0.2)    node[right] {$\vec{B}_B$};
    \end{tikzpicture}
    \caption{...}
    \label{fig:versuch2_parallele_leiter}
\end{figure}

\section{Versuche}

\begin{itemize}
    \item Superposition von Magnetfeldern
    \item Induktion zweier benachbarter Spulen
    \item Thomsonscher Ringversuch
\end{itemize}

\emph{Weitere Versuche hier einfügen.}


\newpage
\section{Elektromagnetismus}
Die nachfolgenden Grundlagen des Elektromagnetismus wurden in der ersten Klausur der zwölften Klasse abgefragt.

\subsection{Relevante Größen und deren Zusammenhänge}

\emph{Hier ergänzen!}

\subsection{Wichtige Konzepte und Vertiefung}

\subsubsection{Elektromagnetisches Spektrum}
\label{sec:elektromagnetisches_spektrum}
Das elektromagnetische Spektrum umfasst die Gesamtheit aller elektromagnetischen Wellen, die sich durch ihre Wellenlänge $\lambda$ und Frequenz $f$ unterscheiden. Es reicht von den langwelligen Radiowellen bis zu den extrem kurzwelligen Gammastrahlen.
Radiowellen ($\lambda \approx 1\,\text{m}$ bis $1\,\text{mm}$, $f \approx 300\,\text{MHz}$ bis $300\,\text{GHz}$) werden etwa für Rundfunk, Radar oder in Mikrowellenherden eingesetzt. Daran schließen sich Mikrowellen und Infrarotstrahlung an. Infrarotstrahlung ($\lambda \approx 1\,\text{mm}$ bis $780\,\text{nm}$) wird unter anderem für Fernbedienungen, Wärmebildkameras und Temperaturmessungen genutzt.
Das sichtbare Licht bildet nur einen sehr kleinen Bereich des Spektrums ($\lambda \approx 780\,\text{nm}$ bis $380\,\text{nm}$) und ist für das menschliche Auge wahrnehmbar. Kurzwelliger ist die Ultraviolettstrahlung ($\lambda \approx 380\,\text{nm}$ bis $1\,\text{nm}$), die technische Anwendungen wie Geldscheinprüfung, Desinfektion oder das Aushärten von Klebstoffen ermöglicht.
Noch energiereicher sind Röntgenstrahlen ($\lambda \approx 1\,\text{nm}$ bis $10\,\text{pm}$), die in der Medizin zur Bildgebung genutzt werden. Am unteren Ende der Wellenlänge liegen schließlich die Gammastrahlen ($\lambda < 10\,\text{pm}$), die beim radioaktiven Zerfall oder in Teilchenreaktionen entstehen und eine sehr hohe Eindringtiefe besitzen.

\subsubsection{Lichtgeschwindigkeit im Vakuum}
\label{sec:lichtgeschwindigkeit_im_vakuum}
Die Lichtgeschwindigkeit im Vakuum ist eine fundamentale Naturkonstante und beträgt exakt $c = \SI{299792458}{\meter\per\second}$. Diese Geschwindigkeit ist unabhängig von der Bewegung der Lichtquelle oder des Beobachters und stellt die maximale Geschwindigkeit dar, mit der sich Informationen oder Energie im Universum ausbreiten können.
In Medien wie Luft, Wasser oder Glas ist die Lichtgeschwindigkeit geringer als im Vakuum. Der Brechungsindex $n$ eines Mediums beschreibt das Verhältnis der Lichtgeschwindigkeit im Vakuum zur Lichtgeschwindigkeit im Medium:
\begin{gather}
    n = \frac{c}{v}
\end{gather}
\begin{center}
    \begin{tabular}{rl}
        $c$ & Lichtgeschwindigkeit im Vakuum \\
        $v$ & Lichtgeschwindigkeit im Medium
    \end{tabular}
\end{center}
Die Lichtgeschwindigkeit spielt eine zentrale Rolle in der speziellen Relativitätstheorie, die von Albert Einstein entwickelt wurde. Sie besagt, dass die Gesetze der Physik in allen Inertialsystemen gleich sind und dass die Lichtgeschwindigkeit im Vakuum für alle Beobachter konstant ist, unabhängig von deren Relativbewegung zueinander.
Heute wird die Lichtgeschwindigkeit mit Hilfe von Lasern und hochpräzisen Zeitmessungen bestimmt. Seit 1983 ist die Lichtgeschwindigkeit im Vakuum als Naturkonstante definiert und beträgt exakt $c = \SI{299792458}{\meter\per\second}$.

\subsubsection{Elektromagnetische Wellen}
\label{sec:elektromagnetische_wellen}
Elektromagnetische Wellen ...

\subsubsection{Maxwell-Gleichungen}
\label{sec:maxwell_gleichungen}
Die Maxwell-Gleichungen sind ein System von vier partiellen Differentialgleichungen, die die fundamentalen Gesetze des Elektromagnetismus beschreiben. Sie wurden von James Clerk Maxwell im 19. Jahrhundert formuliert und fassen alle bekannten Phänomene der Elektrizität und des Magnetismus zusammen. Aus ihnen lässt sich die Existenz elektromagnetischer Wellen ableiten, die sich mit Lichtgeschwindigkeit im Vakuum ausbreiten. 

Die vier Gleichungen in der differenziellen Form lauten:
\begin{enumerate}
    \item \emph{Gauß'sches Gesetz für elektrische Felder:} Ein elektrisches Feld $E$ hat dort eine Divergenz, wo sich Ladungen befinden. $\rho$ steht für die Ladungsdichte. Die Feldlinine beginnen an positiven Ladungen und enden an negativen Ladungen. Divergenz von $E$ bedeutet, dass das $E$-Feld seine Quellen bei einer Ladung hat.
    \begin{gather}
        \nabla \cdot \vec{E} = \frac{\rho}{\varepsilon_0} 
    \end{gather}
    \item \emph{Gauß'sches Gesetz für magnetische Felder:} Die Divergenz von $B$ ist überall null. Das bedeutet: Es gibt keine Monopole. Magnetische Feldlinien sind immer geschlossen und bilden Schleifen. Divergenz von $B$ bedeutet, dass die $B$-Linien geschlossen sind.
    \begin{gather}
        \nabla \cdot \vec{B} = 0 
    \end{gather}
    \item \emph{Faraday'sches Induktionsgesetz:} Ein sich änderndes $B$-Feld erzeugt ein elektrisches Feld. $\delta B / \delta t$ beschreibt die Änderung von B mit der Zeit. Das erklärt die Induktion in Spulen und die Funktionsweise von Generatoren und Transformatoren. Rotation von $E$ entsteht durch die Veränderung des $B$-Feldes.
    \begin{gather}
        \nabla \times \vec{E} = -\frac{\partial \vec{B}}{\partial t}
    \end{gather}
    \item \emph{Ampère-Maxwell-Gesetz:} Ein Magnetfeld wird durch Strom oder durch ein sich änderndes $E$-Feld erzeugt. Hierbei ist $j$ die Stromdichte. Der Zusatzterm $\mu_0 \varepsilon_0 \partial E / \partial t$ heißt auch Verschiebungsstrom. Rotation von $B$ entsteht durch die Veränderung des $E$-Feldes. Der Verschiebungsstrom sorgt dafür, dass das Ampère'sche Gesetz im Vakuum gilt.
    \begin{gather}
        \nabla \times \vec{B} = \mu_0 \vec{J} + \mu_0 \varepsilon_0 \frac{\partial \vec{E}}{\partial t} 
    \end{gather}
\end{enumerate}
Die vier Gleichungen lauten in ihrer integralen Form:
\begin{enumerate}
    \item \emph{Gauß'sches Gesetz für elektrische Felder:} Der elektrische Fluss durch eine geschlossene Oberfläche ist proportional zur eingeschlossenen elektrischen Ladung. Es beschreibt, wie elektrische Ladungen elektrische Felder erzeugen.
    \begin{gather}
        \oint_A \vec{E} \cdot d\vec{A} = \frac{Q}{\varepsilon_0}
    \end{gather}
    \item \emph{Gauß'sches Gesetz für magnetische Felder:} Der magnetische Fluss durch eine geschlossene Oberfläche ist immer null. Dies bedeutet, dass es keine magnetischen Monopole gibt; Magnetfelder sind immer als geschlossene Feldlinien konfiguriert. 
    \begin{gather}
        \oint_A \vec{B} \cdot d\vec{A} = 0
    \end{gather}
    \item \emph{Faraday'sches Induktionsgesetz:} Ein sich zeitlich änderndes Magnetfeld erzeugt ein elektrisches Wirbelfeld. Dies ist die Grundlage der elektromagnetischen Induktion und erklärt die Funktionsweise von Generatoren und Transformatoren.
    \begin{gather}
        \oint_C \vec{E} \cdot d\vec{s} = - \frac{d\Phi_B}{dt}
    \end{gather}
    \item \emph{Ampère-Maxwell-Gesetz:} Ein elektrischer Strom oder ein sich zeitlich änderndes elektrisches Feld (Maxwell'scher Verschiebungsstrom) erzeugt ein magnetisches Wirbelfeld. Diese Erweiterung des ursprünglichen Ampère'schen Gesetzes war entscheidend für die Vorhersage elektromagnetischer Wellen.
    \begin{gather}
        \oint_C \vec{B} \cdot d\vec{s} = \mu_0 \cdot I + \mu_0 \cdot \frac{d\Phi_E}{dt}
    \end{gather}
\end{enumerate}

 Siehe auch: \url{https://physikbuch.schule/maxwells-equations.html#maxwells-equations} (Weitere Erklärung der Maxwell'schen Gleichungen)

\emph{Hier ergänzen!}

\subsection{Apparaturen und Sonstiges}

\subsubsection{Elektromagnetischer Schwingkreis}
\label{sec:elektromagnetischer_schwingkreis}
Ein elektromagnetischer Schwingkreis besteht aus einer Spule (Induktivität $L$) und einem Kondensator (Kapazität $C$), die in Serie oder parallel geschaltet sind. Wenn der Kondensator aufgeladen wird und dann entladen wird, fließt ein Strom durch die Spule, wodurch ein magnetisches Feld aufgebaut wird. Sobald der Kondensator entladen ist, baut das magnetische Feld der Spule eine Spannung auf, die den Kondensator wieder auflädt, jedoch mit umgekehrter Polarität. Dieser Prozess wiederholt sich, wodurch eine Schwingung entsteht.
Der geladene Kondensator besitzt elektrische Feldenergie. Diese treibt die Ladungen beim Entladen des Kondensators an und es fließt ein elektrischer Strom durch die Spule. Nach der Lenzschen Regel behindert die Spule zunächst den Stromfluss, indem sie ein magnetisches Feld aufbaut, das diesem Strom entgegenwirkt (Selbstinduktion). Ist der Kondensator schließlich entladen, ist die Energie als magnetische Feldenergie in der Spule gespeichert. Der Stromfluss würde jetzt zum Erliegen kommen, aber nach der Lenzsche Regel erhält die Spule den Stromfluss weiter aufrecht. Als Energiequelle dient das Magnetfeld, das sich dabei wieder abbaut. Die Ladungen bewegen sich weiter auf die gegenüberliegende Kondensatorplatte und bilden dort ein elektrisches Feld mit umgekehrter Polung. Jetzt wiederholt sich der Vorgang in umgekehrter Richtung.
\begin{figure}[H]
    \centering
    \includegraphics[width=0.75\linewidth]{figures/schwingkreis.png}
    \caption{Schematische Darstellung eines Schwingkreises.}
    \label{fig:schwingkreis}
\end{figure}
So wie jeder mechanische Oszillator besitzt auch ein Schwingkreis eine Eigenfrequenz. Die Eigenfrequenz eines Schwingkreises kann durch die \hyperref[sec:thomsonsche_schwingungsgleichung]{Thomson'sche Schwingungsgleichung} berechnet werden:
\begin{gather}
    \omega_0 = \frac{1}{\sqrt{L C}} \\
    f_0 = \frac{1}{2 \pi \sqrt{L C}}
\end{gather}
\begin{center}
    \begin{tabular}{rl}
        $L$ & Induktivität \\
        $C$ & Kapazität
    \end{tabular}
\end{center}

 Siehe auch: \url{https://physikbuch.schule/tuned-circuits.html#lc-circuit-natural-angular-frequency-derived} (Herleitung der Eigenfrequenz eines Schwingkreises)

\subsubsection{Thomson'sche Schwingungsgleichung eines Schwingkreises}
\label{sec:thomsonsche_schwingungsgleichung}
Betrachten wir den elektrischen Schwingkreis als ein geschlossenes System, so ist die Summe aller Energieformen in diesem System zu jeder Zeit $t$ konstant:
\begin{gather}
    W_{\mathrm{el}}(t) + W_{\mathrm{mag}}(t) = W_{\mathrm{ges}}
\end{gather}
Setzt man die entsprechenden Formeln ein, so kommt man auf folgende Differentialgleichung:
\begin{gather*}
    \frac{1}{2 C} Q^2(t) + \frac{1}{2} L I^2(t) = W_{\mathrm{ges}}
\end{gather*}
Aus
\begin{gather*}
    I(t) = \frac{dQ(t)}{dt} = \dot{Q}(t)
\end{gather*}
folgt:
\begin{gather*}
    \frac{1}{2 C} Q^2(t) + \frac{1}{2} L \dot{Q}^2(t) = W_{\mathrm{ges}}
\end{gather*}
Nun leitet man diese Gleichung nach der Zeit ab und erhält:
\begin{align}
    \frac{1}{C} Q \dot{Q}(t) + L \dot{Q} \ddot{Q}(t) &= 0 \nonumber \\
    I(t) \left(L \ddot{Q} + \frac{1}{C} Q(t)\right) &= 0 \nonumber \\
    L \ddot{Q} + \frac{1}{C} Q(t) &= 0 \nonumber \\
    \ddot{Q} + \frac{1}{L C} Q(t) &= 0
\end{align}
Das entspricht der bekannten Differentialgleichung für eine harmonische Schwingung mit $\omega = \frac{1}{\sqrt{LC}}$. Man darf durch $I(t)$ teilen, da im Schwingkreis $I(t) \neq 0$. 

Auch über die Spannung lässt sich die Thomson'sche Schwingungsgleichung herleiten. Es gilt nach der Maschenregel, dass die Summe aller Spannungen in einem geschlossenen Stromkreis null ist:
\begin{align}
    U_C(t) + U_L(t) &= 0 \\
    \frac{Q}{C} + \abs{- L \cdot \dot{I}(t)} &= 0 \nonumber
\end{align}
Nun leitet man diese Gleichung nach der Zeit ab und erhält:
\begin{align}
    \frac{\dot{Q}(t)}{C} + L \ddot{I}(t) &= 0 \nonumber \\
    \frac{I(t)}{C} + L \ddot{I}(t) &= 0 \nonumber \\
    \ddot{I}(t) + \frac{1}{L C} I(t) &= 0
\end{align}
Auch hier erhält man die bekannte Differentialgleichung für eine harmonische Schwingung mit $\omega = \frac{1}{\sqrt{LC}}$.

\subsubsection{Hertz'scher Dipol}
\label{sec:hertzscher_dipol}
Der Hertz'sche Dipol stellt den Grenzfall eines elektromagnetische Schwingkreises dar. Er besteht aus einem kurzen, geraden Leiterstück, das von einem Wechselstrom durchflossen wird. Durch die Beschleunigung der Ladungen im Leiterstück entstehen elektromagnetische Wellen, die sich im Raum ausbreiten. 

Ein \hyperref[sec:elektromagnetischer_schwingkreis]{Elektromagnetischer Schwingkreis} besteht aus einer Spule und einem Kondensator, die zusammen eine Resonanzfrequenz besitzen. Im Hertz'sche Dipol wird versucht, dessen Frequenz besonders weit zu erhöhen:
\begin{align}
    f &= \frac{1}{2 \pi \sqrt{L C}} \\
    &= \frac{1}{2 \pi \sqrt{\mu_0 \mu_r \frac{N^2 A}{l} \cdot \varepsilon_0 \varepsilon_r \frac{A}{d}}} \nonumber
\end{align}
Um die Frequenz zu erhöhen, muss also $L$ und $C$ möglichst klein gehalten werden. Dazu wird die Spule durch einen geraden Leiter ersetzt, sodass $N = 1$ und $A \approx 0$ gilt. Die exakte Frequenz kann durch eine Anpassung der Länge des Leiters angepasst werden.

\subsubsection{Mikrowellenherd}
\label{sec:mikrowellenherd}
Ein Mikrowellenherd nutzt elektromagnetische Wellen im Mikrowellenbereich ($\lambda \approx 12\,\text{cm}$, $f \approx 2{,}45\,\text{GHz}$), um Nahrungsmittel von innen zu erwärmen. Dabei wird die elektrische Feldkomponente der Wellen ausgenutzt, um polare Moleküle, wie Wasser, in den Lebensmitteln in ständiger Rotation zu versetzen. Diese Molekülorientierungen verursachen Reibung, die als Wärme in den Lebensmitteln umgesetzt wird. 

Die Mikrowellen werden durch einen Magnetron erzeugt, ein spezieller Hochfrequenzgenerator. Über einen Hohlleiter werden die Mikrowellen in den Garraum geleitet, wo sie sich ausbreiten und auf die Lebensmittel treffen. Reflektierende Innenwände sorgen für eine möglichst gleichmäßige Verteilung der Mikrowellen. 

Die Energieübertragung erfolgt hauptsächlich durch die Wechselwirkung des elektrischen Feldes mit den Dipolen in den Molekülen. Die Wirksamkeit hängt von der Frequenz, der Permittivität der Substanz und der Wassergehalt der Lebensmittel ab. Materialien wie Glas, Keramik oder Kunststoff werden kaum erwärmt, da sie keine freien oder polarisierten Ladungen enthalten. 

Die Vorteile des Mikrowellenherds liegen in der schnellen und effizienten Erwärmung, da die Energie direkt in den Lebensmitteln umgesetzt wird, im Gegensatz zu konventionellen Herden, bei denen Wärme über Konvektion oder Wärmeleitung übertragen wird.

% Zu behandelnde Themen:
% X Elektromagnetisches Spektrum
% - Eigenschaften von elektromagnetischen Wellen und Gravitationswellen vergleichen (zum Beispiel Ausbreitungsgeschwindigkeit, Ausbreitung im Vakuum, Transversalwellen)
% - kohärentes Licht als elektromagnetische Welle beschreiben (unter anderem Lichtgeschwindigkeit)
% - Ursache und Struktur elektromagnetischer Felder anhand der Aussagen der Maxwell-Gleichungen im Überblick beschreiben
% X Schwingkreis mit Schwingungsdifferenzialgleichung beschreiben, Energieumwandlungen im Schwingkreis erklären, ...
% - Gemeinsamkeiten und Unterschiede von mechanischen und elektromagnetischen Schwingungen erläutern (zum Beispiel anhand eines Federpendels und eines elektromagnetischen Schwingkreises)
% - den Hertz’schen Dipol als Grenzfall eines elektromagnetischen Schwingkreises erkennen und die daraus entstehende Abstrahlung elektromagnetischer Wellen in Grundzügen beschreiben
% - Mikrowellenherd
% X Thomson'sche Schwingungsgleichung
% - Funktionsweise von RFID-Chips
% X Maxwell-Gleichungen
% - elektromagnetische Wellen

\newpage
\section[Quantenphysik und Wellen]{Wellenoptik und Quantenmechanik}
Die nachfolgenden Grundlagen der Wellenoptik sowie die der Quantenmechanik wurden in der zweiten Klausur der zwölften Klasse abgefragt.

\subsection{Relevante Größen und deren Zusammenhänge}

\subsubsection{Planck-Konstante ($h$)}
\label{sec:planck_konstante}
Die Planck-Konstante $h$ ist eine fundamentale Naturkonstante der Quantenphysik.
Sie legt fest, wie stark Energie und Frequenz quantisiert sind und markiert die Grenze, ab der klassische Physik versagt.
\begin{gather}
    h \approx 6{,}626 \times 10^{-34} \, \si{\joule\second} \\
    h \approx 4{,}136 \times 10^{-15} \, \si{\electronvolt\second}
\end{gather}

\subsubsection{Energie eines Photons ($W_\gamma$)}
\label{sec:energie_photonen}
Die Energie eines Photons $W_\gamma$ ist proportional zur Wellenfrequenz $f$ des Lichts, das auf die Elektrode einfällt.
\begin{gather}
    W_\gamma = h \cdot f = \frac{h \cdot c}{\lambda}
\end{gather}
\begin{center}
    \begin{tabular}{rl}
        $h$ & Planck-Konstante \\
        $f$ & Frequenz des Lichts \\
        $c$ & Lichtgeschwindigkeit \\
        $\lambda$ & Wellenlänge
    \end{tabular}
\end{center}

\subsubsection{Impuls eines Photons ($p_\gamma$)}
\label{sec:impuls_photonen}
Da Photonen keine Masse haben, können sie nicht mit der Formel $p = m \cdot v$ beschrieben werden. Stattdessen wird dem Photon nach der Gleichung $W = m \cdot c^2$ ein Impuls $p_\gamma$ zugeordnet.
\begin{gather}
    p_\gamma = m \cdot v = m \cdot c = \frac{h \cdot f}{c^2} \cdot c = \frac{h \cdot f}{c} = \frac{h}{\lambda}
\end{gather}
\begin{center}
    \begin{tabular}{rl}
        $p_\gamma$ & Impuls des Photons \\
        $h$ & Planck-Konstante \\
        $\lambda$ & Wellenlänge
    \end{tabular}
\end{center}

\subsubsection{Energie der Elektronen beim Photoelektrischen Effekt ($W_\mathrm{kin,\,max}$, $W_\mathrm{lös}$)}
\label{sec:energie_elektronen_photoelektrischer_effekt}
Im Vakuum einer \hyperref[sec:fotozelle]{Fotozelle} wird die Energie $W_\mathrm{kin,\,max}$ der schnellsten Elektronen untersucht, die durch den Photoelektrischen Effekt aus der Elektrode herausgelöst wurden. Ihre Energie $W_\mathrm{kin,\,max}$ ist linear proportional zur Wellenfrequenz $f$ des Lichts, das auf die Elektrode einfällt. Zudem ist die Steigung aller Geraden in einem $W$-$f$-Diagramm gleich der \hyperref[sec:planck_konstante]{Planck-Konstante} $h$ und vom Material unabhängig.
Da herausgelöste Elektronen nur eine bestimmte Energie besitzen, können sie maximal eine Spannung $U_\mathrm{max} = \frac{W_\mathrm{kin,\,max}}{q}$ durchqueren.
\begin{gather}
    W_\mathrm{kin,\,max} = h \cdot f - W_\mathrm{lös} = \frac{1}{2} \cdot m_\mathrm{e} \cdot v_\mathrm{max}^2 = U_\mathrm{max} \cdot q
\end{gather}
\begin{center}
    \begin{tabular}{rl}
        $W_\mathrm{kin,\,max}$ & Maximale kinetische Energie der Elektronen \\
        $W_\mathrm{lös}$ & Ablöseenergie (materialabhängig) \\
        $h$ & Planck-Konstante \\
        $f$ & Frequenz des einfallenden Lichts \\
        $m_\mathrm{e}$ & Elektronenmasse \\
        $v_\mathrm{max}$ & Maximale Geschwindigkeit der Elektronen \\
        $U_\mathrm{max}$ & Maximale durchquerbare Spannung \\
        $q$ & Elementarladung
    \end{tabular}
\end{center}
Die negativen Achsenabschnitte $-W_\mathrm{lös}$ der Geraden hängen vom Metall ab und stehen für die Ablöseenergie. Sie verkleinert die Energie, die von den Photonen ($W_\gamma = h \, f$) an die Elektronen abgegeben wird.

\subsubsection{de Broglie-Wellenlänge ($\lambda_\mathrm{dB}$)}
\label{sec:de_broglie_wellenlaenge}
Mit der de Broglie-Wellenlänge $\lambda_\mathrm{dB}$ kann auch massebehafteten Teilchen wie Elektronen eine Wellenlänge zugeordnet werden.
\begin{gather}
    \lambda_\mathrm{dB} = \frac{h}{p} = \frac{h}{m \cdot v}
\end{gather}
\begin{center}
    \begin{tabular}{rl}
        $\lambda_\mathrm{dB}$ & de Broglie-Wellenlänge \\
        $h$ & Planck-Konstante \\
        $p$ & Impuls des Teilchens \\
        $m$ & Masse des Teilchens \\
        $v$ & Geschwindigkeit des Teilchens
    \end{tabular}
\end{center}

\subsection{Wichtige Konzepte und Vertiefung}

\subsubsection{Photoelektrischer Effekt}
\label{sec:photoelektrischer_effekt}
Der Photoelektrische Effekt ist ein experimenteller Effekt, bei dem Elektronen aus einer Metallplatte abgestoßen werden, wenn sie mit Lichtstrahlen ausgesetzt sind. Die Elektronen werden durch die Lichtstrahlung beschleunigt und können somit einen elektrischen Strom erzeugen.

Ein zentrales Experiment ist der \hyperref[sec:hallwachs_versuch]{Hallwachs-Versuch}.

\subsubsection{Leuchtdioden (LEDs)}
\label{sec:leds}
Bei Leuchtdioden wird die Umkehrung des photoelektrischen Effekts genutzt, um Licht zu erzeugen. Eine LED besteht aus einem Halbleiter mit einem p-n-Übergang. Wird die Diode in Durchlassrichtung betrieben, so werden Elektronen aus dem n-dotierten Bereich und Löcher aus dem p-dotierten Bereich in den Übergangsbereich gedrückt.
Dort rekombinieren die Elektronen mit den Löchern. Dabei geht ein Elektron von einem energetisch höheren Zustand in einen niedrigeren über. Die dabei frei werdende Energie wird in Form eines Photons abgegeben. Die Energie des ausgesendeten Lichts entspricht näherungsweise der Bandlückenenergie des verwendeten Halbleitermaterials. Wird an die LED eine Spannung in Sperrrichtung angelegt, so vergrößert sich die Raumladungszone am p-n-Übergang. Die freien Ladungsträger werden vom Übergangsbereich weggezogen, sodass kein nennenswerter Stromfluss stattfinden kann. Da keine Rekombination von Elektronen und Löchern erfolgt, wird kein Licht emittiert. Solange die angelegte Sperrspannung unterhalb der materialspezifischen Durchbruchspannung bleibt, verhält sich die LED wie ein Isolator.
Die Farbe des emittierten Lichts wird somit durch die Größe der Bandlücke bestimmt und ist materialabhängig. Eine größere Bandlücke führt zu höherfrequentem (kurzwelligem) Licht, eine kleinere Bandlücke zu niederfrequentem (langwelligem) Licht.
Da die Lichtemission direkt auf elektronischen Übergängen beruht, weisen LEDs einen hohen Wirkungsgrad, eine geringe Wärmeentwicklung und eine lange Lebensdauer auf.

\subsubsection{Compton-Effekt}
\label{sec:compton_effekt}
Der Compton-Effekt beschreibt die inelastische Streuung von Röntgen- oder $\gamma$-Strahlung an (nahezu) freien Elektronen. Dabei überträgt ein Photon einen Teil seiner Energie und seines Impulses auf das Elektron, sodass das gestreute Photon eine größere Wellenlänge (geringere Energie) besitzt als zuvor. Die Wellenlängenänderung hängt ausschließlich vom Streuwinkel ab und lässt sich durch die folgende Formel beschreiben:
\begin{gather}
    \Delta \lambda = \frac{h}{m_\mathrm{e} c}\,(1-\cos\theta)
\end{gather}
\begin{center}
    \begin{tabular}{rl}
        $\Delta \lambda$ & Wellenlängenänderung \\
        $h$ & Planck-Konstante \\
        $m_\mathrm{e}$ & Elektronenmasse \\
        $c$ & Lichtgeschwindigkeit \\
        $\theta$ & Streuwinkel
    \end{tabular}
\end{center}

\subsubsection{Röntgenbeugung und Bragg-Reflexion}
\label{sec:roentgenbeugung}
Die Röntgenbeugung wird verwendet, um die atomare Struktur von Kristallen zu untersuchen. Da die Wellenlänge $\lambda$ von Röntgenstrahlung in der Größenordnung der atomaren Abstände im Kristallgitter liegt, tritt Beugung an den Gitteratomen auf. Dieses Phänomen wird vereinfacht als „Reflexion“ an den parallelen Netzebenen des Kristalls beschrieben (Bragg-Reflexion).
\begin{figure}[H]
    \centering
    \includegraphics[width=0.6\linewidth]{figures/bragg.png}
    \caption{Schematische Darstellung der Bragg-Reflexion.}
    \label{fig:bragg}
\end{figure}
\begin{center}
    \begin{tabular}{rl}
        $d$ & Netzebenenabstand (Abstand zweier benachbarter Atomebenen) \\
        $\theta$ & Winkel zwischen einfallendem Strahl und der Netzebene (nicht dem Lot!) \\
        $\delta$ & Hier die Hälfte des \hyperref[sec:gangunterschied]{Gesamt-Gangunterschieds}
    \end{tabular}
\end{center}
Damit ein Intensitätsmaximum auf dem Schirm beobachtet werden kann, muss der Gangunterschied $\delta$ der an zwei benachbarten Netzebenen reflektierten Wellen ein ganzzahliges Vielfaches der Wellenlänge betragen:
\begin{gather}
    \delta_\mathrm{ges} = k \cdot \lambda \nonumber \\
    \frac{\delta_\mathrm{ges}}{2} = \sin(\theta) \cdot d \nonumber \\
    k \cdot \lambda = 2d \, \sin(\theta)
\end{gather}

\subsubsection{Beugung und Reflexion von Elektronen}
\label{sec:elektronenbeugung_reflexion}
Da Elektronen gemäß der de Broglie-Hypothese Welleneigenschaften besitzen ($\lambda_{\mathrm{dB}} = \frac{h}{p}$), können sie ebenfalls an Kristallgittern gebeugt werden. Dies ist ein zentraler Nachweis für den Welle-Teilchen-Dualismus.
\begin{figure}[H]
    \centering
    \includegraphics[width=0.75\linewidth]{figures/elektronenbeugung2.png}
    \caption{Schematische Darstellung der Beugung von Elektronen an einem Kristallgitter.}
    \label{fig:elektronenbeugung}
\end{figure}
An jedem Einzelkristall des polykristallenen Graphitgitters wird das Elektron nach der \hyperref[sec:roentgenbeugung]{Bragg-Reflexion} reflektiert. Da der einfallende Strahl und der reflektierte Strahl jeweils den Winkel $\theta$ zu den Netzebenen einschließen, beträgt der gesamte Ablenkwinkel gegenüber der ursprünglichen Bahn $2\theta$. (Vgl. Abbildung \ref{fig:bragg} zur Bragg-Reflexion).

In der \hyperref[sec:elektronenbeugungsroehre]{Elektronenbeugungsröhre} gilt, sofern $l$ bei jedem Winkel der Durchmesser der Röhre ist:
\begin{gather*}
    \frac{r}{l} = \sin(2\theta)
\end{gather*}
Für kleine Winkel gilt nach der Kleinwinkelnäherung:
\begin{gather*}
    \sin(2\theta) \approx 2\sin(\theta)
\end{gather*}
Somit folgt:
\begin{gather*}
    \frac{r}{l} = 2\sin(\theta) \rightarrow \sin(\theta) = \frac{r}{2l}
\end{gather*}
Durch die Ähnlichkeit des Versuchs kann die Bragg'sche Interferenzbedingung der Röntgenbeugung für die Maxima bei der Interferenz auf die Elektronenbeugung übertragen werden:
\begin{gather*}
    k \, \lambda = 2d \, \sin(\theta) \rightarrow \sin(\theta) = \frac{k \, \lambda}{2d}
\end{gather*}
Wir können diese Ähnlichkeit nun weiter verwenden, indem wir die beiden Sinusbeziehungen gleichsetzen:
\begin{gather}
    \frac{r}{2l} = \frac{k \, \lambda}{2d} \\
    \lambda = \frac{d \cdot r}{k \cdot l}
\end{gather}
\begin{center}
    \begin{tabular}{rl}
        $r$ & Ringdurchmesser auf dem Schirm \\
        $l$ & Röhrendurchmesser \\
        $k$ & Ordnung des Maximums \\
        $d$ & Netzebenenabstand von Graphit
    \end{tabular}
\end{center}
Bei Graphit gibt es aufgrund der Polykristallstruktur zwei Netzebenenabstände bzw. Gitterkonstanten. Bei anderen Materialien kann es auch drei oder sogar mehr Netzebenenabstände geben.

 Siehe auch: \url{https://www.leifiphysik.de/quantenphysik/quantenobjekt-elektron/versuche/elektronenbeugungsroehre-simulation-mintapps}

\subsubsection{Doppelspaltexperiment}
\label{sec:doppelspaltexperiment}
Das Doppelspaltexperiment zeigt, dass Licht und Materie sowohl Wellen- als auch Teilcheneigenschaften besitzen: Schickt man einzelne Teilchen (z.~B. Elektronen oder Photonen) durch zwei enge, parallele Spalte, entsteht auf dem Schirm dahinter ein Interferenzmuster – ein typisches Wellenphänomen –, obwohl jedes Teilchen einzeln detektiert wird. Es demonstriert damit fundamentale Quantenphänomene wie Überlagerung und den Einfluss der Beobachtung auf das Ergebnis.
\begin{figure}[H]
    \centering
    \includegraphics[width=0.9\linewidth]{figures/doppelspalt.png}
    \caption{Schematische Darstellung eines Doppelspalts.}
    \label{fig:doppelspalt}
\end{figure}
\begin{center}
    \begin{tabular}{rl}
        $a$ & Abstand zwischen Spaltebene und Schirm \\
        $g$ & Spaltabstand ($g \ll a$) \\
        $d_k$ & Abstand vom 0. zum $k$-ten Maximum \\
        $\delta$ & \hyperref[sec:gangunterschied]{Gangunterschied} \\
        $\alpha$ & Winkel zwischen Spalten und Auftreffpunkt auf dem Schirm
    \end{tabular}
\end{center}
Wie bei anderen Versuchen gilt auch hier die Kleinwinkelnäherung bis $\alpha \approx 5^\circ$ und $\sin(\alpha) \approx \tan(\alpha)$. Zudem wird angenommen, dass die Spaltbreite vernachlässigt werden kann:
\begin{gather*}
    \sin(\alpha) = \frac{\delta}{g} \\
    \tan(\alpha) = \frac{d_k}{a} \\
    \frac{\delta}{g} = \frac{d_k}{a} \\
    d_k = \frac{\delta \cdot a}{g}
\end{gather*}
Für die Minima gilt $\delta = \frac{2k - 1}{2} \cdot \lambda$ mit $k \in \mathbb{N}$:
\begin{gather}
    d_k = \frac{(2k - 1) \cdot \lambda \cdot a}{2g}
\end{gather}
Für die Maxima neben dem Maximum der 0. Ordnung in der Mitte des Interferenzmusters gilt $\delta = k \cdot \lambda$:
\begin{gather}
    d_k = \frac{k \cdot \lambda \cdot a}{g}
\end{gather}
Jeder Abstand $d_k$ beschreibt hierbei den Abstand zweier Maxima oder Minima, deren Ordnung sich um $k$ unterscheidet. Es gibt keine Minima der 0. Ordnung und $d_k$ wird im Folgenden allgemein für den Abstand zweier Maxima verwendet.
\begin{figure}[H]
    \centering
    \includegraphics[width=0.9\linewidth]{figures/interferenzmuster.png}
    \caption{Darstellung des am Doppelspalt entstehenden Interferenzmusters.}
    \label{fig:interferenzmuster}
\end{figure}

\subsubsection{Interferenz am Gitter}
\label{sec:interferenz_am_gitter}
Interferenz am Gitter bezeichnet das Überlagerungsphänomen von Lichtwellen, die an den vielen, regelmäßig angeordneten Spalten eines Gitters gebeugt werden. Die einzelnen gebeugten Wellen laufen anschließend zusammen und verstärken oder schwächen sich je nach ihrem Gangunterschied. Dadurch entstehen scharfe, gut trennbare Maxima in bestimmten Richtungen.
\begin{figure}[H]
    \centering
    \includegraphics[width=0.7\linewidth]{figures/vielfachspalt.png}
    \caption{Schematische Darstellung eines Vielfachspalts.}
    \label{fig:vielfachspalt}
\end{figure}
\begin{center}
    \begin{tabular}{rl}
        $a$, $d_k$, $\delta$, $\alpha$ & Siehe Doppelspalt \\
        $g$ & Gitterkonstante (Abstand der Spaltmitten)
    \end{tabular}
\end{center}
Beim Mehrfachspalt bzw. beim Gitter überlagern sich die Lichtwellen dann maximal konstruktiv, wenn die Wellen benachbarter Spalte einen Gangunterschied von $\delta = k \cdot \lambda$ besitzen. Ist das nicht der Fall, so kommt es zwischen bestimmten Spalten zu einer destruktiven Interferenz und damit nicht zu einem Hauptmaximum, sondern zu einem Nebenmaximum, wenn dennoch einige Wellen konstruktiv interferieren.
\begin{figure}[H]
    \centering
    \includegraphics[width=0.9\linewidth]{figures/interferenzmuster_mehrfachspalt.png}
    \caption{Darstellung des am Gitter entstehenden Interferenzmusters.}
    \label{fig:gitterinterferenz}
\end{figure}
Auch hier wird zunächst angenommen, dass die Spaltbreite vernachlässigt werden kann.

\begin{gather*}
    \sin(\alpha) = \frac{\delta}{g} \\
    \tan(\alpha) = \frac{d_k}{a} \\
    \frac{\delta}{g} = \frac{d_k}{a}
\end{gather*}
Für die Maxima gilt $\delta = k \cdot \lambda$:
\begin{gather}
    d_k = \frac{k \cdot \lambda \cdot a}{g}
\end{gather}
Die Herleitung der exakten Position der Maxima erfolgt über trigonometrische Beziehungen:
\begin{gather*}
    \sin(\alpha) = \frac{\delta}{g} = \frac{k \cdot \lambda}{g} \\
    \tan(\alpha) = \frac{d_k}{a} \rightarrow \alpha = \tan^{-1}\!\left(\frac{d_k}{a}\right) \\
    \lambda = \frac{\sin(\alpha) \cdot g}{k} = \frac{\sin(\tan^{-1}\!\left(\frac{d_k}{a}\right)) \cdot g}{k} \\
    d_k = a \cdot \tan\!\left(\sin^{-1}\!\left(\frac{k \lambda}{g}\right)\right)
\end{gather*}
Alternativ lassen sich die Beziehungen über den Satz des Pythagoras angeben:
\begin{gather*}
    \sin(\alpha) = \frac{k \cdot \lambda}{g} = \frac{d_k}{\sqrt{a^2 + (d_k)^2}} \\
    \lambda = \frac{g}{k} \, \sin(\alpha) \\
    d_k = a \, \tan(\alpha)
\end{gather*}

\subsubsection{Interferenz an dünnen Schichten}
\label{sec:interferenz_an_duennen_schichten}
Interferenz an dünnen Schichten entsteht, wenn Licht auf eine transparente Schicht trifft, deren Dicke in der Größenordnung der Lichtwellenlänge liegt. Ein Teil des Lichts wird an der Oberseite der Schicht reflektiert, während ein anderer Teil in die Schicht eindringt und an der Unterseite reflektiert wird. Diese beiden reflektierten Teilstrahlen überlagern sich und interferieren miteinander.
Der Gangunterschied zwischen den beiden Teilstrahlen hängt von der Schichtdicke, dem Brechungsindex des Materials und dem Einfallswinkel ab. Je nach Gangunterschied kommt es zu konstruktiver oder destruktiver Interferenz, wodurch bestimmte Wellenlängen verstärkt und andere ausgelöscht werden.
Wichtig ist dabei, dass bei der Reflexion an einem optisch dichteren Medium (höherer Brechungsindex) ein Phasensprung von einer halben Wellenlänge auftritt, während die Reflexion an einem optisch dünneren Medium ohne Phasensprung erfolgt. Dieser Phasensprung muss bei der Berechnung des Gangunterschieds berücksichtigt werden.

Typische Beispiele für Interferenz an dünnen Schichten sind:
\begin{itemize}
    \item \emph{Seifenblasen:} Die schillernden Farben entstehen durch Interferenz an der dünnen Seifenhaut. Da die Schichtdicke über die Oberfläche variiert, werden an verschiedenen Stellen unterschiedliche Farben verstärkt.
    \item \emph{Ölflecken auf Wasser:} Der dünne Ölfilm erzeugt ebenfalls farbige Interferenzmuster durch die Überlagerung der an Ober- und Unterseite reflektierten Lichtwellen.
    \item \emph{Schmetterlingsflügel:} Die schillernden Farben vieler Schmetterlingsarten entstehen durch mikroskopische Strukturen auf den Flügelschuppen, die wie dünne Schichten wirken und das Licht durch Interferenz gezielt verstärken.
\end{itemize}

\subsubsection{Beugung und Interferenz am Einzelspalt}
\label{sec:einzelspalt}
Anders als zuvor wird bei Beugung und Interferenz am Einzelspalt die Breite $b$ des Spaltes selbst nicht mehr vernachlässigt. Das Maximum 0. Ordnung befindet sich weiterhin mittig auf dem Schirm, also vor der Mitte des Spalts.
\begin{figure}[H]
    \centering
    \includegraphics[width=0.5\linewidth]{figures/einzelspalt.png}
    \caption{Schematische Darstellung eines Einzelspalts.}
    \label{fig:einzelspalt}
\end{figure}
\begin{center}
    \begin{tabular}{rl}
        $a$, $g$, $d_k$, $\delta$, $\alpha$ & Siehe Doppelspalt \\
        $b$ & Spaltbreite
    \end{tabular}
\end{center}
Beim Einzelspalt gelten andere Formeln als beim Doppelspalt oder beim Gitter. Hier gilt für die Minima, dass $\delta = k \cdot \lambda$ und für die Maxima, dass $\delta = \frac{2k + 1}{2} \cdot \lambda$. Eine genaue Erklärung ist hier zu finden: \url{https://www.leifiphysik.de/optik/beugung-und-interferenz/grundwissen/einzelspalt} im interaktiven Video.

\begin{gather*}
    \sin(\alpha) = \frac{\delta}{b} \\
    \tan(\alpha) = \frac{d_k}{a} \\
    \frac{\delta}{b} = \frac{d_k}{a} \\
    d_k = \frac{\delta \cdot a}{b}
\end{gather*}
Für die Minima gilt $\delta = k \cdot \lambda$:
\begin{gather}
    d_k = \frac{k \cdot \lambda \cdot a}{b}
\end{gather}
Für die Maxima gilt $\delta = \frac{2k + 1}{2} \cdot \lambda$:
\begin{gather}
    d_k = \frac{(2k + 1) \cdot \lambda \cdot a}{2b}
\end{gather}
Da $\sin(\alpha) \le 1$ sein muss, ist die Anzahl der beobachtbaren Minima und Maxima begrenzt. Aus $\sin(\alpha) = \frac{k \cdot \lambda}{b}$ für die Minima folgt $k \cdot \lambda \le b$, also $k \le \frac{b}{\lambda}$. Entsprechendes gilt für die Maxima.

\subsubsection{Wellenfunktion}
Die Wellenfunktion $\Psi(\vec{x}, t)$ ist eine mathematische Beschreibung des quantenmechanischen Zustands eines Teilchens oder Systems. Sie ist eine komplexwertige Funktion, deren Betragsquadrat $|\Psi(\vec{x}, t)|^2$ die Wahrscheinlichkeitsdichte angibt, das Teilchen zur Zeit $t$ am Ort $\vec{x}$ zu finden. Die Wellenfunktion selbst ist nicht direkt messbar, aber sie enthält alle physikalisch relevanten Informationen über das System und entwickelt sich gemäß der Schrödinger-Gleichung.

\subsection{Apparaturen und Sonstiges}

\subsubsection{Kometen}
Ein Komet ist ein kleiner Himmelskörper, der vorwiegend aus Eis, Staub und Gestein besteht – oft als „schmutziger Schneeball“ bezeichnet. Nähert er sich der Sonne, sublimiert das Eis durch die Erwärmung; diese freigesetzten Gase und mitgerissenen Staubteilchen bilden eine Atmosphäre um den Kern, die Koma.

Es bilden sich dabei zwei Arten von Schweifen aus, die beide grundsätzlich von der Sonne weg zeigen:
\begin{itemize}
    \item \emph{Gasschweif (Plasmaschweif):} Besteht aus ionisierten Gasmolekülen. Er ist schmal, gerade und leuchtet meist bläulich. Seine Ausrichtung wird durch den Sonnenwind (Strom geladener Teilchen) bestimmt, der die Ionen direkt von der Sonne wegdrückt.
    \item \emph{Staubschweif:} Besteht aus schwereren Staubpartikeln. Er ist breiter, leicht gekrümmt und leuchtet weißlich durch reflektiertes Sonnenlicht.
\end{itemize}
Die Ausrichtung des Staubschweifs liefert einen direkten Hinweis auf den Lichtdruck (Strahlungsdruck). Das Sonnenlicht übt eine Kraft auf die Staubteilchen aus, was belegt, dass elektromagnetische Wellen (Photonen) einen Impuls $p = \frac{h}{\lambda} = \frac{E}{c}$ besitzen und diesen auf Materie übertragen können. Ohne diesen Impulsübertrag würde der Schweif allein der Gravitation folgen.

\subsubsection{Gravitationslinsen}
Gravitationslinsen sind Erscheinungen, bei denen die Schwerkraft einer massereichen Struktur (etwa Galaxien oder Galaxienhaufen) den Weg des Lichts von weiter entfernten Objekten messbar krümmt. Dadurch wirkt die Masse wie eine Linse: Das Licht wird abgelenkt, verstärkt, verzerrt oder mehrfach abgebildet.
Im Gegensatz zu optischen Linsen entsteht die Ablenkung hier nicht durch Brechung in einem Medium, sondern ausschließlich durch die Krümmung der Raumzeit gemäß der Allgemeinen Relativitätstheorie.

\subsubsection{Fotozelle}
\label{sec:fotozelle}
Eine Fotozelle reagiert auf einfallendes Licht, indem die Photonen an der Metalloberfläche der ersten Elektrode Elektronen aus dem Material herauslösen (\hyperref[sec:photoelektrischer_effekt]{Photoelektrischer Effekt}). Die freigesetzten Elektronen werden durch die angelegte elektrische Spannung abgebremst und können die zweite Elektrode erreichen, sofern ihre kinetische Energie ausreichend groß ist.
Bei Alkalimetallen wie Caesium genügt sichtbares Licht, um die Elektronen freizusetzen.
\begin{figure}[H]
    \centering
    \includegraphics[width=0.8\linewidth]{figures/fotozelle.pdf}
    \caption{Schematische Darstellung einer Fotozelle.}
    \label{fig:fotozelle}
\end{figure}
Eine Simulation einer Fotozelle ist hier zu finden: \url{https://mintapps.org/html/mint-photoeffect.html}.

\subsubsection{Michelson Interferometer}
Das Michelson-Interferometer ist ein optisches Gerät, das zur Erzeugung von Interferenzmustern durch Teilung eines Lichtstrahls in zwei Arme verwendet wird, die unterschiedliche Weglängen haben können. Es besteht aus einem Strahlteiler, der einen einfallenden Strahl in einen transmittierten und einen reflektierten Strahl aufteilt. Diese beiden Teilstrahlen werden von zwei Spiegeln ($S_1$ und $S_2$) reflektiert und kehren zum Strahlteiler zurück, wo sie in Summe ohne Phasenverschiebung rekombiniert werden und am Detektor oder Schirm Interferenzmuster erzeugen.

Das Michelson-Interferometer wird häufig zur Messung kleiner Abstandsänderungen (wie im LIGO-Experiment) oder zur Spektroskopie eingesetzt, da es empfindlich auf Änderungen der optischen Weglänge in einem der Arme reagiert.
\begin{figure}[H]
    \centering
    \includegraphics[width=0.7\linewidth]{figures/michelson_interferometer.pdf}
    \caption{Schematische Darstellung eines Michelson-Interferometers.}
    \label{fig:michelson_interferometer}
\end{figure}
Die optische Weglängendifferenz $\Delta L$ zwischen den beiden Armen ist entscheidend für die Entstehung von Maxima und Minima des Interferenzmusters und die Messung am Detektor:
\begin{gather}
    \Delta L = 2(l_2 - l_1)
\end{gather}
wobei $l_1$ und $l_2$ die Abstände zu den Spiegeln $S_1$ und $S_2$ vom Strahlteiler sind.

\subsubsection{Mach-Zehnder-Interferometer}
\label{sec:mach_zehnder_interferometer}
Ein Mach-Zehnder-Interferometer teilt einen Lichtstrahl mit einem Strahlteiler in zwei räumlich getrennte Wege auf, führt sie durch zwei unabhängige Arme, in denen beide Lichtstrahlen reflektiert werden. Beide treffen auf einen zweiten Strahlteiler, so dass jeweils eine Hälfte beider Strahlen zu einem der beiden Detektoren oder Schirme führt und die beiden miteinander interferieren.

Entscheidend ist, dass Licht bei Reflexion an einer Grenzfläche zu einem optisch dichteren Medium hin einen Phasensprung von $\pi$ und beim Übergang zwischen Medien durch Beugung ein Phasensprung von $\tfrac{\pi}{2}$ erfährt. Bei Transmission findet in keinem Fall ein Phasensprung statt.

An Detektor 1 (oben) wird destruktive Interferent beobachtet, denn der vom Laser kommende Strahl wird zur Hälfte am ersten Strahlteiler reflektiert, und erfährt dabei einen Phasensprung von $\pi$. Anschließend wird er am oberen Spiegel reflektiert, wobei er nochmals einen Phasensprung von $\pi$ erfährt. Danach wird er am zweiten Strahlteiler reflektiert, hier findet kein Phasensprung statt, weil durch den Übergang zwischen den Medien beim Ein- und Austritt in und aus dem halbdurchlässigen Spiegel wegen der Beugung zusammen ein Phasensprung von $2 \cdot (-\tfrac{\pi}{2}) = -\pi$ entsteht und die eigentliche Phasenverschiebung um $\pi$ durch die Reflexion aufhebt, so dass der obere Lichtstrahl insgesamt um $2\pi$ phasenverschoben wird.
Die andere Hälfte des vom Laser kommenden Strahls wird am ersten Strahlteiler transmittiert und erfährt keine Phasenverschiebung. Anschließend wird der Strahl am Spiegel unten rechts reflektiert, wobei er einen Phasensprung von $\pi$ erfährt. Danach wird er am zweiten Strahlteiler transmittiert, es findet kein Phasensprung statt, so dass der untere Lichtstrahl insgesamt um $\pi$ phasenverschoben wird.
Zwischen den beiden Strahlen existiert also eine Phasendifferenz von $2\pi - \pi = \pi$. Die Strahlen löschen sich gerade gegenseitig aus und die Detektion an Detektor 1 bleibt aus.

An Detektor 2 (rechts) wird konstruktive Interferenz beobachtet. Der vom Laser kommende Strahl wird zur Hälfte am ersten Strahlteiler reflektiert, am oberen Spiegel reflektiert und erhält wie eben einen Phasensprung von $2\pi$.
Die andere Hälfte des vom Laser kommenden Strahls wird wie zuvor am ersten Strahlteiler transmittiert und am unteren Spiegel reflektiert, wobei das Licht einen Phasensprung von $\pi$ erfährt. Danach wird es am zweiten Strahlteiler reflektiert, es findet nochmals ein Phasensprung von $\pi$ statt, so dass die Welle insgesamt also um $2\pi$ phasenverschoben wird.
Zwischen den beiden Strahlen besteht keine Phasendifferenz und die beiden Strahlen interferieren maximal konstruktiv und das Photon wird an Detektor 2 detektiert.

Das Mach-Zehnder-Interferometer ist als Transmissionsinterferometer besonders nützlich für die Untersuchung von Phasenverschiebungen in transparenten Proben, da beide Teilstrahlen die Probe durchlaufen können, ohne reflektiert zu werden.
\begin{figure}[H]
    \centering
    \includegraphics[width=0.9\linewidth]{figures/mach_zehnder_interferometer.pdf}
    \caption{Schematische Darstellung eines Mach-Zehnder-Interferometers.}
    \label{fig:mach_zehnder_interferometer}
\end{figure}

\subsubsection{Interferenzmuster der Interferometer}
\label{sec:interferenzmuster_interferometer}
Die kreisförmigen Interferenzmuster, die bei Interferometern wie dem Michelson-Interferometer beobachtet werden können, entstehen, weil der optische Gangunterschied zwischen den beiden Teilstrahlen vom Beobachtungswinkel abhängt. Für einen festen Gangunterschied bilden alle Punkte, die unter demselben Winkel zur optischen Achse liegen, einen Kreis. Das Zentrum des Interferenzmusters (direkt auf der optischen Achse) zeigt je nach dem dortigen Gangunterschied konstruktive oder destruktive Interferenz.

\subsubsection{Knallertest}
\label{sec:knallertest}
Der Knallertest ist ein Gedankenexperiment der Quantenphysik. Es zeigt, dass es unter bestimmten Bedingungen möglich ist, Informationen über ein Objekt zu gewinnen, ohne dass es zu einer direkten physikalischen Wechselwirkung kommt. Dieses Prinzip wird als wechselwirkungsfreie Quantenmessung bezeichnet.

Gedanklich betrachtet man eine Bombe, die so empfindlich ist, dass bereits die Absorption eines einzelnen Photons eine Explosion auslöst. Neben funktionsfähigen Bomben gibt es jedoch in gleicher Anzahl auch Imitate, die für Photonen vollständig transparent sind. Ziel des Experiments ist es, eine funktionsfähige Bombe eindeutig zu identifizieren, ohne sie zur Explosion zu bringen.

Dazu wird ein \hyperref[sec:mach_zehnder_interferometer]{Mach-Zehnder-Interferometer} verwendet. In einen der beiden Wege (beispielsweise den unteren) wird die zu überprüfende Bombe eingebracht. Ohne Bombe sieht man wie erwartet destruktive Interferenz am oberen Detektor und konstruktive Interferenz am rechten Detektor.

Ein Imitat verhält sich genauso wie das leere Interferometer und nur rechts registriert man ein Photon. Befindet sich hingegen eine funktionsfähige Bombe im Interferometer, so wirkt sie als Messapparat. Nimmt das Photon den Weg mit der Bombe, wird es absorbiert und die Bombe explodiert (Wahrscheinlichkeit 50\%). Nimmt das Photon den anderen Weg, findet keine Explosion statt. Allerdings ist in diesem Fall die Interferenz aufgehoben, da prinzipiell eine Welcher-Weg-Information vorliegt. An jedem der Detektoren registriert man 50\% der Photonen.

Man misst also am rechten Detektor die Photonen, die bei Imitaten registriert werden. Gleichzeitig misst man hier auch die Hälfte der Photonen, die bei einer funktionierenden Bombe den oberen Weg genommen haben. Insgesamt sind das $\tfrac{1}{2} + \tfrac{1}{8} = \tfrac{5}{8}$ aller Photonen, die rechts gemessen werden.
Am oberen Detektor wird die andere Hälfte der Photonen, die bei einer funktionierenden Bombe den oberen Weg genommen haben, gemessen. Das ist $\tfrac{1}{8}$ aller Photonen.
Das letzte Viertel der Photonen bringt eine Bombe zum explodieren.

Das Gedankenexperiment verdeutlicht, dass bereits die bloße Möglichkeit einer Messung – also die potenzielle Wechselwirkung – den Zustand eines Quantensystems beeinflusst. Eine tatsächliche physische Wechselwirkung ist dafür nicht notwendig. In weiterentwickelten Experimenten, die diesen Effekt ausnutzen, lässt sich die Erfolgswahrscheinlichkeit eines solchen Nachweises ohne Explosion sogar nahezu auf 100\% steigern.

\begin{figure}[H]
    \centering
    \includegraphics[width=0.9\linewidth]{figures/knallertest.pdf}
    \caption{Wahrscheinlichkeitsfluss beim Knallertest. Die Breite der Pfade entspricht der Wahrscheinlichkeit jedes Ausgangs.}
    \label{fig:knallertest}
\end{figure}

\subsubsection{Delayed Choice an Doppelspalt und Interferometer}
\label{sec:delayed_choice}
Am klassischen Doppelspalt hängt das Auftreten einer Interferenz davon ab, ob nach dem Durchtreten des Photons der Spalte eine Detektion erfolgt. Ohne eine Detektion wird das Photon in eine Superposition zweier Zustände überführt und eine Interferenz entsteht. Mit einer Detektion wird das Photon in einen definierten Zustand überführt und eine Interferenz verhindert. Auf dem Schirm erkennt man zwei Streifen.

Es stellt sich die Frage, wann sich das Photon entscheidet, ob es sich als Welle oder als Teilchen verhält: Entscheidet es sich beim Passieren der Schlitze oder entscheidet es sich erst, wenn es gemessen wird?

John Wheeler schlug vor, die Entscheidung, ob man misst oder nicht, erst zu treffen, nachdem das Photon die Schlitze bereits passiert hat, aber bevor es auf dem Schirm auftrifft.
Das verblüffende Ergebnis: Selbst wenn die Entscheidung zur Messung erst fällt, nachdem das Photon die Schlitze theoretisch schon passiert hat, verhält sich das Photon genau so, wie es die Messung erfordert: Entscheiden wir uns spät für eine Weg-Messung kommt es zum Teilchenverhalten (keine Interferenz). Entscheiden wir uns spät gegen eine Weg-Messung kommt es zum Wellenverhalten (Interferenz). Es wirkt so, als ob das Photon in der Gegenwart (bei der Entscheidung zu messen) die Vergangenheit (das Verhalten an den Schlitzen) beeinflussen könnte. Dieses Paradoxon wird auch als Delayed Choice Paradoxon bezeichnet.

Die zentrale Einsicht des Delayed-Choice-Experiments besteht darin, dass die Frage „Wann entscheidet sich das Photon?“ falsch gestellt ist. Das Photon besitzt weder vor noch während des Experiments festgelegte klassische Eigenschaften wie „Welle“ oder „Teilchen“. Diese Begriffe beschreiben vielmehr unterschiedliche experimentelle Kontexte.

Nach quantenmechanischer Beschreibung wird das Photon nach dem Durchgang durch die Spalte durch einen einzigen Quantenzustand beschrieben, der alle physikalisch möglichen Alternativen enthält. Solange keine Messung erfolgt, bleibt dieser Zustand kohärent. Erst die konkrete Messanordnung – also die Frage, welche Observablen zugänglich gemacht werden – legt fest, welche Aspekte dieses Zustands experimentell sichtbar werden.

Die verzögerte Entscheidung zur Wegmessung ändert daher nicht rückwirkend das frühere Verhalten des Photons. Vielmehr war das Photon zu keinem Zeitpunkt eindeutig „auf einem Weg“ oder „auf beiden Wegen“. Die spätere Wahl der Messung bestimmt lediglich, welche Information aus dem bereits existierenden Quantenzustand extrahiert wird.

Das scheinbare Paradoxon entsteht nur, wenn man dem Photon rückblickend klassische Eigenschaften zuschreibt, die es quantenmechanisch nie besessen hat.

\subsection{Versuche}

\subsubsection{Hallwachs-Versuch}
\label{sec:hallwachs_versuch}
Der Hallwachs-Versuch, durchgeführt von Wilhelm Hallwachs im Jahr 1887, demonstrierte erstmals den äußeren photoelektrischen Effekt. Er zeigte, dass eine negativ geladene Zinkplatte ihre Ladung verliert, wenn sie mit ultraviolettem Licht bestrahlt wird. Eine positiv geladene Platte hingegen behält ihre Ladung. Dies deutete darauf hin, dass Licht Elektronen aus der Metalloberfläche herauslösen kann, aber nur, wenn die Energie der Photonen (und damit die Frequenz des Lichts) einen bestimmten Schwellenwert überschreitet. Der Versuch war ein wichtiger experimenteller Beitrag zur Entwicklung der Quantenphysik und zur Erklärung des photoelektrischen Effekts durch Albert Einstein.

Wird eine Glasplatte zwischen Lichtquelle und Zinkplatte platziert, die auch ohne Beschichtung das UV-Licht blockiert, kann dieses nicht mehr auf die Zinkplatte treffen und der photoelektrische Effekt bleibt aus.
\begin{figure}[H]
    \centering
    \includegraphics[width=0.5\linewidth]{figures/hallwachs_versuch.png}
    \caption{Schematische Darstellung des Hallwachs-Versuchs.}
    \label{fig:hallwachs_versuch}
\end{figure}

\subsubsection{Jönsson-Experiment und Elektronenbeugungsröhre}
\label{sec:elektronenbeugungsroehre}
Das Jönsson-Experiment, durchgeführt von Claus Jönsson im Jahr 1961, ist eine wegweisende Demonstration des Doppelspaltexperiments mit Elektronen. Es zeigte erstmals, dass Elektronen, die einzeln durch einen Doppelspalt geschickt werden, auf einem dahinterliegenden Schirm ein Interferenzmuster erzeugen, ähnlich dem, das bei Lichtwellen beobachtet wird. Dies belegt die Wellennatur von Materie im Allgemeinen und die Welle-Teilchen-Dualität von Elektronen. Obwohl jedes Elektron als einzelnes Teilchen detektiert wird, entsteht das Muster nur durch die Überlagerung der Wahrscheinlichkeitswellen des Elektrons, das scheinbar beide Spalte gleichzeitig durchquert.

Eine Elektronenbeugungsröhre besteht aus einer Elektronenquelle, einer Graphitfolie und einem Fluoreszenzschirm. Die Elektronen werden durch die Graphitfolie gebeugt und treffen auf dem Fluoreszenzschirm auf. Auch mit diesem Expeiment kann die Wellennatur von Elektronen beobachtet werden.
\begin{figure}[H]
    \centering
    \includegraphics[width=0.9\linewidth]{figures/jönsson_doppelspalt.jpg}
    \caption{Schematische Darstellung einer Elektronenbeugungsröhre.}
    \label{fig:joensson_doppelspalt}
\end{figure}
Durch Glühemission werden Elektronen freigesetzt und mit einer Anodenspannung \(U_{a}\) (meist bis 5 kV) beschleunigt. Ein dünner Strahl trifft auf eine polykristalline Graphitschicht. Graphit dient hier als Beugungsgitter mit zwei Gitterkonstanten, wobei die Abstände der Atome (Netzebenen) in der Größenordnung der Wellenlänge liegen (\(d_{1}\approx 213\,\si{\pico\meter}\), \(d_{2}\approx 123\,\si{\pico\meter}\)). Die gebeugten Elektronen erzeugen auf einem Fluoreszenzschirm ein Muster aus konzentrischen Ringen.
\begin{figure}[H]
    \centering
    \includegraphics[width=0.6\linewidth]{figures/graphitfolie.png}
    \caption{Schematische Darstellung einer Graphitschicht.}
    \label{fig:graphitfolie}
\end{figure}
Die Bragg-Bedingung beschreibt die Bedingungen, unter denen Wellen (wie Röntgenstrahlen oder Elektronenwellen) an einem Kristallgitter konstruktiv interferieren, also ein Maximum (einen hellen Punkt oder Ring) bilden.
In der Elektronenbeugungsröhre dient sie dazu, aus dem Radius der beobachteten Ringe auf die Wellenlänge der Elektronen oder die Gitterabstände im Graphit zu schließen.
\begin{gather}
    k \cdot \lambda = 2d \cdot \sin\left(\theta\right)
\end{gather}

\paragraph{Weitere Versuche}
\begin{itemize}
    \item Interferenz bei polychromatischem Licht am Overhead-Projektor
    \item Bestimmung der Breite eines Haares
    \item Bestimmung der Breite der Rillen einer CD
\end{itemize}

\emph{Weitere Versuche hier einfügen.}


\newpage

% --- Abschluss ---

\end{document}
