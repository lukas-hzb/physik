% --- Grundlegende Pakete ---
\usepackage[utf8]{inputenc}    % Zeichencodierung
\usepackage[T1]{fontenc}       % Schriftart-Kodierung
\usepackage[ngerman]{babel}    % Deutsche Sprache
\usepackage{csquotes}
\usepackage{lmodern}           % Moderne Schriftart
\usepackage{microtype}         % Mikrotypografie-Verbesserungen
\usepackage{parskip}           % Absätze durch Leerzeilen statt Einrückung
\usepackage[margin=2.5cm]{geometry} % Seitenränder

% --- Weitere nützliche Pakete ---
\usepackage{amsmath, amssymb}  % Mathematische Symbole und Formeln
\usepackage{physics}
\usepackage{siunitx}
\usepackage{enumitem}          % Weitere Art der Aufzählung
\usepackage{booktabs}          % Schöne Tabellen
\usepackage{graphicx}          % Bilder einbinden
\usepackage{xcolor}            % Farbige Schrift
\PassOptionsToPackage{hyphens}{url}\usepackage{hyperref} % Hyperlinks im PDF

% --- Zusätzliche Pakete für verbesserte Darstellung ---
\usepackage{tikz}              % Für Diagramme und Zeichnungen
\usetikzlibrary{calc}          % Für komplexe Berechnungen in TikZ
\usepackage{tcolorbox}         % Für farbige Boxen
\usepackage{multicol}          % Für mehrspaltige Layouts
\usepackage{float}             % Für eine bessere Platzierung von Figuren
\usepackage{xurl}              % Links werden immer gebrochen

% --- Anpassungen ---
\hypersetup{
    unicode=true,        % Allows non-ASCII characters in bookmarks
    colorlinks=true,
    linkcolor=blue!50!black,
    filecolor=blue,
    urlcolor=blue,
    citecolor=blue,
    pdftitle={Physik Lernzettel Klausur 4 - Magnetismus},
    pdfauthor={Lukas Harzbecker}
}

\setlist[itemize]{label=--, left=0.55em}    % Alternative Listen

\setlength{\parindent}{0pt}    % Keine Einrückung bei Absätzen

% --- Kopf- und Fußzeilen ---
\usepackage{fancyhdr}

\fancypagestyle{titleandtocstyle}{
    \fancyhf{}
    \fancyfoot[C]{} % Leerer Fußbereich für keine Seitenzahl
    \renewcommand{\headrulewidth}{0pt}
    \renewcommand{\footrulewidth}{0pt}
}

\fancypagestyle{contentstyle}{
    \fancyhf{}
    \fancyhead[L]{Lukas Harzbecker}
    \fancyhead[C]{Physik-LK -- Alle Themen}
    \fancyhead[R]{\rightmark}
    \fancyfoot[C]{\thepage}
    \renewcommand{\headrulewidth}{0.5pt}
    \renewcommand{\footrulewidth}{0pt}
}

% Section-Kurztitel fuer Header setzen (ohne Nummer)
\renewcommand{\sectionmark}[1]{\markright{#1}}
\renewcommand{\subsectionmark}[1]{}
\renewcommand{\subsubsectionmark}[1]{}

% --- Titelseite ---
\author{Lukas Harzbecker}
\date{\today}
