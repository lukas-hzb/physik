\documentclass[tikz, border=10pt]{standalone}
\usepackage{pgfplots}
\pgfplotsset{compat=1.18}
\renewcommand{\familydefault}{\sfdefault}
\usepackage[utf8]{inputenc}
\usepackage[T1]{fontenc}
\usepackage[ngerman]{babel}
\usepackage{amsmath}

% TikZ Einstellungen
\usetikzlibrary{calc, positioning, decorations.pathreplacing, shadows, arrows.meta}

% --- Farbpalette ---
\definecolor{colImitat}{RGB}{179, 217, 255}     
\definecolor{colBombe}{RGB}{255, 179, 179}      
\definecolor{colAbsorbiert}{RGB}{120, 120, 120} 
\definecolor{colSuccess}{RGB}{255, 165, 0}      

\begin{document}

\begin{tikzpicture}[
    yscale=1.3, % Skalierung aus deinem Code übernommen
    font=\small,
    boxlabel/.style={
        align=center, 
        anchor=center,
        text=black
    },
    problabel/.style={
        align=center,
        fill=white,
        inner sep=1pt,
        text=black!80,
        font=\footnotesize\itshape,
        rounded corners=2pt,
        fill opacity=0.8,
        text opacity=1
    }
]

    % --- Konfiguration der Geometrie ---
    \def\wBox{2.5}  
    \def\wGap{2.5}  
    
    % ==============================================
    % 1. START-SÄULEN
    % ==============================================
    \fill[colImitat] (0, 4) rectangle (\wBox, 8);
    \node[boxlabel] at (0.5*\wBox, 6) {\textbf{Imitat}\\(50\%)};

    \fill[colBombe] (0, 0) rectangle (\wBox, 4);
    \node[boxlabel] at (0.5*\wBox, 2) {\textbf{Scharfe}\\\textbf{Bombe}\\(50\%)};

    % ==============================================
    % 2. VERBINDUNG ZUR MITTE
    % ==============================================
    % Imitat
    \fill[colImitat!60] (\wBox, 4) -- (\wBox+\wGap, 4) -- (\wBox+\wGap, 8) -- (\wBox, 8) -- cycle;
    \node[problabel] at (\wBox+0.5*\wGap, 6) {100\% ohne\\eindeutigen\\Weg};

    % Bombe -> Absorption
    \fill[colAbsorbiert!60] (\wBox, 0) -- (\wBox+\wGap, 0) -- (\wBox+\wGap, 2) -- (\wBox, 2) -- cycle;
    \node[problabel] at (\wBox+0.5*\wGap, 1) {50\% durch\\Bombe};

    % Bombe -> Frei
    \fill[colBombe!60] (\wBox, 2) -- (\wBox+\wGap, 2) -- (\wBox+\wGap, 4) -- (\wBox, 4) -- cycle;
    \node[problabel] at (\wBox+0.5*\wGap, 3) {50\% durch\\freien Weg};

    % ==============================================
    % 3. MITTLERE SÄULEN
    % ==============================================
    \fill[colImitat] (\wBox+\wGap, 4) rectangle (2*\wBox+\wGap, 8);
    \node[boxlabel, font=\footnotesize] at (1.5*\wBox+\wGap, 6) {Verhält sich\\wie leeres\\Interferometer};

    \fill[colBombe] (\wBox+\wGap, 2) rectangle (2*\wBox+\wGap, 4);
    \node[boxlabel, font=\footnotesize] at (1.5*\wBox+\wGap, 3) {Wellenfunktion\\kollabiert};

    \fill[colAbsorbiert] (\wBox+\wGap, 0) rectangle (2*\wBox+\wGap, 2);
    \node[boxlabel, color=white, font=\footnotesize] at (1.5*\wBox+\wGap, 1) {Wellenfunktion\\kollabiert\\und\\EXPLOSION};

    % ==============================================
    % 4. VERBINDUNG ZUM ZIEL
    % ==============================================
    % Imitat -> Rechts
    \fill[colImitat!60] (2*\wBox+\wGap, 4) -- (3*\wBox+2*\wGap, 4) -- (3*\wBox+2*\wGap, 8) -- (2*\wBox+\wGap, 8) -- cycle;
    \node[problabel] at (2.5*\wBox+1.5*\wGap, 6) {100\% unbestimmt};

    % Bombe (Frei) -> Rechts
    \fill[colBombe!60] (2*\wBox+\wGap, 3) -- (3*\wBox+2*\wGap, 3) -- (3*\wBox+2*\wGap, 4) -- (2*\wBox+\wGap, 4) -- cycle;
    \node[problabel] at (2.5*\wBox+1.5*\wGap, 3.5) {50\% unbestimmt};

    % Bombe (Frei) -> Oben (Detektor)
    \fill[colSuccess!60] (2*\wBox+\wGap, 2) -- (3*\wBox+2*\wGap, 0.5) -- (3*\wBox+2*\wGap, 1.5) -- (2*\wBox+\wGap, 3) -- cycle;
    \node[problabel] at (2.5*\wBox+1.5*\wGap, 1.75) {50\% erkannt};

    % ==============================================
    % 5. ZIEL-SÄULEN
    % ==============================================
    
    % --- Detektor RECHTS ---
    \draw[gray, fill=white] (3*\wBox+2*\wGap, 2.8) rectangle (4*\wBox+2*\wGap, 8.2);
    \fill[colImitat] (3*\wBox+2*\wGap, 4) rectangle (4*\wBox+2*\wGap, 8);
    \fill[colImitat] (3*\wBox+2*\wGap, 3) rectangle (4*\wBox+2*\wGap, 4); 
    
    \node[boxlabel, align=center] at (3.5*\wBox+2*\wGap, 5.4) {
        \textbf{RECHTS}\\
        (62,5\%)\\
        \footnotesize Keine Aussage\\möglich
    };

    % --- Detektor OBEN ---
    \draw[gray, fill=white] (3*\wBox+2*\wGap, 0.3) rectangle (4*\wBox+2*\wGap, 1.7);
    \fill[colSuccess!85] (3*\wBox+2*\wGap, 0.5) rectangle (4*\wBox+2*\wGap, 1.5);
    
    \node[boxlabel, align=center] at (3.5*\wBox+2*\wGap, 1) {
        \textbf{OBEN}\\ 
        (12,5\%)\\
        \footnotesize Bombe erkannt
    };
    
    % Beschriftungen
    \node[anchor=south, gray, font=\scshape] at (0.5*\wBox, 8.2) {Objekt};
    \node[anchor=south, gray, font=\scshape] at (1.5*\wBox+\wGap, 8.2) {Interferometer};
    \node[anchor=south, gray, font=\scshape] at (3.5*\wBox+2*\wGap, 8.2) {Detektor};

\end{tikzpicture}

\end{document}